\chapter{Results on Complexity of \cbis}
In this chapter, we mention our major results on the complexity 
of the \ccsp(\mrelset) problem. First, we show that for monotone relations, there is an
approximation preserving reduction from the \ccsp(\mrelset) problem to the \cbis\ problem.
Next, we show a connection between the complexity of the \chom(\mH) problem on bipartite graphs
and directed graphs.
At the end, we show that there is an approximation preserving reduction from the \cbis\ problem 
to the \ccsp(\mrelset) problem if \mrelset\ consists of an \emph{\RBA} relation.

For constraint languages consisting of a binary relation, the \ccsp(\mrelset) problem is
the same as the \chom(\mH) problem; hence, we use them interchangeably depending on the on the context.

%%%%%%%%%%%%%%%%%%%%%%%%%%%%%%%%%%%%%%%%%%%%%%%%%%%%%%%%%%%%%%%%%%%%%%%%%%%%%%%%%%%%%%%%%%%%%%%
% Monotone Graphs
%%%%%%%%%%%%%%%%%%%%%%%%%%%%%%%%%%%%%%%%%%%%%%%%%%%%%%%%%%%%%%%%%%%%%%%%%%%%%%%%%%%%%%%%%%%%%%%
\section{Monotone Relations}
In this section, we extend the definition of \emph{monotone} constraint languages defined in
\cite{Trichotomy} from Boolean domains to multi-valued domains and show that for
any monotone relation \mrelset\ there is an AP-reduction from the \ccsp(\mrelset) problem to
the \cbis\ problem.

\begin{defi}
Let \mrelset\ be a constraint language over the domain \mD; \mrelset\ is \emph{monotone}
if there exists distributive lattice \(L\) with over universe \mD\ such that every relation 
in \mrelset\ is closed under operations \(vee\) and \(\wedge\) defined by \(L\)\@.
\end{defi}

After studying the complexity of the \ccsp(\mrelset) problem for maximal clones on 3-element sets
and the \chom(\mH) problem on 3-vertex graphs, we observed that \ccsp(\mrelset) over 
monotone relations are AP-reducible to the \cds\ problem. We generalize them by
Theorem~\ref{theorem:monotone}\@.

\begin{theorem} \label{theorem:monotone}
Let \mrelset\ be a constraint language over a domain \mD, if 
\mrelset\ is monotone then \ccsp(\mrelset) \maple\ \cbis\@.
\end{theorem}

\begin{proof}
For any monotone constraint language \mrelset, there is a distributive lattice \(L\) with operations
\(\vee\) and \(\wedge\)\@. By the Birkhoff's Representation Theorem, every finite distributive lattice
is isomorphic to a lattice ordered Boolean vector space of dimension \(k\)\@.
Denote the isomorphism functions by \(\cube: D \to \setof{0,1}^k\) and 
\(\recube : \setof{0,1}^k\to D\)\@. The definition of the \(\cube\) function
is extended to tuples, as \(\cube(x_1,\dotsc,x_l)=(\cube(x_1),\dotsc,\cube(x_l))\)\@.
The definition of \(\cube\) is further extended to relations as 
\(\cube(R) = \setof{\cube(\varrho) \mid \varrho \in R}\)\@.

Take \(\relset'=\setof{\cube(R)\mid R\in \relset}\)\@. The functions \(\cube\) and
\(\recube\) form a bijection between the solutions for any CSP(\mrelset) instance and the 
corresponding CSP(\(\relset'\)) instance. Hence, there is a parsimonious reduction between 
the \ccsp(\mrelset) problem and the \ccsp(\(\relset'\)) problem. Now, we prove that
\(\relset'\) is monotone. For any relation \(R' \in \relset'\) we have:
\[\vx \in R \Leftrightarrow \cube(\vx) \in R'\]
We show that for any two tuples \(\vx,\vy\in R'\) we have 
\(\vx \vee \vy, \vx\wedge\vy\in R'\).
\begin{eqnarray*}
\vx,\vy \in R'  \\
\recube(\vx),\recube(\vy)\in R  \\
\recube(\vx) \vee \recube(\vy) \in R  \\
\cube(\recube(\vx)\wedge\recube(\vy))\in R' \\
\vx \wedge \vy \in R' \\
\end{eqnarray*}
Similarly \(\vx \vee \vy \in R'\)\@.
\(\relset'\) is monotone and by Theorem~\ref{theorem:thrichotomy}, we have 
\ccsp(\(\relset'\))\maple\ \cbis\@. Consequently, \ccsp(\mrelset) \maple\ \cbis\@.
\end{proof}

Originally, we came up with a different reduction from the \ccsp(\(\relset'\)) problem
to the \cds\ problem. However, the proof used in \cite{Thrichotomy} for
Theorem~\ref{theorem:thrichotomy} is simpler than our proof; hence, there is need to mention it.
\begin{comment}
The reduction uses Theorem~\ref{theorem:majority} to break arbitrary relation to binary relation.
A function \(m: D^3\to D\) is said to be a \emph{majority function}, if for any \(x,y\in D\) we have
\(m(x,x,y) = m(x,y,x) = m(y, x, x) = x\)\@. We denote the projection of \mR\ on coordinates \(i\) and \(j\) by \(R_{1,2}\) and projections of \mR\ on coordinates from \(i\) to \(j\) by \(R_{i-j}\)\@.


\begin{theorem} [Bergman's Double-projection Theorem (Unpublished)]\label{theorem:majority}
Let \mR\ be a relation of arity \mn\ on domain \(D\) and
\(m\) be a majority function on the same domain.
If \mm\ is a polymorphism for \mR, then \mR\ can be expressed by all 
its binary projections as \(R(x_1,\dotsc,x_n) = \bigwedge_{i,j} R_{i,j}(x_i,x_j)\)\@.
\end{theorem}

We provide our own proof for this theorem.

\begin{proof}
It is trivial that \mR\ implies all its binary projection. We  prove that
binary projections of \mR\ implies \mR\@.
We proceed by induction on \(t\) the arity of the relation \mR\@.
For \(t=2\), the statement is trivial. We show that
if the statement is true for \(t=n-1\) it is also true for \(t=n\)\@.
We want to prove \(\bigwedge_{i,j} R_{i,j}(x_i,x_j) \implies R(x_1,\dotsc, x_n)\).
By induction hypothesis \(R_{1-(n-1)}(x_1,\dotsc,x_{n-1})\) and \(R_{2-n}(x_2,\dots,x_n)\)
is true because \(R_{1-(n-1)}\) and \(R_{2-n})\) are closed under \mm\@.
By hypothesis, \(R_{1,n}(x_1,x_n)\) is also true. Since these relations are projection of \mR,
there are \(y_1,\dotsc,y_n \in D\) such that 
\(R(x_1,\dotsc,x_{n-1},y_n)\), \(R(y_1,x_2,\dotsc,x_n)\), and \(R(x_1,y_2,\dotsc,y_{n-1},x_n)\)
are true. Since \mm\ is a polymorphism of \mR, \(R(x_1,\dotsc,x_n)\) is true.
\end{proof}

For every monotone
relation, we can find a majority projection using \(\vee\) and \(\wedge\) operators. One
example would be \(m(x,y,z)=(x \vee y) \wedge (x \vee z) \wedge (y \vee z)\)\@. 
Thus, every instance of the \ccsp(\(\relset'\)) problem can be parsimoniously reduced to
a \ccsp(\\)) instance, where \(\relset''\) is the binary monotone constraint language.
Finally, \ccsp(\(\relset''\)
\end{comment}
%%%%%%%%%%%%%%%%%%%%%%%%%%%%%%%%%%%%%%%%%%%%%%%%%%%%%%%%%%%%%%%%%%%%%%%%%%%%%%%%%%%%%%%%%%%%%%%
% BiMonotone Graphs
%%%%%%%%%%%%%%%%%%%%%%%%%%%%%%%%%%%%%%%%%%%%%%%%%%%%%%%%%%%%%%%%%%%%%%%%%%%%%%%%%%%%%%%%%%%%%%%

\section{Bimonotone Graphs}
In this part, we extend the results from the previous section to bipartite
graphs and show that the for any \emph{bimonotone} graph \mH, there is AP-reduction from the \chom(H)
problem to the \cbis\ problem.

\begin{theorem} [Bipartite Orientation] \label{theorem:bior}
Let \mH\ be a bipartite graph and let \(A\) and \(B\) be an arbitrary
bipartition of \mH\@. For \(H'\) the directed graph obtained from
\mH\ by orienting the edges from \(A\) to \(B\), we have \chom(\mH) \maple \chom(\(H'\)).
\end{theorem}

\begin{proof}
Let \mG\ be an input graph for the \chom(\mH) problem. If \mG\ is not bipartite then 
\(\hom(G,H)=0\). Restrict \mG\ to bipartite graphs also restrict \mG\ to connected graphs
by Lemma~\ref{lem:connected}\@. Let \(X\) and \(Y\) be bipartition of \mG\ and let \(G_1\) 
and \(G_2\) be the graphs obtained from \mG\ by orienting all edges from \(X\) to \(Y\) and
from \(Y\) to \(X\), receptively. We claim that \(\hom(G,H) = \hom(G_1, H') + \hom(G_2, H')\).
Every homomorphism from \mG\ to \mH\ has to either map all the vertices in
\(X\) to \(A\) and all the vertices in \(Y\) to \(B\) or vice-versa. The homomorphisms that
map \(X\) to \(A\) and \(Y\) to \(B\) are exactly the same as homomorphisms from \(G_1\) to
\(H'\) and similarly the homomorphisms that map \(Y\) to \(A\) and \(X\) to \(B\)
are exactly the same as homomorphisms from \(G_2\) to \(H'\)\@. This implies that
\chom(\mH) \maple\ \chom(\(H'\))\@.
\end{proof}

Typically bipartite graphs are not monotone; however, if they are oriented as above,
we may get monotone graph.

\begin{defi} [Bimonotone Graphs]
Let \(H\) be a bipartite graph and let \(A\) and \(B\) be an arbitrary
bipartition of \(H\). Let \(H'\) be the directed graph obtained from
\(H\) by orienting the edges from \(A\) to \(B\)\@. The graph \mH\ is said to be \emph{bimonotone}
if \(H'\) is monotone.
\end{defi}

Note that, choice for bipartition does not affect \(H'\) being monotone or not.

\begin{theorem} [Bimonotone]
For any bimonotone graph \mH, we have \chom(\mH) \maple \cbis\@.
\end{theorem}

\begin{proof}
For any bimonotone graph \mH, by Theorem~\ref{theorem:bior} there is a directed monotone 
graph \(H'\) such that \chom(\mH) \maple \chom(\(H'\))\@. By Theorem~\ref{theorem:monotone},
\chom(\(H'\) \maple \cbis\@. Thus, \chom(\mH) \maple \cbis\@.
\end{proof}
%%%%%%%%%%%%%%%%%%%%%%%%%%%%%%%%%%%%%%%%%%%%%%%%%%%%%%%%%%%%%%%%%%%%%%%%%%%%%%%%%%%%%%%%%%%%%%%
% RBA Relations
%%%%%%%%%%%%%%%%%%%%%%%%%%%%%%%%%%%%%%%%%%%%%%%%%%%%%%%%%%%%%%%%%%%%%%%%%%%%%%%%%%%%%%%%%%%%%%%

\section{RBA Relations}
In this section we will introduce a family of relations called \RBA\ relations and show that
approximately solving the \ccsp(\mrelset) problem over an \RBA\ relations is harder than
approximately solving the \cbis problem. These relations
can be seen as graphs; since most of the proofs are better expressed by graphs,
we will mostly use the graph notations.

\begin{defi} [\RBA]
A relations is \RBA\ if it is binary, reflexive, and asymmetric.
\end{defi} 

As digraphs, the \RBA\ relations are reflexive oriented digraphs. Let \(H=(V,E)\) be 
an \RBA\ graph; for each \(v \in V\), we have \((v,v) \in E\)
and \((u,v) \in E\) and \((v,u) \in E\) implies \(u=v\)\@.

\begin{figure}[h]
\center{\input{figs/rba.pdftex}}
\caption{An RBA relation}
\end{figure}

Let \(\NHab Hab\) be vertices that are both in out-neighbours of \(a\) and in-neighbours of \(b\).
Note that if \mH\ is reflexive and \((a,b)\in E(H)\),
\(a,b\in \NHab Hab\)\@. Let \(\Hab\) be the subgraph of \mH\ induced by \(\NHab Hab\),
\emph{Magnitude} of \mH\ represented as \(\magf(H)\) is defined as
\(\magf(H) = \max \limits_{(a,b) \in E(H)} \setof{|\NHab Hab}|\)\@.

\begin{figure}[h]
\center{\input{figs/shade2.pdftex}}
\caption{An RBA graph \mH\ such that \(\magf(H)=2\)}
\end{figure}

\begin{lemma} \label{lem:triangle-free}
For an \RBA\ graph \mH, if \(\magf(H)=2\) then \chom(\mH) \mapge\ \cbis\@.
\end{lemma}

\begin{proof}
Let \mH\ be a an \RBA\ graph such that \(\magf(H)=2\), 
we reduce \cds\ problem to \chom(\mH)\@.
For given partial order \((P,\preceq)\), let \(G=(V,E)\) be the corresponding directed (acyclic) graph.
Take \(G'=(V',E')\) as \(V'= V \cup \setof{u,v}\) and \(E' = E \cup \setof{(u,v)} \cup
\setof{(u,x),(x,v) \mid x\in V}\)\@.

For any homomorphism \(h: G' \to H\), if \(h(u)=h(v)=s\), then \(h(x)=s\) for all \(x\in V\);
otherwise, \((h(u),h(v))\in E(H)\) and \(h\) corresponds to a DownSet in \((P,\preceq)\)\@.

The number of homomorphism with \(h(u)=h(v)\) is equal to \(|V(H)|\) and
the number of homomorphism with \(h(u)\neq h(v)\) is equal to \(|E(H)|\) times the 
number of DownSets in \((P,\preceq)\)\@. Hence,
\[\hom(G',H) = |V(H)| + |E(H)|\cdot \#ds(P,\preceq).\]
This implies that \chom(\mH) \mapge\ \cds\@.
\end{proof}

\begin{defi} [Polar Graph]
Let \mH\ be an \RBA\ graph. \mH\ is \emph{polar}
if there exists vertices \(a,b\in V(H)\),
such that \((a,x), (x,b) \in E(H)\) for all vertices \(x \in V(H)\) (including \(a\) and \(b\)).
\end{defi}

Note that, for a polar graph \mH, we have \(\NHab Hab = V(H)\) and \(\magf(H)=|V(H)|\)\@.

\begin{figure}[h]
\center{\input{figs/polar.pdftex}}
\caption{A polar graph}
\end{figure}

\begin{lemma} \label{lem:k-fixing}
For any \RBA\ graph \mH\ that \(k(H) > 2\), there is an \RBA\ graph
\(H'\) such that  each connected component of \(H'\) is a
polar graph, \(\magf(H) = \magf(H')\), and
\chom(\mH) \mapge\ \chom(\(H'\)).
\end{lemma}

\begin{proof}
Let \(G=(V,E)\) be a graph and \(Y\) be a set of \(t\) fresh vertices
(the value of \(t\) will be determined later), and let \(G'=(V',E')\) be
graph such that \(V'=V \cup Y \cup \setof{u,v}\) and \(E'=E \cup \setof{(u,v)} \cup
\setof{(u,x),(x,v) \mid x\in V \cup Y}\)\@.

For any two vertices \(a,b \in V(H)\) that \((a,b)\in E(H)\), consider a homomorphisms
\(h: G'\to H\) that \(h(u)=a\) and \(h(v)=b\). Since for any vertex \(x\) in \(V\) or \(Y\),
\((u,x)\) and \((x,v)\) are edges in \(G'\), as a result \(x\) has to be mapped to \(\NHab Hab\)\@.
Any vertex in \(Y\) can be freely mapped to any vertex in \(\NHab Hab\)\@. Thus, we have:
\begin{eqnarray*}
\hom(G',H) &=&  
\sum_{(a,b)\in E(H)}\left|\NHab Hab\right|^ t \cdot
\hom(G,\Hab)\\
&=& 
\bigsum{\substack{a,b\in V(H) \\
\magf(H_{a,b})=\magf(H)}} 
\magf^t(H) \cdot \hom(G,H_{a,b})+
\bigsum{\substack{a,b\in V(H) \\ \magf(H_{a,b}) < \magf(H)}}k^t
(H_{a,b})\cdot \hom(G,H_{a,b})
\end{eqnarray*}

Divide both sides of the formula by \(k^t(H)\)\@. We have
\[\frac{\chom(G',H)}{\magf^t(H)} =  
\bigsum{\substack{a,b\in V(H) \\ \magf(H_{a,b})=\magf(H)}} 
\hom(G',H_{a,b}) + 
\bigsum{\substack{a,b\in V(H) \\ \magf(H_{a,b})<\magf(H)}}
\left(\frac{\magf(H_{a,b})}{\magf(H)} \right )^t
\cdot\hom(G',H_{a,b})
\]
Let \(n=|V|\) and let \(m=|V(H)|\). We have \(\frac{\magf(\Hab)}{\magf(H)} \le \frac{m-1}{m}\)
and \(\hom(G,H_{a,b}) \le n^m\). \((\frac{m-1}{m})^t \approx e^{\frac{-t}{m}}\)
and \(n^m = e^{m\log n}\)\@. The number of such tuples is at most \(n^2 = e^{2\log n}\).
For \(t > (m^2+2m) \log n\), the second part of the summation is less the \(1\).
Thus, we have a reduction from \chom(\mH) to \chom(\(H'\)), where \(H'\) is defined as:
\[H'=\bigcup_{\substack{a,b\in V(H) \\
\magf(H_{a,b})=\magf(H)}} H_{a,b}.\]
\(H'\) is an \RBA\ graph and each component of \(H'\) 
is \(H_{a,b}\) for some \(a\) and \(b\) hence satisfies the conditions.
\end{proof}

\begin{lemma} \label{lem:connected}
For an \RBA\ graph \mH, if every connected component of 
\mH\ is a polar graph then there is a component \(H'\) of \mH\ 
such that \chom(\mH) \mapge \chom(\(H'\)).
\end{lemma}

\begin{proof}
There are two cases:
\begin{itemize}
\item All components of \mH\ are isomorphic. Let \(H'\) be a connected component of \mH\ and
\mm\ be the number of components of \mH\@. For any graph \mG\ as input for the
\chom(\mH) problem, \(\hom(G,H)=\hom^m(G,H')\). Hence, \chom(H) \mapge \chom(\(H'\))\@.
\item Let \(H_1\) and \(H_2\) be two connected components for \mH\ that are not isomorphic.
By Lov\'{a}sz's Theorem there is a connected graph \(Z\) such that \(\hom(Z,H_1) \neq \hom(Z,H_2)\)\@.
Let \(Z_1,\dotsc,Z_t\) be \(t\) fresh copies of \(Z\), where \(t\) is a large number to be 
define later.
Let \(G=(V',E')\) be a graph with \(V'=V \cup V(Z_1)\cup \dotsb \cup V(Z_t) \cup \setof{u,v} \)
and \(E'=E \cup \setof{(u,v)} \cup \setof{(u,x),(x,v) \mid x \in V \cup  Z_1 \cup \dotsb \cup Z_t} \cup
E(Z_1)\cup \dotsb \cup E(Z_t)\)\@. Let \(q\) be \(\max_{a,b \in V(H)} \hom(Z,H_{a,b})\) and
let \(H'=\bigcup_{\substack{a,b\in V(H) \\ 
\hom(G,H_{a,b})=q}} H_{a,b}\)


For any homomorphism \(h: G'\to H\), consider \(a=h(u)\)  and \(b=h(v)\). We have:
\begin{eqnarray*}
\hom(G',H) &=&
\sum_{a,b\in V(H)} \hom(G,H_{a,b}) \cdot \hom^t(Z, H_{a,b}) \\
&=&
\bigsum{\substack{a,b\in V(H)\\ \hom(Z, H_{a,b}) < q}} \hom(G, H_{a,b}) \cdot q^t + 
\bigsum{\substack{a,b\in V(H)\\ \hom(Z, H_{a,b}) < q}} 
\hom(G, H_{a,b}) \cdot \hom^t(Z,H_{a,b})\\
&=&
\hom(G',H') \cdot q^t + \bigsum{\substack{a,b\in V(H)\\ \hom(Z, H_{a,b}) < q}} 
\hom(G, H_{a,b}) \cdot \hom^t(Z,H_{a,b})
\end{eqnarray*}
Divide both sides by \(q^t\), and let \(n=|V|\) and \(m=\magf(H)\)\@. We have:
\[
\frac{\hom(G',H)}{q^t} = 
\hom(G',H') + \hom(G, H_{a,b}) \cdot 
\bigsum{\substack{a,b\in V(H)\\ \hom(Z, H_{a,b}) < q}} 
\left(\frac{\hom(Z,H_{a,b})}{q}\right)^t 
\]
The second part of right hand side can be bounded as:
\begin{eqnarray*}
\bigsum{\substack{a,b\in V(H)\\ \hom(Z, H_{a,b}) < q}} 
\left(\frac{\hom(Z,H_{a,b})}{q}\right)^t  &\le &
\sum_{a,b\in V(H)} m\cdot \left(1-\frac{1}{q}\right)^t \\
&\approx & n^2 \cdot m \cdot e^{\frac{t}{q}} \\
\end{eqnarray*}
Similar evaluation indicates that
for \(t=2q\log(nm)\) the right hand is smaller than \(1\); hence, \chom(\mH) \mapge\ \chom(\(H'\)). 
Now \(H'\) consists of some of the components of \mH; reconsidering cases results a single component
of \mH\@.
\end{itemize}
\end{proof}

By combining Lemme \ref{lem:connected} and \ref{lem:k-fixing} 
we derive the following lemma:\todo{??}

\begin{lemma} \label{lem:k-reduction}
Let \mH\ be a polar graph, if \(\magf(H) > 2\), then
there is an \RBA\ graph \(H'\) such that \(\magf(H') < \magf(H)\)
and \chom(\mH) \mapge\ \chom(\(H'\))\@.
\end{lemma}

\begin{proof}
Let \(a\) be a vertex in \mH\ such that for all \(x\in V(G)\), \((a,x)\in E(H)\) and
let \(S = V(H) \setminus \setof a\).
For any other vertex \(w \in S\), we have \(d^-(w) > 1\); on the other hand \(d^-(a)=1\),
by Lemma~\ref{lem:pinning}, \chom(\mH) \mapge\ \chom(\(H'\)) where \(H'\) is 
the induced sub-graph by \(S\)\@.
\end{proof}

\begin{theorem}
For every non-empty \RBA\ graph \mH, \chom(\mH) \mapge \cbis\@.
\end{theorem}


\begin{proof}
By contradiction, consider a \RBA\ graph \mH\ with the least magnitude
such that \mH\ is not \cbis -hard. By Lemma~\ref{lem:triangle-free}, \(\magf(H) > 2\)\@. 
By lemma~\ref{lem:k-fixing}, there exists a \RBA\ graph \(H'\) such that 
\chom(\(H_1\)) \maple \chom(\mH) and  
and \(\magf(H)=\magf(H_1)\) and each component of \(H_1\) is a polar graph. 
By lemma \ref{lem:connected}, there is a connected component \(H_2\)
of \(H_1\) such that \chom(\(H_2\)) \maple\ \chom(\(H_1\)) \maple\ \chom(\mH)\@.
Finally  by Lemma~\ref{k-reduction}, there exists an \RBA\ graph \(H_3\) such that 
\chom(\(R_3\)) \maple\ \chom(\(H_2\)) \maple\ \chom(\mH) 
and \(\magf(H_3) < \magf(H)\); this contradicts with \mH\ having the minimum magnitude.
\end{proof}

\section{Conclusion and Future works}
TODO