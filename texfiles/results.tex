\chapter{Results on Complexity of \cbis} \label{chp:results}
In this chapter, we mention our major results on complexity 
of the \ccsp(\mrelset) problem. We first show that for \emph{monotone} relations, there is an
approximation preserving reduction from the \ccsp(\mrelset) problem to the \cbis\ problem.
Next, we extend the definition of monotone graphs to bipartite graphs. We also provide a characterization of monotone graph for reflexive graphs.
At the end, we show that there is an approximation preserving reduction from the \cbis\ problem 
to the \ccsp(\mrelset) problem if \mrelset\ consists of an \emph{\RBA} relation.

For constraint languages consisting of a binary relation, the \ccsp(\mrelset) problem is
the same as the \chom(\mH) problem; hence, we use them interchangeably depending on the on the context.

%%%%%%%%%%%%%%%%%%%%%%%%%%%%%%%%%%%%%%%%%%%%%%%%%%%%%%%%%%%%%%%%%%%%%%%%%%%%%%%%%%%%%%%%%%%%%%%
% Monotone Graphs
%%%%%%%%%%%%%%%%%%%%%%%%%%%%%%%%%%%%%%%%%%%%%%%%%%%%%%%%%%%%%%%%%%%%%%%%%%%%%%%%%%%%%%%%%%%%%%%
\section{Monotone Relations}
In this section, we extend the definition of \emph{monotone} constraint languages defined in
\cite{Trichotomy} from Boolean domains to multi-valued domains and show that for
any monotone relation \mrelset\ there is an AP-reduction from the \ccsp(\mrelset) problem to
the \cbis\ problem.

\begin{defi}
Let \mrelset\ be a constraint language over the domain \mD; \mrelset\ is \emph{monotone}
if there exists distributive lattice \(L\) with over universe \mD\ such that every relation 
in \mrelset\ is closed under operations \(vee\) and \(\wedge\) defined by \(L\)\@.
\end{defi}

After studying complexity of the \ccsp(\mrelset) problem for maximal clones on 3-element sets
in \cite{mvl} also complexity of the \chom(\mH) problem on 3-vertex graphs,
we noticed that \ccsp(\mrelset) problem over 
monotone relations are AP-reducible to the \cdsp\ problem. We generalize the observation by
Theorem~\ref{thrm:monotone}\@.

\begin{theorem} [Hedayaty, Bulatov 2009 Unpublished]\label{thrm:monotone}
Let \mrelset\ be a constraint language over a domain \mD, if 
\mrelset\ is monotone then \ccsp(\mrelset) \maple\ \cbis\@.
\end{theorem}

\begin{proof}
For any monotone constraint language \mrelset, there is a distributive lattice \(L\) with operations
\(\vee\) and \(\wedge\)\@. By the Birkhoff's Representation Theorem, every finite distributive lattice
is isomorphic to a lattice ordered Boolean vector space of dimension \(k\)\@.
Denote the isomorphism functions by \(\cube: D \to \setof{0,1}^k\) and 
\(\recube : \setof{0,1}^k\to D\)\@. The definition of the \(\cube\) function
is extended to tuples, as \(\cube(x_1,\dotsc,x_l)=(\cube(x_1),\dotsc,\cube(x_l))\)\@.
The definition of \(\cube\) is further extended to relations as 
\(\cube(R) = \setof{\cube(\varrho) \mid \varrho \in R}\)\@.

Take \(\relset'=\setof{\cube(R)\mid R\in \relset}\)\@. The functions \(\cube\) and
\(\recube\) form a bijection between the solutions for any CSP(\mrelset) instance and the 
corresponding CSP(\(\relset'\)) instance. Hence, there is a parsimonious reduction between 
the \ccsp(\mrelset) problem and the \ccsp(\(\relset'\)) problem. Now, we prove that
\(\relset'\) is monotone. For any relation \(R' \in \relset'\) we have:
\[\vx \in R \Leftrightarrow \cube(\vx) \in R'\]
We show that for any two tuples \(\vx,\vy\in R'\) we have 
\(\vx \vee \vy, \vx\wedge\vy\in R'\).
\begin{eqnarray*}
\vx,\vy \in R'  \\
\recube(\vx),\recube(\vy)\in R  \\
\recube(\vx) \vee \recube(\vy) \in R  \\
\cube(\recube(\vx)\wedge\recube(\vy))\in R' \\
\vx \wedge \vy \in R' \\
\end{eqnarray*}
Similarly \(\vx \vee \vy \in R'\)\@.
\(\relset'\) is monotone and by Theorem~\ref{thrm:trichotomy}, we have 
\ccsp(\(\relset'\))\maple\ \cbis\@. Consequently, \ccsp(\mrelset) \maple\ \cbis\@.
\end{proof}

Originally, we came up with a different reduction from the \ccsp(\(\relset'\)) problem
to the \cdsp\ problem. However, the proof used in \cite{Trichotomy} for
Theorem~\ref{thrm:trichotomy} is simpler than our proof; hence, there is no need to 
mention our original proof.
\begin{comment}
The reduction uses Theorem~\ref{thrm:majority} to break arbitrary relation to binary relation.
A function \(m: D^3\to D\) is said to be a \emph{majority function}, if for any \(x,y\in D\) we have
\(m(x,x,y) = m(x,y,x) = m(y, x, x) = x\)\@. We denote the projection of \mR\ on coordinates \(i\) and
 \(j\) by \(R_{1,2}\) and projections of \mR\ on coordinates from \(i\) to \(j\) by \(R_{i-j}\)\@.


\begin{theorem} [Bergman's Double-projection Theorem (Unpublished)]\label{thrm:majority}
Let \mR\ be a relation of arity \mn\ on domain \(D\) and
\(m\) be a majority function on the same domain.
If \mm\ is a polymorphism for \mR, then \mR\ can be expressed by all 
its binary projections as \(R(x_1,\dotsc,x_n) = \bigwedge_{i,j} R_{i,j}(x_i,x_j)\)\@.
\end{theorem}

We provide our own proof for this theorem.

\begin{proof}
It is trivial that \mR\ implies all its binary projection. We  prove that
binary projections of \mR\ implies \mR\@.
We proceed by induction on \(t\) the arity of the relation \mR\@.
For \(t=2\), the statement is trivial. We show that
if the statement is true for \(t=n-1\) it is also true for \(t=n\)\@.
We want to prove \(\bigwedge_{i,j} R_{i,j}(x_i,x_j) \implies R(x_1,\dotsc, x_n)\).
By induction hypothesis \(R_{1-(n-1)}(x_1,\dotsc,x_{n-1})\) and \(R_{2-n}(x_2,\dots,x_n)\)
is true because \(R_{1-(n-1)}\) and \(R_{2-n})\) are closed under \mm\@.
By hypothesis, \(R_{1,n}(x_1,x_n)\) is also true. Since these relations are projection of \mR,
there are \(y_1,\dotsc,y_n \in D\) such that 
\(R(x_1,\dotsc,x_{n-1},y_n)\), \(R(y_1,x_2,\dotsc,x_n)\), and \(R(x_1,y_2,\dotsc,y_{n-1},x_n)\)
are true. Since \mm\ is a polymorphism of \mR, \(R(x_1,\dotsc,x_n)\) is true.
\end{proof}

For every monotone
relation, we can find a majority projection using \(\vee\) and \(\wedge\) operators. One
example would be \(m(x,y,z)=(x \vee y) \wedge (x \vee z) \wedge (y \vee z)\)\@. 
Thus, every instance of the \ccsp(\(\relset'\)) problem can be parsimoniously reduced to
a \ccsp(\\)) instance, where \(\relset''\) is the binary monotone constraint language.
Finally, \ccsp(\(\relset''\)
\end{comment}
%%%%%%%%%%%%%%%%%%%%%%%%%%%%%%%%%%%%%%%%%%%%%%%%%%%%%%%%%%%%%%%%%%%%%%%%%%%%%%%%%%%%%%%%%%%%%%%
% Bipartite Monotone Graphs
%%%%%%%%%%%%%%%%%%%%%%%%%%%%%%%%%%%%%%%%%%%%%%%%%%%%%%%%%%%%%%%%%%%%%%%%%%%%%%%%%%%%%%%%%%%%%%%

\section{Bipartite Monotone Graphs}
In this part, we extend the results from the previous section to bipartite
graphs and define monotone bipartite graphs.

\begin{theorem} [Bipartite Orientation] \label{thrm:bior}
Let \mH\ be a bipartite graph and let \(A\) and \(B\) be an arbitrary
bipartitions of \mH\@. For \(H'\) the directed graph obtained from
\mH\ by orienting the edges from \(A\) to \(B\), we have \chom(\mH) \maple \chom(\(H'\)).
\end{theorem}

\begin{proof}
Let \mG\ be an input graph for the \chom(\mH) problem. If \mG\ is not bipartite then 
\(\hom(G,H)=0\). Restrict \mG\ to bipartite graphs also restrict \mG\ to connected graphs
by Lemma~\ref{lem:connected}\@. Let \(X\) and \(Y\) be bipartition of \mG\ and let \(G_1\) 
and \(G_2\) be the graphs obtained from \mG\ by orienting all edges from \(X\) to \(Y\) and
from \(Y\) to \(X\), receptively. We claim that \(\hom(G,H) = \hom(G_1, H') + \hom(G_2, H')\).
Every homomorphism from \mG\ to \mH\ has to either map all the vertices in
\(X\) to \(A\) and all the vertices in \(Y\) to \(B\) or vice-versa. The homomorphisms that
map \(X\) to \(A\) and \(Y\) to \(B\) are exactly the same as homomorphisms from \(G_1\) to
\(H'\) and similarly the homomorphisms that map \(Y\) to \(A\) and \(X\) to \(B\)
are exactly the same as homomorphisms from \(G_2\) to \(H'\)\@. This implies that
\chom(\mH) \maple\ \chom(\(H'\))\@.
\end{proof}

Typically bipartite graphs are not monotone; however, with the latter orientation
we can extend the definition of monotone graphs to bipartite graphs.
we may get monotone graph.

\begin{defi} [Bipartite Monotone Graphs]
Let \(H\) be a bipartite graph and let \(A\) and \(B\) be an arbitrary
bipartitions of \(H\). Let \(H'\) be the directed graph obtained from
\(H\) by orienting the edges from \(A\) to \(B\)\@. The graph \mH\ is said to be
\emph{monotone} if \(H'\) is monotone.
\end{defi}

Note that, the choice for bipartition of \(H\) is not significant for
\(H'\) being monotone or not.

\begin{theorem} [Bipartite Monotone Graphs]
For any bipartite monotone graph \mH, we have \chom(\mH) \maple \cbis\@.
\end{theorem}

\begin{proof}
For any bipartite monotone graph \mH, by Theorem~\ref{thrm:bior} there is a directed monotone 
graph \(H'\) such that \chom(\mH) \maple \chom(\(H'\))\@. By Theorem~\ref{thrm:monotone},
\chom(\(H'\)) \maple \cbis\@. Thus, \chom(\mH) \maple \cbis\@.
\end{proof}

%%%%%%%%%%%%%%%%%%%%%%%%%%%%%%%%%%%%%%%%%%%%%%%%%%%%%%%%%%%%%%%%%%%%%%%%%%%%%%%%%%%%%%%%%%%%%%%
% Characterization of Monotone Graphs
%%%%%%%%%%%%%%%%%%%%%%%%%%%%%%%%%%%%%%%%%%%%%%%%%%%%%%%%%%%%%%%%%%%%%%%%%%%%%%%%%%%%%%%%%%%%%%%
\section{Characterization of Monotone Graphs}
TODO write some introduction about this section. We will give a characterization of
monotone graphs. We restrict this section to non-directed graphs. Also for simplicity,
we restrict to connected graphs as well. By Theorems~\ref{thrm:pavol}~and~\ref{thrm:npcp},
\mH\ is monotone AP-reducible to \csat\ unless \mH\ is a reflexive or bipartite graph.	
In this work we only cover reflexive graphs.

Discussion about interval and proper interval graph.

\begin{example}
\mH\ is a monotone graph; however, \mH\ is not a proper interval graph.
\end{example}

\begin{example}
\mH\ is a interval graph; however, \mH\ is not a monotone graph.
\end{example}

\begin{defi} [Cover Graph]
Also mention \emph{Hess Diagram}.
\end{defi}

\begin{defi} [Length of a Lattice]
\end{defi}


Let \(L\)be a distributive lattice of length \(k\) and let \(h\) be  a homomorphism from a 
k-Cube \(Q\) to \(\mathcal{H}(L)\)\@.

TODO Continue ...

%%%%%%%%%%%%%%%%%%%%%%%%%%%%%%%%%%%%%%%%%%%%%%%%%%%%%%%%%%%%%%%%%%%%%%%%%%%%%%%%%%%%%%%%%%%%%%%
% RBA Relations
%%%%%%%%%%%%%%%%%%%%%%%%%%%%%%%%%%%%%%%%%%%%%%%%%%%%%%%%%%%%%%%%%%%%%%%%%%%%%%%%%%%%%%%%%%%%%%%
\section{RBA Relations}
In this section,  we introduce the family of \RBA\ relations and show that
for any \RBA\ relation \mR, we have \cbis \maple \ccsp(\mR)\@.
\RBA\ relations can be also interpreted as graphs; since most of the proofs in this part
are better expressed by graphs, we mostly focus on graph notations.

\begin{defi} [\RBA]
A relations is \RBA\ if it is binary, reflexive, and asymmetric.
\end{defi} 

A \RBA\ relation can be seen as a reflexive oriented digraph. Let \(H=(V,E)\) be 
an \RBA\ graph; for each \(v \in V\), we have \((v,v) \in E\)
and \((u,v) \in E\) and \((v,u) \in E\) implies \(u=v\)\@.

\begin{figure}[h]
\center{\input{figs/rba.pdftex}}
\caption{An RBA relation}
\end{figure}

Let \(\NHab Hab\) be the set of vertices that are both in
out-neighbours of \(a\) and in-neighbours of \(b\).
Note that if \mH\ is reflexive and \((a,b)\in E(H)\),
both \(a\) and \(b\) are in \(\NHab Hab\)\@.
Let \(\Hab\) be the subgraph of \mH\ induced by \(\NHab Hab\).
\emph{Magnitude} of \mH\ represented as \(\magf(H)\) is defined as
\(\max \limits_{(a,b) \in E(H)} \setof{|\NHab Hab|}\)\@.

\begin{figure}[h]
\center{\input{figs/shade2.pdftex}}
\caption{An RBA graph \ensuremath{H} such that \ensuremath{\magf(H)=2}}
\end{figure}

\begin{lemma} \label{lem:triangle-free}
For an \RBA\ graph \mH, if \(\magf(H)=2\) then \chom(\mH) \mapge\ \cbis\@.
\end{lemma}

\begin{proof}
Let \mH\ be a an \RBA\ graph such that \(\magf(H)=2\), 
we reduce the \cdsp\ problem to the \chom(\mH) problem\@.
For a given partial order \((P,\preceq)\), let \(G=(V,E)\) be the corresponding directed
(acyclic) graph. Take \(G'=(V',E')\) where
\(V'= V \cup \setof{u,v}\) and \(E' = E \cup \setof{(u,v)} \cup
\setof{(u,x),(x,v) \mid x\in V}\)\@.

For any homomorphism \(h: G' \to H\), if \(h(u)=h(v)=s\), then \(h(x)=s\) for all \(x\in V\);
otherwise, \((h(u),h(v))\in E(H)\) and for all \(x\in V\), we have either 
\(h(x)=h(u)\) or \(h(x)=h(v)\); thus,
\(h\) corresponds to a DownSet in \((P,\preceq)\)\@.

The number of homomorphisms with \(h(u)=h(v)\) is equal to \(|V(H)|\) and
the number of homomorphisms with \(h(u)\neq h(v)\) is equal to \(|E(H)|\) times the 
number of DownSets in \((P,\preceq)\)\@. Hence,
\[\hom(G',H) = |V(H)| + |E(H)|\cdot \cds(P,\preceq).\]
In other words:
\[\cds(P,\preceq) = \frac{\hom(G',H) - |V(H)|}{|E(H)|}.\] 
By Lemma~\ref{lem:linear} we have \chom(\mH) \mapge\ \cdsp\@.
\end{proof}

\begin{defi} [Polar Graph]
Let \mH\ be an \RBA\ graph. \mH\ is \emph{polar}
if there are vertices \(a,b\in V(H)\),
such that \((a,x), (x,b) \in E(H)\) for all vertices \(x \in V(H)\) (including \(a\) and \(b\)).
\end{defi}

Note that for a polar graph \mH, we have \(\NHab Hab = V(H)\) and \(\magf(H)=|V(H)|\)\@.

\begin{figure}[h]
\center{\input{figs/polar.pdftex}}
\caption{A polar graph}
\end{figure}

\begin{lemma} \label{lem:k-fixing}
For any \RBA\ graph \mH\ that \(k(H) > 2\), there is an \RBA\ graph
\(H'\) such that each connected component of \(H'\) is a
polar graph, \(\magf(H) = \magf(H')\), and
\chom(\mH) \mapge\ \chom(\(H'\)).
\end{lemma}

\begin{proof}
Take the graph \(H'\) as union of \(\Hab\) for all \((a,b)\in E(H)\) and
\(\magf(H_{a,b})=\magf(H)\)\@. Note that, each component of \(H'\) is \(\Hab\) for 
some \(a\) and \(b\); thus, each component of \(H'\) is a polar graph.
We show that \chom(\(H'\)) \maple \chom(H)\@.

Let \(G=(V,E)\) be the input graph for the \chom(\(H'\)) problem 
and let \(Y\) be a set of \(t\) fresh vertices
(the value of \(t\) will be determined later). Take graph \(G'=(V',E')\) where
\(V'=V \cup Y \cup \setof{u,v}\) and \(E'=E \cup \setof{(u,v)} \cup
\setof{(u,x),(x,v) \mid x\in V \cup Y}\)\@.

Consider all homomorphisms \(h: G'\to H\)\@.
For any \(a,b\in V(H)\) that \(h(u)=a\) and \(h(v)=b\), we have \((a,b)\in E(H)\); also
for any vertex \(x\) in \(V\) or \(Y\), since \((u,x)\) and \((x,v)\) are edges in \(G'\), 
\(x\) has to be mapped to \(\NHab Hab\)\@. Note that any vertex in \(Y\) can be freely mapped
to any vertex in \(\NHab Hab\); however, mapping of the vertices of \(V\) should preserve the edges 
of \mG\@. Thus, we have:
\begin{eqnarray*}
\hom(G',H) &=&  
\sum_{(a,b)\in E(H)}\left|\NHab Hab\right|^ t \cdot
\hom(G,\Hab)\\
&=& 
\bigsum{\substack{(a,b)\in E(H) \\
\magf(H_{a,b})=\magf(H)}} 
\magf^t(H) \cdot \hom(G,H_{a,b})+
\bigsum{\substack{(a,b)\in E(H) \\ \magf(H_{a,b}) < \magf(H)}}k^t
(H_{a,b})\cdot \hom(G,H_{a,b})
\end{eqnarray*}

Divide both sides of the formula by \(k^t(H)\)\@. We have:
\begin{eqnarray*}
\frac{\chom(G',H)}{\magf^t(H)} &=& 
\bigsum{\substack{(a,b)\in E(H) \\ \magf(H_{a,b})=\magf(H)}} 
\hom(G,H_{a,b}) + 
\bigsum{\substack{(a,b)\in E(H) \\ \magf(H_{a,b})<\magf(H)}}
\left(\frac{\magf(H_{a,b})}{\magf(H)} \right )^t
\cdot\hom(G,H_{a,b})\\
\frac{\chom(G',H)}{\magf^t(H)} &=& 
\hom(G,H') + \bigsum{\substack{(a,b)\in E(H) \\ \magf(H_{a,b})<\magf(H)}}
\left(\frac{\magf(H_{a,b})}{\magf(H)} \right )^t
\cdot\hom(G,H_{a,b})\\
\end{eqnarray*}
Let \(n=|V|\) and let \(m=|V(H)|\). We have \(\frac{\magf(\Hab)}{\magf(H)} \le \frac{m-1}{m}\),
\(E(H)\le n^2\), and \(\hom(G,H_{a,b}) \le n^m\).
We can simplify the formulas as \((\frac{m-1}{m})^t \approx e^{\frac{-t}{m}}\),
\(n^m = e^{m\log n}\), and \(n^2 = e^{2\log n}\).
For \(t > (m^2+2m) \log n\), the second part of the summation is less the \(1\).
Thus, by Lemma~\ref{lem:linear} \chom(\mH) \mapge \chom(\(H'\)).
\end{proof}

\begin{lemma} \label{lem:onecomp}
Let \mH\ be an \RBA\ graph such that each connected component of \mH\ is a polar graph.
There is connected component \(H'\) of \mH, such that \chom(\(H'\)) \maple \chom(\mH)\@.
\end{lemma}

\begin{proof}
There are two cases:

First case, all connected components of \mH\ are isomorphic.
Let \(H'\) be a connected component of \mH\ and
\mm\ be the number of connected components of \mH\@.
For any graph \mG\ as input for the \chom(\mH) problem, \(\hom(G,H)=\hom^m(G,H')\).
By Lemma~\ref{lem:logarithm}, \chom(H) \mapge \chom(\(H'\))\@.

Second case, there are two connected components of \mH, \(H_1\) and \(H_2\) that are not isomorphic.
By Lov\'{a}sz's Theorem, there is a connected graph \(Z\) such that \(\hom(Z,H_1) \neq \hom(Z,H_2)\)\@.
Let \(Z_1,\dotsc,Z_t\) be \(t\) fresh copies of \(Z\), where \(t\) is a large number to be 
determined later.
Take the graph \(G=(V',E')\) where \(V'=V \cup V(Z_1)\cup \dotsb \cup V(Z_t) \cup \setof{u,v} \)
and \(E'=E \cup \setof{(u,v)} \cup \setof{(u,x),(x,v) \mid x \in V \cup  Z_1 \cup \dotsb \cup Z_t} \cup
E(Z_1)\cup \dotsb \cup E(Z_t)\)\@. Take \[q= \max \limits_{(a,b) \in E(H)} \hom(Z,H_{a,b})\] also 
take \[H'=\bigcup_{\substack{(a,b)\in E(H) \\ 
\hom(G,H_{a,b})=q}} H_{a,b}\]

Consider all homomorphisms \(h: G'\to H\)\@.
For any \(a,b\in V(H)\) that \(h(u)=a\) and \(h(v)=b\), we have \((a,b)\in E(H)\)\@.
For any vertex \(x\) in \(V\) or \(Z_i\), since \((u,x)\) and \((x,v)\) are edges in \(G'\), 
\(x\) has to be mapped to \(\NHab Hab\)\@. Thus, we have:
\begin{eqnarray*}
\hom(G',H) &=&
\sum_{(a,b)\in E(H)} \hom(G,H_{a,b}) \cdot \hom^t(Z, H_{a,b}) \\
&=&
\bigsum{\substack{(a,b)\in E(H)\\ \hom(Z, H_{a,b}) = q}} \hom(G, H_{a,b}) \cdot q^t + 
\bigsum{\substack{(a,b)\in E(H)\\ \hom(Z, H_{a,b}) < q}} 
\hom(G, H_{a,b}) \cdot \hom^t(Z,H_{a,b})\\
&=&
\hom(G,H') \cdot q^t + \bigsum{\substack{(a,b)\in E(H)\\ \hom(Z, H_{a,b}) < q}} 
\hom(G, H_{a,b}) \cdot \hom^t(Z,H_{a,b})
\end{eqnarray*}
Divide both sides by \(q^t\), and let \(n=|V|\) and \(m=\magf(H)\)\@. We have:
\[
\frac{\hom(G',H)}{q^t} = 
\hom(G,H') + 
\bigsum{\substack{(a,b)\in E(H)\\ \hom(Z, H_{a,b}) < q}} 
\hom(G, H_{a,b}) \cdot \hom^t(Z,H_{a,b}) \cdot (\frac{1}{q})^t
\]
The second part of right hand side can be bounded as:
\begin{eqnarray*}
\bigsum{\substack{(a,b)\in E(H)\\ \hom(Z, H_{a,b}) < q}} 
\hom(G, H_{a,b}) \cdot \hom^t(Z,H_{a,b}) \cdot (\frac{1}{q})^t & \le &
n^2 \cdot n^m \cdot \left(\frac{q-1}{q}\right)^t\\
& \approx & e ^ {2\log n} \cdot e^{m\log n} \cdot e^{-\frac{t}{q}} \\
\end{eqnarray*}

For \(t > q \cdot (m+2)\log n\), it is less then one; hence, we have
\(hom(G,H')=\frac{\hom(G',H)}{q^t}\)\@. By Lemma~\ref{lem:linear},
we have \chom(\mH) \mapge\ \chom(\(H'\)). 

\(H'\) is a subgraph of \mH, with \(\magf(H')=\magf(H)\) and connected components of \(H'\)
are proper subsets of components of \mH\@. After considering the cases multiple times,
\(H'\) will consist of only a single component.
\end{proof}

Using Lemme \ref{lem:onecomp} and \ref{lem:k-fixing} we show that we can reduce the 
\(\magf(H)\) function without making the problem easier.

\begin{lemma} \label{lem:k-reduction}
Let \mH\ be an \RBA\ graph, if \(\magf(H) > 2\), then
there is an \RBA\ graph \(H'\) such that \(\magf(H') < \magf(H)\)
and \chom(\mH) \mapge\ \chom(\(H'\))\@.
\end{lemma}

\begin{proof}
By Lemma~\ref{lem:k-fixing}, there is an \RBA\ graph \(H_1\) such that
every component of \(H_1\) is a polar graph, \(\magf(H_1)=\magf(H)\), and 
\chom(\(H_1\)) \maple \chom(\mH)\@.

By Lemma~\ref{lem:onecomp}, there is a connected component \(H_2\) of \(H_1\) such that,
\(\magf(H_2)=\magf(H_1)\) and \chom(\(H_2\)) \maple \chom(\(H_1\))\@.

Consider the graph \(H_2\) as relation \mR\ over the domain \mD\@.
\mR\ is a polar relation, so there is an element \(a \in D\) such that for all \(x \in D\)
we have \(R(a, x)\)\@. Let \(S = D \setminus \setof a\)\@. We have
\(R(x,a)\) implies \(x=a\) and for any \(y \in S\), we have \(R(a, y), R(y, y)\)\@.
Thus, by Lemma~\ref{lem:pinning}, \ccsp(\setof{R,S}) \maple \ccsp(\mR)\@.

For \(H'=H_2 - a\), \(\magf(H') < \magf(H)\) and
\[\chom(H') \aple \ccsp(\setof{R,S}) \aple \ccsp(R) \aple \chom(H_2) \aple \chom(H).\]
\end{proof}

\begin{theorem}
For every non-empty \RBA\ graph \mH, \chom(\mH) \mapge \cbis\@.
\end{theorem}


\begin{proof}
By contradiction, consider an \RBA\ graph \mH\ with minimum \(\magf(H)\)
such that the \chom(\mH) problem is not AP-reducible to the \cbis\ problem.
By Lemma~\ref{lem:triangle-free}, \(\magf(H) > 2\)\@. 
By lemma~\ref{lem:k-reduction}, there exists an \RBA\ graph \(H'\) such that 
\chom(\(H'\)) \maple \chom(\mH) and \(\magf(H) < \magf(H')\)\@.
This contradicts with \mH\ having the minimum \(\magf(H)\).
\end{proof}
