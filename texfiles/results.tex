\chapter{Results on Complexity of \cbis}
%%%%%%%%%%%%%%%%%%%%%%%%%%%%%%%%%%%%%%%%%%%%%%%%%%%%%%%%%%%%%%%%%%%%%%%%%%%%%%%%%%%%%%%%%%%%%%%
% Monotone Graphs
%%%%%%%%%%%%%%%%%%%%%%%%%%%%%%%%%%%%%%%%%%%%%%%%%%%%%%%%%%%%%%%%%%%%%%%%%%%%%%%%%%%%%%%%%%%%%%%
\section{Monotone Relations}
\begin{defi}
A set of relations(\(\relset\))  is monotone if there is distributive
lattice \mD\ such that every relation \(R \in \relset\) is closed under
operations \(\vee\) and \(\wedge\) defined by \(D\)\@.
\end{defi}

Every distributive lattice is isomorphic to a induced subset of k-cube for 
some k and every node of the k-cube (also distributive lattice) can be 
addressed using a binary code.
Let \(\cube(v)\) be as the binary address of vertex \(v\) also
\(\recube(\bar t)\) is vertex at address \(\bar t\)\@.
\(\cube\) and \(\recube\) can be extended to vectors as
\(\cube(\bar x) = (\cube(\bar x_1),\cube(\bar x_2), \cdots, \cube(\bar x_l))\) 
and \(\bar x = \recube(\cube(\bar x))\).

\begin{theorem} \label{theorem:binary}
If a set of relations \mrelset\ is monotone then \(\ccsp(\relset) \aple \cbis\)
\end{theorem}

\begin{proof}
First consider the functions \(\cube\) and \(\recube\) used for
binary encoding of the domain \mrelset\ using \(k\) bits
Construct \(\relset'\) by replacing each relation \mR\ in \mrelset\ with a new 
relation \(R'\)\@.
Domain of \(R'\) will be \(\setof \zo\)\@. Arity of \(R'\) will be \(k\) times
arity of \(R'\)(k is the lenght of encoding) and:
\[\bar x \in R \Leftrightarrow \cube(\bar x)\in R'\]

\begin{eqnarray*}
\left. \begin{array}{l} \bar x_i \in R' \\ \bar x_j \in R'  \end{array}\right \} 
\Rightarrow  \\
\left \{ \begin{array}{l} \recube(\bar x_i) \in R \\ \recube(\bar x_j) \in R \end{array} \right \}
\Rightarrow \\
\left \{ \begin{array}{l} \recube(\bar x_i) \vee \recube(\bar x_j) \in R \\ \recube(\bar x_i) \wedge \recube(\bar x_j) \in R \end{array} \right \}
\Rightarrow  \\
\left \{ \begin{array}{l} \cube(\recube(\bar x_i) \vee \recube(\bar x_j)) \in R' \\ \cube(\recube(\bar x_i) \wedge \recube(\bar x_j)) \in R' \end{array} \right \} 
\Rightarrow  \\
\left \{ \begin{array}{l} \bar x_i \vee \bar x_j \in R' \\ \bar x_i \wedge \bar x_j \in R' \end{array} \right \} 
\end{eqnarray*}
\(\relset'\) is  monotone; thus by \cite{Tricotomy}
\(\ccsp(\relset') \aple \cbis\)\@. On the other side it is easy to see 
\(\ccsp(G,\relset) = \ccsp(\cube(G), \relset')\) hence 
\(\ccsp(\relset) \aple \cbis\)\@.
\end{proof}

%%%%%%%%%%%%%%%%%%%%%%%%%%%%%%%%%%%%%%%%%%%%%%%%%%%%%%%%%%%%%%%%%%%%%%%%%%%%%%%%%%%%%%%%%%%%%%%
% BiMonotone Graphs
%%%%%%%%%%%%%%%%%%%%%%%%%%%%%%%%%%%%%%%%%%%%%%%%%%%%%%%%%%%%%%%%%%%%%%%%%%%%%%%%%%%%%%%%%%%%%%%

\section{BiMonotone Graphs}
TODO: extend the results to bipartite graphs.
Let \mG\ be a bipartite graph such that if the graph resulted by orientation of edges
from one component to other is monotone then \ccsp(\mG) \(\aple\) \cbis\@. 

%%%%%%%%%%%%%%%%%%%%%%%%%%%%%%%%%%%%%%%%%%%%%%%%%%%%%%%%%%%%%%%%%%%%%%%%%%%%%%%%%%%%%%%%%%%%%%%
% RBA Relations
%%%%%%%%%%%%%%%%%%%%%%%%%%%%%%%%%%%%%%%%%%%%%%%%%%%%%%%%%%%%%%%%%%%%%%%%%%%%%%%%%%%%%%%%%%%%%%%

\section{RBA Relations}
\begin{defi} [\RBA]
A relations \mR\ is said to be \RBA\ if and only if \mR\ is binary, reflexive, and asymmetric.
\end{defi} 

In other word, the relation is isomorphic to a reflexive oriented digraph. Let \mH\ be a 
graph isomorph to an \RBA\ relation; for each \(v \in V(H)\), we have \((v,v) \in E(H)\)
and \((u,v)\in E(H)\) and \((v,u) \in E(H)\) implies \(u=v\)\@.
\begin{figure}[h]
\center{\input{figs/rba.pdftex}}
\caption{An RBA relation}
\end{figure}
Let \(\nplus a\) be the set of out-neighbours of \(a\) and let \(\nminus b\) be
the set of in-neighbours of \(a\). Note that, \(a\in \nplus a,\nminus a\). 
Denote neighbours of \(a\) common with \(b\) by notations \(\naplus a b\) and \(\naminus a b\)
where,
\[\naplus a b = \nplus a \cap \nplus b\]
and 
\[\naminus a b = \nplus a \cap \nminus b.\]
Denote in-degree and out-degree of \(a\) with commonly used notations \(d^+(a)=|\nplus a|\)
and \(d^-(a)=|\nminus a|\)\@. Extend the notations for degree with 
\(d^+_a(b)=|\naplus a b|\) and \(\dmab=|\naminus a b|\)\@.

Let \(H_{a,b}\) be the subgraph of \mH\ induced by \(\naplus a b]\)\@.

\begin{defi}[Magnitude]
For a \RBA\ \mR, \emph{magnitude} of \mR\ represented by 
\(\magf(R)\) is defined as 
\(\magf(R) = \max \limits_{a,b \in V(R)} \setof{\dmab}\)\@.
\end{defi}

\begin{lemma} \label{lem:triangle-free}
For all \RBA\ \mR\ with \(\magf(R)=2\), \ccsp(\mR) \(\apge\) \cbis\@.
\end{lemma}
\begin{figure}[h]
\center{\input{figs/rbak2.pdftex}}
\caption{An RBA relation \mR\ such that \(\magf(R)=2\)}
\end{figure}

\begin{defi} [\(Pol(G)\)]\label{def:pol}
Let \(\probi=(V,\mathcal{C})\) be a \ccsp\ instance with constraint language  \(\setof R\)\@.
\(Pol(\probi)\) is an instance \(\probi'(V',\mathcal{C}')\) of \ccsp(\mR) such that
\(V' = V \cup \setof{u,v}\) and \(\mathcal{C}' = \mathcal{C} \cup \setof{\const{(u,v), R}} \cup
\setof{\const{(u,x), R}, \const{(x,v),R} \mid x \in V}\)\@.
\end{defi}

\begin{proof}
Let \mR\ be a an \RBA\ relation such that \(\magf(R)=2\), 
we reduce \cds\ problem to \ccsp(\mR)\@.
Let \(\probi(V,\mathcal{C}\) be an instance of \(\cds\) problem expressed as a \ccsp instance,
and \(\probi'(V',\mathcal{C}' = Pol(\probi')\)\@.
Denote variables in \(V'- V\) by \(u\) and \(v\) such that \(\const{(u,v),R}\in \mathcal{C}'\)\@.

Let \(\varphi\) be assignment of \(\probi'\)\@. If we have \(\varphi(u)=\varphi(v)=s\) 
for some \(s\) in \(D(R)\) then \(\varphi(x)=s\) for all variables \(x\in V\).
The number of such assignments is equal to the size of domain of \mR\@.

Now let \(\varphi\) be an assignment such that \(\varphi(u) \neq \varphi(v)\).
Since \(\const{(u,v),R}\) is a constraint in \(C'\) then
\((\varphi(u),\varphi(v)) \in R\) \@.
This implies that there will be a one-to-one correspondence between possible assignments 
\(\varphi\) for \(\probi'\) with \(\varphi(u) \neq \varphi(v)\)  and downsets in \(\probi\)\@; 
\[\#\probi' = |D(R)| + |R|\cdot \#\probi \]
which implies \ccsp(\mR) \(\apge\) \cds\@.
\end{proof}

\begin{defi} [Polar Relation]
Let \mR\ be an \RBA\ relation. We say that \mR\ is a 
\emph{Polar} if there exists elements \(a\) and \(b\) 
such that \((a,x), (x,b) \in R\) for all elements \(x\) in \(D(R)\) (including \(a\) and \(b\))
In other words \(\nplus a = \nminus b = D(R)\) or \(\magf(R)=|V(R)|\)\@.
\end{defi}

\begin{figure}[h]
\center{\input{figs/polar.pdftex}}
\caption{A Polar relation}
\end{figure}

\begin{lemma} \label{lem:k-fixing}
If \mR\ is an \RBA\ relation  such that \(k(R) > 2\) there is a \RBA\ 
\(R'\) such that  each connected component of \(R'\) is a
polar relation, and \(\magf(R) = \magf(R')\), and
\ccsp(\mR) \(\apge\) \ccsp(\(R'\)).
\end{lemma}

\begin{proof}
Let \(\probi(V,\mathcal{C})\) be an instance of \ccsp(\mR) and let \(Y\) be a set of variables 
with size \(t\) (the value of \(t\) will be determined later), and 
\(\probi'(V',\mathcal{C}') = Pol((V+Y,\mathcal{C}))\).
Denote variables in \(V' - V - Y\) by \(u\) and \(v\) such that \(\const{(u,v),R}\in \mathcal{C}'\)\@.

Denote the instance of \ccsp(\(R'\) all constraint of the form \(\const{S,R}\)
replaced with \(\const{S,R'}\) by \(\probi(R')\)\@.

Considering possible assignment \(\varphi\) for \(\probi'\) with
\(a = \varphi(u)\) and \(b = \varphi(v)\) we have:
\begin{eqnarray*}
\#\probi' &=& 
\sum_{a,b\in D(R))}\left(\dmab\right)^ t \cdot
\#\probi(R_{a,b})\\
 &=& 
\bigsum{\substack{a,b\in D(R) \\
\magf(R_{a,b})=\magf(R)}} 
\magf^t(R) \cdot \#\probi(R_{a,b})+
\bigsum{\substack{a,b\in D(R) \\ \magf(R_{a,b})  < \magf(R)}}k^t
(R_{a,b})\cdot\#\probi(R_{a,b})
\end{eqnarray*}

Choose \(t\) large enough, i.e., \(\Omega(n^2)\) where \(n=|V|\) and divide both
sides of the previous formula by \(k^t(R)\)\@. We have
\[\frac{\#\probi}{\magf^t(R)} =  
\bigsum{\substack{a,b\in D(R) \\ \magf(R_{a,b})=\magf(R)}} 
\#\probi(R_{a,b}) + 
\bigsum{\substack{a,b\in D(R) \\\magf(R_{a,b})<\magf(R)}}
\left(\frac{\magf(R_{a,b})}{\magf(R)} \right )^t
\cdot\#\probi(R_{a,b})
\]

The second part of the summation has a value less than \(1\). 
\todo{Need more proof for \(<1\)}
We will be counting the solutions for 
\[R'=\bigcup_{\substack{a,b\in D(R) \\
\magf(R_{a,b})=\magf(R)}} R_{a,b}\]
Let \(R'\) be an \RBA\ relation such that each component of \(R'\) 
is \(R_{a,b}\) for some \(a\) and \(b\) hence satisfies the conditions.
\end{proof}


\begin{lemma} \label{lem:connected}
If \mR\ is a non-connected \RBA\ relation and every component of 
\mR\ is a polar relation then there is a component \(R'\) of \mR\ 
such that \ccsp(\mR) \(\apge\) \ccsp(\(R'\)).
\end{lemma}



\begin{proof}
We consider two cases: First presume all components of \mR\ are isomorphic.
\todo{Maybe easier argument for this case}
Let \(\probi(V,\mathcal{C})\) be an instance of \ccsp(\mR)\@.
Let \(Y\) be a set of variables of size \(t\)(the value of \(t\) will be determined later) 
and \(\probi'(V', mathcal{C}') = Pol((V \cup Y , \mathcal{C}))\).
Denote variables in \(V' - V - Y\) by \(u\) and \(v\) such that \(\const{(u,v),R}\in \mathcal{C}'\)\@.

Denote the instance of \ccsp(\(R'\) all constraint of the form \(\const{S,R}\)
replaced with \(\const{S,R'}\) by \(\probi(R')\)\@.

Considering possible assignment \(\varphi\) for \(\probi'\) with
\(a = \varphi(u)\) and \(b = \varphi(v)\) we have:

\begin{eqnarray*}
\#\probi' &=& \sum_{a,b\in D(R)} \#\probi(R_{a,b}) \cdot (\dmab)^t \\
&=&C \cdot\#\probi(R') \cdot k^t(R) + 
\sum_{\substack{a,b\in D(R)\\ \dmab < \magf(R)}} 
\#\probi(R_{a,b}) \cdot (\dmab)^t
\end{eqnarray*}


where \(C = |\setof{(a,b): \dmab=\magf(R)}|\) and \(R'\) is a
connected component of \mR\@. Note that the choice of \(R'\) is not significant 
because in this case all components of \mR\ are isomorphic.

By choosing \(t\) sufficiently large, i.e., \(\Omega(n^2)\) where \(n=|V|\) implies
that \ccsp(\mR) \(\apge\) \ccsp(\(R'\))\@. 
Since \(R'\) is a connected component of \mR, this case is proved.

For the second case, presume there are two components \(C_1\) and \(C_2\) of
\mR\ such that they are not isomorphic. By Lovasz Theorem there
is a connected instance \(Z\) such that \(\#Z(C_1) \neq \#Z(C_2)\)\@. 
Let \(\probi'(V',\mathcal{C}' = Pol((V+m \times Z, \mathcal{C})\). Here we mean 
\(m\) copies of \(Z\) for an integer \(m\) whose value will be
decided later. Let \(q\) be \(\max_{a,b \in D(R)} \#Z(R_{a,b})\).

Considering possible assignment \(\varphi\) for \(\probi'\) with
\(a = \varphi(u)\) and \(b = \varphi(v)\) we have:


\begin{eqnarray*}
\#\probi'(R')&=&\sum_{a,b\in D(R)} \#\probi(R_{a,b}) 
\cdot \#Z^m(R_{a,b}) \\
&=&\probi(R') \cdot q^m + 
\sum_{\substack{a,b\in D(R)\\ \#Z(R_{a,b}) < q}} 
\#\probi(R_{a,b}) \cdot \#Z^m(R_{a,b})
\end{eqnarray*}

where \[
R'=\bigcup_{\substack{a,b\in D(R) \\ 
\#Z(R_{a,b})=q}} R_{a,b}\]
Choosing \(m\) large enough, i.e., \(\Omega(n^2)\) where \(n=|V|\) 
and dividing TODO by \(q^m\) implies \ccsp(\mR) \(\apge\) \ccsp(\(R'\)). 
Here \(R'\) is derived from \mR\ by removing some of the components. 

Consider the latter case using \(R'\) instead of \mR\
until all remaining components of \mR\ are isomorphic.
\end{proof}


By combining lemme \ref{lem:connected} and \ref{lem:k-fixing} 
we derive the following lemma:

\begin{lemma} \label{lem:k-reduction}
For any polar relation \mR\ such that \(\magf(R) > 2\),
there is an \RBA\ relation \(R'\) such that \(\magf(R') < \magf(R)\)
and \ccsp(\mR) \(\apge\) \ccsp(\(R'\))\@.
\end{lemma}


\begin{proof}
Let \(a\) be an element in \mR\ such that \(\nplus a = D(R))\)\@.
We know \(d^-(a)=1\) 
For all other elements in \(D(R)\) \(w\), 
we have \(d^-(w) > 1\) (\(a,w\in \nminus w\)). So, \mR\
implements unary relation.
\(S = D(R)\setminus \setof a\) also
implements the binary relation 
\(R'=R \land S(x) \land S(y) = 
R[D(R) \setminus \setof a]\).
\(R'\) is an \RBA\ relation that \(\magf(R') < \magf(R)\) 
and \ccsp(\mR) \(\apge\) \ccsp(\(R'\))\@.
\end{proof}

\begin{theorem}
For every non-empty \RBA\ relation \mR, \ccsp(\mR) is \cbis-hard.
\end{theorem}


\begin{proof}
By contradiction, consider a \RBA\ relation \mR\ with the least magnitude
such that \mR\ is not \cbis-hard. By Lemma \ref{lem:triangle-free},
\(\magf(R) > 2\)\@. 
By lemma \ref{lem:k-fixing}, there exists a \RBA\ relation \(R'\) such that 
\ccsp(\(R'\)) \(\aple\) \ccsp(\mR) and  
and \(\magf(R)=\magf(R')\) and each component of \(R'\) is a polar relation. 
By lemma \ref{lem:connected} there is a connected component \(R''\)
of \(R'\) such that \ccsp(\(R''\)) \(aple\) \ccsp(\(R'\)) \(\aple\) \ccsp(\mR).
Finally  by lemma \ref{k-reduction} 
there exists a \RBA\ \(R'''\) such that 
\ccsp(\(R'''\)) \(aple\) \ccsp(\(R''\)) \(aple\) \ccsp(\(R'\)) \(\aple\) \ccsp(\mR).
and \(\magf(R''') < \magf(R)\); this contradicts with \mR\ having the minimum magnitude.
\end{proof}
