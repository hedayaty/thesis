\chapter{Approximated Reductions}
In this chapter we will provide a number of lemmas used for approximated counting. \todo{Write More}
%%%%%%%%%%%%%%%%%%%%%%%%%%%%%%%%%%%%%%%%%%%%%%%%%%%%%%%%%%%%%%%%%%%%%%%%%%%%%%%%%%%%%%%%%%%%%%%
% Connected Components
%%%%%%%%%%%%%%%%%%%%%%%%%%%%%%%%%%%%%%%%%%%%%%%%%%%%%%%%%%%%%%%%%%%%%%%%%%%%%%%%%%%%%%%%%%%%%%%
First, we show that by restricting the input to connected structures the complexity of 
\ccsp(\mrelset) does not change. We extend the definition of connectedness for structures more general 
than graphs as follows.

Let \(\probi = (V, C)\) be an instance of \ccsp(\mrelset)\@. Let H(\(\probi\)) be a graph
with the vertex set \mV\ and it contains an edge \(\setof{u,v}\) if and only if 
\(u\) and \(v\) appear in the same scope for a constraint in \(\probi\)\@.
We say an instance \(\probi\) is connected if and only if \(H(\probi)\) is a connected graph.
Let \cccsp(\mrelset) denote the problem \ccsp(\mrelset) to connected instances.

\begin{lemma} \label{lemma:connected}
For any constraint language \mrelset\ the problem \ccsp(\mrelset) is AP-interreducible with
\cccsp(\mrelset)\@.
\end{lemma}

\begin{proof}
The reduction of \cccsp(\mrelset) to \ccsp(\mrelset) is trivial. Now, let \(\probi\) be an instance of
\ccsp(\mrelset), and let \(\probi_1,\dotsc,\probi_r\) be its connected components. Take \(\eps > 0\) 
and set \(\delta = \frac{\eps}{2r}\)\@. Our reduction, given an instance \((\probi,\eps)\) calls
the algorithm for \cccsp(\mrelset) on instances \((\probi_1,\delta), \dotsc,(\probi_r,\delta)\) 
and outputs \(N = N_1 \dotsm N_r\),
where \(N_i\) is the answer given by the oracle on \((\probi_i,\delta)\)\@.

We claim that the above reduction is an AP-reduction. First of all, observe that is is polynomial
time, and the instances it produces satisfy the conditions of AP-reductions. It remains to show
that if the oracle approximates the solutions with relative error \(\delta\) then the reduction 
provides approximation within \(\eps\)\@.

Since we can assume \(\eps\) is small, we have \((1 - \delta)^r \ge 1 - 2r\delta = 1 - \eps \) and
\( ( 1 + \delta)^r \le 1 + 2r\delta = 1 + \eps\)\@. Now if 
the actual solution to \(\probi, \probi_1,\dotsc,\probi_r\) are \(N',N_1,\dotsc,N'_r\), then 
we obviously have \(N'=N'_1\dotsm N'_r\)\@. Also
\[1\delta < \frac{N_1}{N'_1} < 1 + \delta,\]
Therefore,
\[\frac{N}{N'} < \frac{N_1 \dotsm N_r}{N'_1\dotsm N'_r} < (1 + \delta)^r \le 1 + \eps.\]
The lower bound is proved in the same manner.
\end{proof}

Hence, from now on in all the proves we can presume that the input structures are connected.
%%%%%%%%%%%%%%%%%%%%%%%%%%%%%%%%%%%%%%%%%%%%%%%%%%%%%%%%%%%%%%%%%%%%%%%%%%%%%%%%%%%%%%%%%%%%%%%
% Linear relationship
%%%%%%%%%%%%%%%%%%%%%%%%%%%%%%%%%%%%%%%%%%%%%%%%%%%%%%%%%%%%%%%%%%%%%%%%%%%%%%%%%%%%%%%%%%%%%%%
\todo{Write something involving adding constant and multiplication by a constant factor}
Next, we show that if the solution for two problems are related by a linear transformation then they
are AP-interreducible. \todo{Write something to relate to the lemma}

\begin{lemma}\label{lemma:linear}
Let \(\varphi\) be a polynomial time computable function that maps every instance of a
counting problem \(A\) to an instance of a counting problem problem \(B\) in such a way that
there are constants \(d > 0\) and \(c\) (not necessarily positive)such that
for any \(A\)-instance \(\probi\) we have
\(\#\probi = d\cdot \#\varphi(\probi) + c)\)\@. Then in any instance of \(A\)
has at least one solution 
(in case \(c < 0\)), or every instance of \(B\) has at least one solution (in case \(c>0\)), then
there is an AP-reduction from \(A\) to \(B\)\@.
\end{lemma}
\begin{proof}
The AP-reduction works as follows: On an instance \(\probi\) of \(A\) and \(\eps > 0\) it makes 
an oracle call \(\varphi(\probi),\delta)\) to \(B\), where \(\delta = \frac{\eps}{p}\) and 
\(p = 1 + |c|/d\) if \(|c|/d\) is integer, and \(p = 1 + \frac{|C|}{d\ceil{|c|/d}-c/d}\) otherwise,
and if the oracle's reply is \(N\), it returns \(dN + c\)\@. Clearly the algorithm make polynomially 
many stores and oracle call, and the oracle request is of the correct form.
Thus, it suffices to show that if the oracles's solution is within relative error of \(\delta\) then
the algorithm gives an \(1+\eps\)-approximation of \(\#\probi\)\@.

Let the exact and approximation solutions for \(\varphi(\probi)\) be \(N'\) and \(N\), respectively;
then the exact and approximation solutions for \(\probi\) are \(\frac{dN'}{dN'+c} \le p\)\@.
Indeed,if \(c > 0\) this fraction does not exceed 1\@. If \(c < 0\) then 
it follows from the assumption \(dN' + c > 0\) which implies \(N' > |c|/d\)\@.
Thus the relative error can be bounded by 
\[\frac{dN + c}{dN' + c} \le \frac{(1+\delta)dN' + c}{dN'+c} = 1 + \frac{\delta d N'}{d N' + c} \le 1 + \frac{\eps}{p}p = 1 + \eps.\]

The inequality \(1-\eps \le \frac{dN + c}{dN' + c}\) is similar.
\end{proof}

%%%%%%%%%%%%%%%%%%%%%%%%%%%%%%%%%%%%%%%%%%%%%%%%%%%%%%%%%%%%%%%%%%%%%%%%%%%%%%%%%%%%%%%%%%%%%%%
% Projection
%%%%%%%%%%%%%%%%%%%%%%%%%%%%%%%%%%%%%%%%%%%%%%%%%%%%%%%%%%%%%%%%%%%%%%%%%%%%%%%%%%%%%%%%%%%%%%%
Let \mR\ be a \(k\)-ary relation and \(S=\setof{i_1,\dotsc,i_l} \subseteq \setof{1,\dotsc,k}\)\@.
By \(\proj S R\) we denote the \emph{projection} of \mR\ onto the set \(S\) of its coordinate
positions, that is, the relation \(\setof{(a_{i_1},\dotsc,a_{i_l}) \mid (a_1,\dotsc,a_k)\in R}\)\@.
Observe that \(\proj S R\) can be pp-defined in \mR\ by quantifying away all coordinate positions of
\mR\ except for those in \(S\)\@.
Although existential quantification is not known to give rise to AP-reducible problems, in some
cases it does.

\begin{lemma}\label{lem:projection}
If \mrelset\ is a constraint language and for some \(k\)-ary relation \(R\in\relset\)
there is a set \(S\subset \setof{k-1}\) such that \(|\proj S R|=|R|\) then 
\(\ccsp(\relset \cup \setof{\proj S R}) \aple \ccsp(\relset)\).
\end{lemma}

\begin{proof}
The AP-reduction is constructed as follows: Given an instance \(\probi\)
of \(\ccsp(\relset\cup \{\proj S R\})\) with variable set \(V\) we define an 
instance \(\probi'\) of \(\ccsp(\relset)\) such that it has the same number of solutions.
Let \(m=k=|S|\)\@. The instance \(\probi'\) includes all the variables of \(\probi\)
and all its constraints except for those having \(\proj S R\) as the constraint relation. For each
constraint in \(\probi\) of the form \(\const{(v_1,\dotsc,v_l),\proj S R})\)
we introduce a constraint \(\const{(w_1,\dotsc,w_k),R}\), where 
\(w_i=v_{i_j}\) if \(i\in S\) and \(i=i_j\), or a new variable that does not appear in any other constraint scope. 

Clearly, the restriction of any solution of \(\probi'\) onto \(V\) is a solution of \(\probi\)\@.
Furthermore, the condition \(|\proj S R|=|R|\) implies that any solution of \(\probi\)
can be extended to a solution of \(\probi'\) in a unique way. The lemma is proved.
\end{proof}

%%%%%%%%%%%%%%%%%%%%%%%%%%%%%%%%%%%%%%%%%%%%%%%%%%%%%%%%%%%%%%%%%%%%%%%%%%%%%%%%%%%%%%%%%%%%%%%
% Pinning
%%%%%%%%%%%%%%%%%%%%%%%%%%%%%%%%%%%%%%%%%%%%%%%%%%%%%%%%%%%%%%%%%%%%%%%%%%%%%%%%%%%%%%%%%%%%%%%

We will need several auxiliary statements. The first of them generalizes the Pinning Lemma from
\cite{Trichotomy}. This lemma allows one to add an extra unary relation to the constraint language.
The proof of Lemma~\ref{lem:pinning} also follows closely the proof in \cite{Trichotomy}. 
This is why we omit a portion of computation in our proof. It is convenient to consider a subset of the domain as a unary relation.

\begin{lemma}[Extended Pinning]\label{lem:pinning}
Let \(\relset\) be a constraint language over the set \(D=\setof{k-1}\)\@,
and let for a certain subset \(S \subset D\) 
there be an \(l\)-ary relation \(R \in \relset\) and a coordinate position \(j\),
\(1 \le j \le l\), such that for any \(a\in S\) the relation \mR\ has more tuples
\(\vara\) with \(\vara[j]=a\) than tuples \(\varb\) with
\(\varb[j] \notin S\)\@. Then \(\ccsp(\relset \cup \setof{S}) \aple \ccsp(\relset)\)\@.
\end{lemma}

\begin{proof}
Fix an \(l\)-ary relation \(R \in \relset\)\@, and 
a coordinate position \(j\) such that \mR\ and \(j\) satisfy the conditions of the lemma.
Let also \(w\) be the minimal (over elements \(a\in S\)) number of tuples \(\vara\)
such that \(\vara[j]=a\), and let \(w'\) be the number of tuples \(\varb\) with
\(\vara[j] \notin S\). By the conditions of the lemma \(w'<w\)\@.

Consider an instance \(\probi\) of \(\ccsp(\relset \cup \setof{S})\)
with \mn\ variables. Let \(N_S\) be the set of
variables which occur in the scope of constraints of \(\probi\) with relation \(\probi\)\@. 
Set \(n_S = |N_S|\) and \(m = \ceil{\frac{n+2}{\lg{\frac{w}{w'}}}}\)\@.
Construct an instance \(\probi'\) of \(\ccsp(\relset)\) as follows:
\begin{itemize}
\item
The set of variables of \(\probi'\) includes all variables from \(\probi\),
and also, for each variable \(x \in N_S\), any \(u \in \setof{\oneto m}\), and any
\(v\in\setof{k}-\{j\}\) a fresh variable \(x_{u,v}\)\@. 
\item 
Include all constraints from \(\probi\) other than those involving \(S\)\@. 
\item
For each constraint \(C=\const{(x),S}\) from \(\probi\) include \mm\ constraints whose
relation is \mR, variable \(x\) occupies the \(j\)th position in the scope,
and the variable \(x_{u,v}\) is in the \(v\)th position of the \(u\)th constraint.
\end{itemize}

Now any solution of \(\probi\) can be extended in at least \(w^{mn_S}\) 
ways to a solutions of \(\probi'\)\@, provided all variables from \(N_S\)
take values from \(S\)\@. On the other hand, every assignment that does not satisfy this
condition can be extended in at most \(w^{m(n_S-1)}w^{\prime m}\) ways.
There exist no more than \(k^n\) such solutions. Therefore if \(N\) and \(N'\)
denote the number of solutions to \(\probi\) and \(\probi'\), respectively, then
\[Nw^{mn_S} \le N' \le Nw^{mn_S} + k^nw^{m(n_s-1)}w'^m.\]
So, for a properly chosen \mm, 
\[N \le \frac{N'}{w^{mn_S}} \le N + \frac{1}{4},\]
which implies
\[N = \left\lfloor \frac{N'}{w^{mn_S}} \right\rfloor.\]
Now we can complete the proof in exactly the same way as in \cite{Trichotomy}.
\todo{Write up this part}
\end{proof}
%%%%%%%%%%%%%%%%%%%%%%%%%%%%%%%%%%%%%%%%%%%%%%%%%%%%%%%%%%%%%%%%%%%%%%%%%%%%%%%%%%%%%%%%%%%%%%%
% Maximization
%%%%%%%%%%%%%%%%%%%%%%%%%%%%%%%%%%%%%%%%%%%%%%%%%%%%%%%%%%%%%%%%%%%%%%%%%%%%%%%%%%%%%%%%%%%%%%%

The next technique is a strong tool which generalizes above theorems. \todo{Mention More}

\begin{defi}[Max-implementation]\label{def:max}
Let \mrelset\ be a set of relations on a set \mD, and let \mR\ be an \(n\)-ary relation on \mD\@. 
Let \(\probi\) be an instance of \ccsp(\mrelset) over the set of variables consisting of 
\(V = V_x |cup V_y\), where \(V_x = \setof{x_1,x_2,\dotsc,x_n}\) and 
\(V_y=\setof{y_1,y_2,\dotsc,y_q}\)\@. For any assignment of \(\varphi: V_x \to D\),
let \(\#\varphi\) be the number of assignments \(\psi : V_y \to D\)\@ such that 
\(\varphi \cup \psi\) satisfy \(\probi\)\@. Let \(M\) be the maximum value of 
\(\#\varphi\) among all assignments of \(V_x\)\@. The instance \(\probi\) is said to be
a \emph{max-implementation} of \mR\ if a tuple \(\varphi\) is an \mR\ if and only if
\(\#\varphi = M\)\@.
\end{defi}

\begin{theorem}\label{theo:max}
If there is max-implementation of \mR\ by \mrelset, then \ccsp(\(\relset \cup \setof R\))~\(\aple\)~
\ccsp(\mrelset)\@.

\begin{proof}
For any instance \(\probi_1 = (V_1,\mathcal{C}_1)\) of  \ccsp(\(\relset \cup \setof R\)) we
construct instance \(\probi_2 = (V_2,\mathcal{C}_2)\) of  \ccsp(\mrelset) as follows.
\begin{itemize}
\item Choose sufficiently large integer \mm(to be determined later)
\item Let \(C_1,\dotsc,C_l\in\mathcal{C}_1\) be constraints from \(\probi_1\)involving
\mR, \(C_i=\const{s_i,R }\)\@. Set \(V_2 =V_1\cup(V_1^i\cup\dotsb \cup V_m^i)\), where \(V_j^i\)
is a fresh copy of \(V_y\) from Definition~\ref{def:max}\@.
\item Let \(\mathcal{C}\) be the set of constraints from \(\
\end{itemize}
\end{proof}


\end{theorem}