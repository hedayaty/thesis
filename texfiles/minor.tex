\chapter{Techniques for AP-Reductions}
In this chapter we provide a number of lemmas which can be used for obtaining AP-reductions.
These results have been communicated in \cite{maxi}, \cite{mvl}\@.
First, we show that we can limit the input problem to connected structures. 
Next, we show that AP-reductions are possible when solutions for the two problems are related by a 
linear transformation. Next, we introduce a technique called pinning , which can be also used
to derive AP-reductions. Finally, we introduce max-definability, which 
is our best tool for deriving AP-reductions. Some of these techniques will be used in subsequent chapters. 

%%%%%%%%%%%%%%%%%%%%%%%%%%%%%%%%%%%%%%%%%%%%%%%%%%%%%%%%%%%%%%%%%%%%%%%%%%%%%%%%%%%%%%%%%%%%%%%
% Connected Components
%%%%%%%%%%%%%%%%%%%%%%%%%%%%%%%%%%%%%%%%%%%%%%%%%%%%%%%%%%%%%%%%%%%%%%%%%%%%%%%%%%%%%%%%%%%%%%%
\section{Connected graphs}
First, we show that by restricting the input to connected structures the complexity of 
\ccsp(\mrelset) does not change. We extend the definition of connectedness from graphs
to general structures.

Let \(\probi = (D, V, \conset)\) be an instance of \ccsp(\mrelset)\@.
Let H(\(\probi\)) be a graph
with the vertex set \mV, that contains an edge \(uv\) if and only if 
\(u\) and \(v\) appear in the same scope for a constraint in \(\conset\)\@.
We say an instance \(\probi\) is \emph{connected} if the graph \(H(\probi)\) is connected.
Let \cccsp(\mrelset) denote the problem \ccsp(\mrelset) limited to connected instances.

\begin{lemma}\label{lem:connected}
For any constraint language \mrelset\ the problem \ccsp(\mrelset) is AP-interreducible with
\cccsp(\mrelset)\@.
\end{lemma}

\begin{proof}
The reduction of \cccsp(\mrelset) to \ccsp(\mrelset) is trivial. Now, let \(\probi\) be an instance of
\ccsp(\mrelset), and let \(\probi_1,\dotsc,\probi_r\) be its connected components. Take \(\eps > 0\) 
and set \(\delta = \frac{\eps}{2r}\)\@. Given an instance \((\probi,\eps)\), our reduction calls
the algorithm for \cccsp(\mrelset) on instances \((\probi_1,\delta), \dotsc,(\probi_r,\delta)\) 
and get outputs \(N = N_1 \dotsm N_r\),
where \(N_i\) is the answer given by the oracle on \((\probi_i,\delta)\)\@.

We claim that the above reduction is an AP-reduction. First of all, observe that it is polynomial
time, and the instances it produces satisfy the conditions of AP-reductions. It remains to show
that if the oracle approximates the solutions with relative error \(\delta\) then the reduction 
provides approximation within \(\eps\)\@.

Since we can assume \(\eps\) is small, we have \((1 - \delta)^r \ge 1 - 2r\delta = 1 - \eps \) and
\( ( 1 + \delta)^r \le 1 + 2r\delta = 1 + \eps\)\@. If 
the actual solutions to \(\probi, \probi_1,\dotsc,\probi_r\) are \(N',N'_1,\dotsc,N'_r\), then 
we obviously have \(N'=N'_1\dotsm N'_r\)\@. The rest of the proof goes as follows.
\begin{eqnarray*}
&\left|\frac{N'_i - N_i}{N_i}\right| &\le \delta \\
&\left|\frac{N'_i}{N_i} -1 \right| &\le \delta \\
1 - \delta  \le &\frac{N'_i}{N_i} &\le 1 + \delta \\
(1-\delta)^r \le &\frac{N'_1\cdot N'_2 \dotsm N'_r}{N_1\cdot N_2 \dotsm N_r} &\le (1 + \delta)^r \\
1-\eps \le &\frac{N'}{N} &\le 1+\eps \\
&\left|\frac{N'}{N}-1\right|&\le \eps\\
&\left|\frac{N'-N}{N}\right|&\le \eps.
\end{eqnarray*}
\end{proof}
Hence, in the later proofs we can restrict the input structures to connected structures.
%%%%%%%%%%%%%%%%%%%%%%%%%%%%%%%%%%%%%%%%%%%%%%%%%%%%%%%%%%%%%%%%%%%%%%%%%%%%%%%%%%%%%%%%%%%%%%%
% Linear relationship
%%%%%%%%%%%%%%%%%%%%%%%%%%%%%%%%%%%%%%%%%%%%%%%%%%%%%%%%%%%%%%%%%%%%%%%%%%%%%%%%%%%%%%%%%%%%%%%
\section{Linear Transformations}
Next, we show that if the solutions of \ccsp(\mrelset) and \ccsp(\(\relset'\)) are linearly related on all inputs,
then they are AP-interreducible. 

\begin{lemma} \label{lem:linear}
Let \(\varphi\) be a polynomial time computable function that maps every instance of a
counting problem \(A\) to an instance of a counting problem problem \(B\) in such a way that
there are constants \(d > 0\) and \(c\) (not necessarily positive) such that
for any \(A\)-instance \(\probi\), we have
\(\#\probi = d\cdot \#\varphi(\probi) + c\)\@. If every instance of \(A\)
has at least one solution 
(in case \(c < 0\)), or every instance of \(B\) has at least one solution (in case \(c>0\)), then
there is an AP-reduction from \(A\) to \(B\)\@.
\end{lemma}

\begin{proof}
The AP-reduction works as follows. On an instance \(\probi\) of \(A\) and \(\eps > 0\) it makes 
an oracle call \((\varphi(\probi),\delta)\) to \(B\), where \(\delta = \frac{\eps}{p}\) for 
\(c > 0\), \(p = 1\) otherwise, \(p = \frac1-c/d\) and
if the oracle's reply is \(N\), it returns \(dN + c\)\@. Clearly the algorithm make polynomially 
many steps and oracle calls, and the oracle request is of the correct form.
Thus, it suffices to show that if the oracle's solution is within relative error of \(\delta\) then
the algorithm gives an \(1+\eps\)-approximation of \(\#\probi\)\@.

Let the exact and approximation solutions for \(\varphi(\probi)\) be \(N'\) and \(N\), respectively;
then the exact and approximation solutions for \(\probi\) are  
\(dN' + c\) and \(dN + c\), respectively. Since \(\probi > 0\), we can 
assume that \(dN+c,dN'+c,N,N'>0\)\@. With the choice of \(p\), it is trivial that
 \(\frac{dN}{dN+c} \le p\)\@. Thus we have \(\frac{dN}{dN+c}\delta \le \eps\).
All we need to show is that if \(\left|\frac{N-N'}{N}\right|<\delta\) holds then
\(\left|\frac{dN-dN'}{dN+c}\right|<\eps\) holds as well. This is seen as follows:
\begin{eqnarray*}
\frac{dN}{dN+c}\delta & \le & \eps \\
1 + \frac{dN}{dN+c}\delta & \le & 1 + \eps \\
\frac{d N (1+\delta)+c}{d N + c} & \le & \eps.\\
\end{eqnarray*}
The inequality \(\eps \le \frac{d N (1-\delta)+c}{d N + c}\) is similar.
Combining these with the assumption that \(1-\delta \le \frac{N'}{N} \le 1 + \delta \),
we get \(1-\eps \le \frac{dN'+c}{dN+c} \le 1 + \eps\). We continue with the proof as follows:
\begin{eqnarray*}
1-\eps \le \frac{dN'+c}{dN+c} & \le & 1 + \eps \\
\left|1 + \frac{dN' + c}{dN + c}\right| &\le & \eps \\
\left| \frac{dN-dN'}{dN+c}\right| & \le& \eps.
\end{eqnarray*}
This completes the proof.
\end{proof}

\begin{lemma} \label{lem:logarithm}
Let \(\varphi\) be a polynomial time computable function that maps every instance of a
counting problem \(A\) to an instance of a counting problem problem \(B\) in such a way that
there is a constant \(m>0\) such that
for any \(A\)-instance \(\probi\) we have
\(\#\probi = (\#\varphi(\probi))^m\)\@.
Then there is an AP-reduction from \(A\) to \(B\)\@.
\end{lemma}

\begin{proof}
The AP-reduction works as follows: on an instance \(\probi\) of \(A\) and \(\eps > 0\) it makes 
an oracle call \((\varphi(\probi),\delta)\) to \(B\), where \(\delta = \frac{\eps}{m}\), and 
and if the oracle's reply is \(N\), it returns \(N^m\)\@. Clearly the algorithm make polynomially 
many steps and oracle calls, and the oracle request is of the correct form.
Thus, it suffices to show that if the oracle's solution is within relative error of \(\delta\) then
the algorithm gives an \(1+\eps\)-approximation of \(\#\probi\)\@.

Let the exact and approximation solutions for \(\varphi(\probi)\) be \(N'\) and \(N\), respectively;
then the exact and approximation solutions for \(\probi\) are  
\(N'^m\) and \(N^m\), respectively.
All we need to show is that if \(\left|\frac{N-N'}{N}\right|<\delta\) holds then
\(\left|\frac{N^m-N'^m}{N^m}\right|<\eps\) holds as well. Without loss of generality we assume
\(N>N'\), and proceed as follows:
\begin{eqnarray*}
\frac{|N-N'|}{N} & \le & \delta \\
\frac{N'}{N} & \le & 1 + \delta \\
\frac{N'^m}{N^m} & \le & (1 + \delta) ^ m\\
\frac{N'^m}{N^m} & \le & 1 + m\cdot\delta \\
\frac{N'^m}{N^m} & \le & 1 + \eps \\
\frac{N^m - N'^m}{N^m} & \le & \eps.
\end{eqnarray*}
The inequality \(\eps \le \frac{N'^m - N^m}{N^m}\) is similar.
This completes the proof.
\end{proof}

%%%%%%%%%%%%%%%%%%%%%%%%%%%%%%%%%%%%%%%%%%%%%%%%%%%%%%%%%%%%%%%%%%%%%%%%%%%%%%%%%%%%%%%%%%%%%%%
% Projection
%%%%%%%%%%%%%%%%%%%%%%%%%%%%%%%%%%%%%%%%%%%%%%%%%%%%%%%%%%%%%%%%%%%%%%%%%%%%%%%%%%%%%%%%%%%%%%%
\section{Projection}
Let \mR\ be a \(k\)-ary relation and \(S=\setof{i_1,\dotsc,i_l} \subseteq \setof{1,\dotsc,k}\)\@.
By \(\proj S R\) we denote the \emph{projection} of \mR\ onto the set \(S\) of its coordinate
positions, that is, the relation \(\{(a_{i_1},\dotsc,a_{i_l}) \mid (a_1,\dotsc,a_k)\in R\}\)\@.
Observe that \(\proj S R\) is pp-definable in \mR\ by quantifying away all coordinate
positions of \mR\ except for those in \(S\)\@.
Although existential quantification is not known to give rise to AP-reducible problems, in some
cases it does.

\begin{lemma} \label{lem:projection}
Let \mrelset\ be a constraint language over the domain \mD\ and
 let \mR\ be some \(k\)-ary relation in \mrelset\@.
If there is a set \(S\subset D\) such that \(|\proj S R|=|R|\) then 
\ccsp(\(\relset \cup \setof{\proj S R}\)) \maple\ \ccsp(\mrelset)\@.
\end{lemma}

\begin{proof}
The AP-reduction is constructed as follows: Given an instance \(\probi=(D,V,\conset)\)
of \ccsp(\(\relset\cup \{\proj S R\}\)), we define an 
instance \(\probi'=(D,V',\conset')\) of \(\ccsp(\relset)\) with the same number of solutions.
Let \(l=|S|\)\@. \(V'\) includes all the variables of \(V\) and
\(\conset'\) includes all constraints from \mconset\ except for those with \(\proj S R\). For each
constraint \(C=\const{\proj S R,(v_1,\dotsc,v_l)}\) in \mconset\ 
we include a new constraint \(C'=\const{R,(w_1,\dotsc,w_k)}\) in \(\conset'\), where 
\(w_i=v_{i_j}\) if \(i\in S\) and \(i=i_j\), otherwise a fresh variable.

Clearly, the restriction of any solution of \(\probi'\) onto \(V\) is a solution of \(\probi\)\@.
Furthermore, the condition \(|\proj S R|=|R|\) implies that any solution of \(\probi\)
can be extended to a solution of \(\probi'\) in a unique way.
\end{proof}


%%%%%%%%%%%%%%%%%%%%%%%%%%%%%%%%%%%%%%%%%%%%%%%%%%%%%%%%%%%%%%%%%%%%%%%%%%%%%%%%%%%%%%%%%%%%%%%
% Pinning
%%%%%%%%%%%%%%%%%%%%%%%%%%%%%%%%%%%%%%%%%%%%%%%%%%%%%%%%%%%%%%%%%%%%%%%%%%%%%%%%%%%%%%%%%%%%%%%
\section{Pinning}
Theorem \ref{trm:partial} implies that if relation \mR\ can be expressed as conjunctions 
of relations from \mrelset, \ccsp(\(\relset \cup \setof R\)) \(\aple\) \ccsp(\mrelset)\@.
The conditions for this theorem, even for parsimonious reductions, are too strict. 
In the rest of this chapter we provide several more lenient conditions.

The ability to tie certain CSP variables to specific values in hardness proofs is 
\emph{pinning}\@. This idea was introduced by Creignou and Hermann \cite{Nadia}. The idea
is also used in many other proofs \cite{bulatov07,Dyer,Trichotomy,madu}. Pinning can be
viewed as showing that for a constraint language \mrelset\ and a set \(S\), 
\ccsp(\(\relset \cup \setof S\)) \(\aple\) \ccsp(\mrelset) holds.

\begin{lemma} [Pinning and reflexive elements] \label{lem:refpin}
Let \mrelset\ be a set of relations on a set \mD, and let for a certain subset \(S \subset D\)
there be a relation \mR\ such that \(x\in S\) if and only if 
\((x,x,\dotsc,x)\in R\)\@. If such a relation \mR\ exists in \mrelset\ then
\ccsp(\(\relset \cup \setof S\)) \maple\ \ccsp(\mrelset)\@.
\end{lemma}

\begin{proof}
\(S\) can be viewed as a unary relation and can be expressed as a predicate 
\(\varphi(x) = R(x,x,\dotsc,x)\)\@. By Theorem~\ref{trm:partial}
\ccsp(\(\relset \cup \setof S\)) \(\aple\) \ccsp(\mrelset)\@.
\end{proof}

Next lemma generalizes the Pinning Lemma from \cite{Trichotomy}.
This lemma allows one to add an extra unary relation to the constraint language. 
The proof of Lemma~\ref{lem:pinning} also follows closely the proof in \cite{Trichotomy}. 

\begin{lemma}[Extended Pinning]\label{lem:pinning}
Let \(\relset\) be a constraint language over the set \mD\ of size \(k\)\@,
and let for a certain subset \(S \subset D\) 
there be an \(l\)-ary relation \(R \in \relset\) and a coordinate position \(j\),
\(1 \le j \le l\), such that for any \(a\in S\) the relation \mR\ has more tuples
\(\vara\) with \(\vara[j]=a\) than tuples \(\varb\) with
\(\varb[j] \notin S\)\@. Then \(\ccsp(\relset \cup \setof{S}) \aple \ccsp(\relset)\)\@.
\end{lemma}

\begin{proof} 
Fix an \(l\)-ary relation \(R \in \relset\)\@, and 
a coordinate position \(j\) such that \mR\ and \(j\) satisfy the conditions of the lemma.
Let also \(w\) be the minimal (over elements \(a\in S\)) number of tuples \(\vara\)
such that \(\vara[j]=a\), and let \(w'\) be the number of tuples \(\varb\) with
\(\vara[j] \notin S\). By the conditions of the lemma \(w'<w\)\@.

Consider an instance \(\probi=(D,V,\conset)\) of \ccsp(\(\relset \cup \setof{S}\))
with \mn\ variables. Let \(N_S\) be the set of
variables which occur in the scope of constraints of \(\conset\) with relation \(S\)\@. 
Set \(n_S = |N_S|\) and \(m = \ceil{\frac{n+2}{\lg{\frac{w}{w'}}}}\)\@.
Construct an instance \(\probi'=(D,V',\conset')\) of \(\ccsp(\relset)\) as follows:
\begin{itemize}
\item
\(V'\) includes all variables from \(V\),
and also, for each variable \(x \in N_S\), any \(u \in \setof{\oneto m}\), and any
\(v\in\setof{0,1,\dotsc,k}-\{j\}\) a fresh variable \(x_{u,v}\)\@. 
\item 
\(\conset'\) includes all constraints from \mconset\ other than those involving \(S\)\@. 
\item
For each constraint \(C=\const{S, (x)}\) from \mconset\ include \mm\ constraints whose
relation is \mR, variable \(x\) occupies the \(j\)th position in the scope,
and the variable \(x_{u,v}\) is in the \(v\)th position of the \(u\)th constraint.
\end{itemize}

Now any solution of \(\probi\) can be extended in at least \(w^{mn_S}\) 
ways to a solutions of \(\probi'\)\@, provided all variables from \(N_S\)
take values from \(S\)\@. On the other hand, every assignment that does not satisfy this
condition can be extended in at most \(w^{m(n_S-1)}w^{\prime m}\) ways.
There exist no more than \(k^n\) such solutions. Therefore if \(N\) and \(N'\)
denote the number of solutions to \(\probi\) and \(\probi'\), respectively, then
\[N\cdot w^{mn_S} \le N' \le Nw^{mn_S} + k^nw^{m(n_s-1)}w'^m.\]
So, for a properly chosen \mm, 
\[N \le \frac{N'}{w^{mn_S}} \le N + \frac{1}{4},\]
By Lemma~\ref{lem:linear} the linear transformations preserve AP-reduction
and this completes the proof.
\end{proof}

\begin{example}
Let \(R=\{(1,2), (1,3), (1,4), (2,1), (2,2), (2,3), (3,1), (3,4)\}\) be a relation
over domain \(D=\{1,2,3,4\}\) and  let \(S=\{1,2\}\)\@.

The number of tuples \((1,x)\in R\) is 3, the number of tuples \((2,x) \in R\) 
is also 3; however, the number of tuple \((x,y)\in R\) where \(x\in\{3,4\}\) is 2
which less than 3\@. By Lemma~\ref{lem:pinning}, \ccsp(\{R, S\}) is AP-reducible to
\ccsp(\{R\})\@.

As an application, we can now use this fact to deduce 
\[\mathrm{\cisp} \aple \mathrm{\ccsp(}\{R\}\mathrm{)}.\]
We first show that \cisp\(\aple\)\ccsp(\(\{R,S\}\))\@.
For any input graph \(G=(V,E)\) for the problem \cisp, take
\(\probi = (\{1,2,3,4\}, V, \conset)\) as an instance of \ccsp(\(\{R,S\}\)) where
\[\conset = \{\const{R, (x,y)}\mid xy \in E\} \cup \{\const{S, (x)} \mid x \in V\}.\]
Let \(\varphi\) be a solution of \(\probi\)\@. For any vertex \(x \in V\), \(\varphi(x)\in \{1,2\}\)\@. Thus, for any edge \(xy\) in \mG, \((\varphi(x),\varphi(y))\in 
 \{(1,2),(2,1),(2,2)\}\). Hence, there is a one-to-one correspondence between s independent sets
 \mG\ and solutions of \(\probi\)\@. Thus,
 \[\mathrm{\cisp} \aple \mathrm{\ccsp(}\{R,S\}\mathrm{)} \aple \mathrm{\ccsp(}
 \{R\}\mathrm{)}\]
\end{example}

%%%%%%%%%%%%%%%%%%%%%%%%%%%%%%%%%%%%%%%%%%%%%%%%%%%%%%%%%%%%%%%%%%%%%%%%%%%%%%%%%%%%%%%%%%%%%%%
% Maximization
%%%%%%%%%%%%%%%%%%%%%%%%%%%%%%%%%%%%%%%%%%%%%%%%%%%%%%%%%%%%%%%%%%%%%%%%%%%%%%%%%%%%%%%%%%%%%%%
\section{Maximization}
Maximization is a powerful tool to prove hardness results among many problems \ccsp(\mrelset)\@.
We believe this technique may play a role in proving a classification theorem for the approximability of the problems \ccsp(\mrelset)\@.

\begin{defi}[Max-implementation]\label{def:max}
Let \mrelset\ be a set of relations over the domain \mD,
and \mR\ be an \mn-ary relation over the same domain. 
Let \(\probi\) be in an instance of \(\ccsp(\relset)\) over the
set of variables consisting of \(V=V_x \cup V_y\), where \(V_x = \{x_1,x_2,\dots,x_n\}\)
and \(V_y=\{y_1,y_2,\dotsc,y_m\}\)\@. For any assignment \(\varphi: V_x\to D\),
let \(\#\varphi\) be the number of assignments \(\psi:V_y\to D\) such
that \(\varphi \cup \psi\) satisfy \(\probi\)\@. Let \(M\) be the maximum value of \(\#\varphi\)
among all assignments of \(V_x\)\@. The instance \(\probi\) is said to be a \emph{max-implementation}
of \mR\ if a tuple \(\varphi\) is in \mR\ if and only if \(\#\varphi = M\)\@.
\end{defi} 

If \mR\ has a max-implementation over \mrelset, we say it is \emph{max-implementable} by \mrelset\@.

\begin{example}
Let \(\probi = (\{1,2,3\}, \{x_1,x_2,y_1,y_2,y_3\}, \conset)\) be an instance
of \ccsp(\mrelset)\@. Suppose the numbers of solutions of \(\probi\) for each possible
assignment of \(x_1, x_2\) are according to the Table~\ref{tab:maxi}\@.
Here, the maximum possible value for the solutions, \(M\) in the definition, is 7.
Let \mR\ be the set of values for \(x_1,x_2\) such that
the number of solutions of \(\probi\) is 7, that is 
\[R = \{(1,1), (1,2), (1,3), (2,2), (2,3), (3,3)\}.\]
Here, \(\probi\) is a max-implementation of the relation \mR\ by the constraint language 
\mrelset\@.
\begin{table}
\begin{eqnarray*}
\begin{array}{c|ccccccccc}
x_1 & 1 & 1 & 1 & 2 & 2 & 2 & 3 & 3 & 3\\
x_2 & 1 & 2 & 3 & 1 & 2 & 3 & 1 & 2 & 3\\
\hline
\#\probi & 7 & 7 & 7 & 5 & 7 & 7 & 4 & 3 & 7\\ 
\end{array}
\end{eqnarray*}
\caption{Number of solutions of \ensuremath{\probi} for each possible assignment of \ensuremath{x_1,x_2}}\label{tab:maxi}
\end{table}
\end{example}


\begin{theorem}[Maximization]\label{theo:max}
Let \mrelset\ be a set of relations over the domain \mD,
and \mR\ be an \mn-ary relation on the same domain. 
If \mR\ is max-implementable by \mrelset\ then \ccsp(\(\relset \cup \setof R\))\maple\ccsp(\mrelset)\@.
\end{theorem}

\begin{proof}
Let \(\probi\) be a max-implementation of \mR\ (see Definition~\ref{def:max}),
For any instance \(\probi_1=(D,V_1,\conset_1)\) of \ccsp(\(\relset \cup R\))
we construct in instance \(\probi_2=(D,V_2,\conset_2)\) of \ccsp(\mrelset) as follows:
\begin{itemize}
\item Choose a sufficiently large integer \mm\ (to be determined later)
\item Let \(C_1,C_2,\dotsc, C_l \in \conset\) be constraints from \(\probi_1\) involving \mR, say
\(C_i=\const{\varrho_i, R}\)\@. Set \(V_2 = V_1 \bigcup_{i=1}^l (V_1^i \cup \dotsb V_m^i)\), where each 
\(V_j^i\) is a separate copy of \(V_y\) from Definition~\ref{def:max}\@.
\item Let \(\conset\) be the set of constraints of \(\probi\)\@. Set \(\conset_2 = (\conset_1 - \{C_1,\dotsc,C_l\}) \cup 
\bigcup_{i=1}^l(C_1^i\cup \dotsb C_m^i)\), where each \(C_j^i\) is a copy of \(\conset\) defined as follows.
For each \(\const{\varrho, Q} \in \conset\), we include \(\const{\varrho_j^i,Q}\) into \(C_j^i\)
where \(\varrho_j^i\) is obtained from \(\varrho\) by replacing every variable from \(V_y\) 
with its copy from \(V_j^i\)\@.
\end{itemize}
Now, it can be easily seen that every solution of \(\probi_1\) can be extended to a solution of \(\probi_2\)
in \(M^{lm}\) ways. Observe that sometimes the restriction of a solution \(\psi\)
of \(\probi_2\) to \(V_1\) is not a solution of \(\probi_1\)\@. Indeed, it may happen that
although \(\psi\) satisfies every copy of \(\conset_j^i\) ,
its restriction to \(\varrho_j^i\) does not belong to \mR; however, this restriction does not have
sufficiently many extensions to solutions of \(\probi_1\)\@.
On the other hand, any assignment to \(V_1\) that is not a solution to
\(\probi_1\) can be extended to a solution of \(\probi_2\) in at most \((M-1)^m\cdot M^{(l-1)m}\)
ways. Hence, 
\[M^{lm}\cdot \#\probi_1 \le  \#\probi_2  
 \le  M^{lm} \cdot \#\probi_1 + |V|^{|D|} \cdot (M-1)^{m} \cdot M ^{(l-1)m}.
 \]
The we output \(\lfloor N \rfloor\), where \(N=\#\probi_2/M^{lm}\)\@.

We want to choose \mm\ large enough such that the following holds.
\begin{eqnarray*}
|V|^{|D|} \cdot \left(\frac{M-1}{M}\right)^{m} & \le & 1 \\
\log (|V|^{|D|}) + \log\left( \left(1-\frac{1}{M}\right)^{m} \right) & \le &  0 \\
|D| \cdot \log |V| + m \cdot \log \left(1-\frac{1}{M}\right) & \le & 0 \\ 
|D| \cdot \log |V| & \le & m \cdot -\log \left(1-\frac{1}{M}\right) \\ 
\frac{|D| \cdot \log |V|}{-\log \left(1-\frac{1}{M}\right)} & \le & m.\\
\end{eqnarray*}
For any \(0<x<1\) we have \(\-log (1-x) > x \); hence, 
\[
\frac{|D| \cdot \log |V|}{-\log \left(1-\frac{1}{M}\right)} \le 
M \cdot |D| \cdot \log |V|
\]
This implies for \(m \ge M \cdot |D| \cdot \log |V|\), we have 
\(\#P=\lfloor N \rfloor\)\@.
\end{proof}

%Given a scope of variables \(\varrho\) and a predicate \(\varphi\) expressed as above definition
%\(\varphi\) we can be also viewed as a set of constraints.
\begin{comment}
\begin{theorem}[Maximization]\label{theo:max}
Let \mrelset\ be a set of relations over the domain \mD,
and \mR\ be an \mn-ary relation on the same domain. 
If \mR\ is max-definable by \mrelset\ then \ccsp(\(\relset \cup \setof R\))~\maple~
\ccsp(\mrelset)\@.
\end{theorem}

\begin{proof}
For any instance \(\probi = (D,V,\conset)\) of  \ccsp(\(\relset \cup \setof R\)) we
construct an instance \(\probi' = (D, V',\conset')\) of  \ccsp(\mrelset) by using the
predicate \(\varphi\) and the numbers \(M(M_\varphi)\) and \(k\) from Definition~\ref{def:max} as follows:
\begin{itemize}
\item 
Choose a sufficiently large integer \mm\ (to be determined later)
\item
Let \(C_1,\dotsc,C_l\in\conset\) be the constraints with \mR\ where
\(C_i=\const{R,\varrho_i}\)\@. For each \(1\le i \le l\) and for each \(1\le j\le m\), set
\(V^j_i = \{v^j_{i,1},\dotsc,v^j_{i,k}\}\) and set \(\conset^j_i\) to be the set of constraints
corresponding to \(\varphi\) as mentioned before with the scope \(\varrho_{i,1},\dotsc,\varrho_{i,n},
v^j_{i,1},\dotsc,v^j_{i,k}\)\@.
\item Set \(V'=V \cup \bigcup_{j=1}^m \bigcup_{i=1}^l V^j_i\)
\item Set \(\conset' = \conset - \setof{C_1,\dotsc,C_l} \cup 
\bigcup_{j=1}^m \bigcup_{i=1}^l \conset^j_i\)

Now, it is easily seen, every solution of \(\probi\) can be extended to a solution of \(\probi'\)
in \(M^{lm}\) ways. Observe that sometimes the restriction of a solution \(\psi\)
of \(\probi'\) to \(V\) is not a solution of \(\probi\)\@. Indeed, it may happen that
although \(\psi\) satisfies every constraint from \(\conset - \setof{C_1,\dotsc,C_l}\),
its restriction to \(\varrho_i\) does not belong to \mR; however, this restriction does not have
sufficiently many extensions to solutions of \(\probi'\)\@.
On the other hand, any assignment to \(V\) that is not a solution to
\(\probi\) can be extended to a solution of \(\probi\) in at most \((M-1)^m\cdot M^{(l-1)m}\)
ways. Hence, 
\[M^{lm}\cdot \#\probi \le  \#\probi'  
 \le  M^{lm} \cdot \#\probi + |V|^{|D|} \cdot (M-1)^{m} \cdot M ^{(l-1)m}.\]
The we output \(\lfloor N \rfloor\), where \(N=\#\probi'/M^{lm}\)\@.

We want to choose \mm\ large enough such that the following holds.
\begin{eqnarray*}
|V|^{|D|} \cdot \left(\frac{M-1}{M}\right)^{m} & \le & 1 \\
\log (|V|^{|D|}) + \log\left( \left(1-\frac{1}{M}\right)^{m} \right) & \le &  0 \\
|D| \cdot \log |V| + m \cdot \log \left(1-\frac{1}{M}\right) & \le & 0 \\ 
|D| \cdot \log |V| & \le & m \cdot -\log \left(1-\frac{1}{M}\right) \\ 
\frac{|D| \cdot \log |V|}{-\log \left(1-\frac{1}{M}\right)} & \le & m.
\end{eqnarray*}
For any \(0<x<1\) we have \(\-log (1-x) > x \); hence, 
\[
\frac{|D| \cdot \log |V|}{-\log \left(1-\frac{1}{M}\right)} \le 
M \cdot |D| \cdot \log |V|
\]
This implies for \(m \ge M \cdot |D| \cdot \log |V|\), we have 
\(\#P=\lfloor N \rfloor\)
\end{itemize}
\end{proof}
\end{comment}

As an example of the applications of the theorem we mention a corollary bellow.

Let \mR\ be a relation of arity \(l\) over \mD\@.
For \(x\in D\), we denote by \(s_{R,i}(x)\) the number of tuples in \mR\ in which \(x\) is in position \(i\)\@.

\begin{cor}  \label{cor:degree}
If there exists a set \(S\subseteq D\) and real numbers \(a_1,a_2,\dotsc,a_l, M\), such that 
for every \(x\in S\) we have
\[\sum_{i=1}^l a_i\cdot s_{R,i}(x) = M,\]
and for every \(x\not\in S\) we have
\[\sum_{i=1}^l a_i\cdot s_{R,i}(x) < M,\]
then \(\ccsp(\{R, S\}) \aple \ccsp(R)\)\@.
\end{cor}

