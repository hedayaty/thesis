\chapter{CSP and \ccsp} 
The Constraint Satisfaction Problem (CSP) is the problem of deciding if it is possible to
assign values to some variables such that given constraints are satisfied. \ccsp\ 
is the problem of finding the number of possible assignments for a set of constraints.
Consider the following example:

\begin{example}[\tcoloring]\label{exm:3col}
The \tcoloring\ problem is the problem of coloring vertices of the input graph \(G\)
with colors \(\setof{r,g,b}\) such that no two adjacent vertices are colored the same color.
This problem can be formulated with a set of constraints as follows.
Let \(V=\setof{v_1,v_2,\dotsc,v_n}\)
be a set of variables where each variable corresponds to a vertex of
the input graph \(G\)\@. The goal is to find a function
\(f:V\to \setof{r,g,b}\) such that for each edge \((u,v)\) of \(G\), \(f(u)\neq f(v)\) holds.
\end{example}

In the above example,
the CSP is the problem of deciding
if there is a function satisfying all the given constraints. \ccsp\ is the
problem of finding the number of such functions. Note that any CSP can then be expressed as 
deciding if the answer for \ccsp\ with the same set of constraints is greater zero.

%%%%%%%%%%%%%%%%%%%%%%%%%%%%%%%%%%%%%%%%%%%%%%%%%%%%%%%%%%%%%%%%%%%%%%%%%%%%%%%%%%%%%%%%%%%%%%%
%%%%%%%%%%%%%%%%%%%%%%%%%%%%%%%%%%%%%%%%%%%%%%%%%%%%%%%%%%%%%%%%%%%%%%%%%%%%%%%%%%%%%%%%%%%%%%%

Let \(D\) be a set of elements; any subset of \(D^k\) is a \emph{relation} of arity \(k\) with
domain \(D\)\@. A \emph{constraint language} with domain \(D\) is a set of
relations with domain \(D\)\@.
In this work domains and relations are always finite.

A CSP instance \(\mathcal{P}\) with constraint language \mrelset\ is a tuple 
\((D,V,\conset)\) where:
\begin{itemize}
\item \(D\) is a set of values which is the domain of \mrelset,
\item \(V\) is a set of variables,
\item \mconset is a collection of constraints where each constraint consists of a scope \(\varrho\),
which is a tuple of variables from \(V\) and a relation \(R\) from \mrelset\
of the same arity as \(\varrho\)\@. A constraint is represented as \(\const{R, \varrho}\)\@.
\end{itemize}

An assignment for \(\mathcal{P}\) is a function \(\varphi\) from \(V\) to \(D\)\@. The assignment
\(\varphi\) is satisfying if the scope of each constraint is
mapped to a tuple of the corresponding relation.

\begin{defi}[CSP(\mrelset)]
For a fixed constraint language \mrelset, CSP(\mrelset) is the problem of deciding 
whether a CSP instance \(\mathcal{P}\) with constraint language \mrelset\
has a satisfying assignment.
\end{defi}

For a constraint language \mrelset\ that consists of only a single relation \(R\),
we use CSP(\(R)\) instead of CSP(\mrelset) for simplicity.

\begin{example}[\tcoloring]\label{exm:3colcsp}
We continue with the \tcoloring\ problem from Example~\ref{exm:3col}.
Let \mG\ be the input graph for the \tcoloring\ problem.
One way to formulate \tcoloring\ problem using a CSP(\mrelset) is as follows.
Let \(D=\setof{r,g,b}\) be the domain and \(NEQ_{rgb}\), the binary dis-equality relation on domain \(D\)\@, consists of all tuples from \(D^2\) except the ones whose elements are equal, that is
\[NEQ_{rgb}=\setof{(r,g),(r,b),(g,r),(g,b),(b,r), (b,g)}\]

Let \(\varphi\) be an assignment which is a function from vertices of \mG\ to \mD\@.
For each edge \((u,v)\) of \mG\, \(\varphi(u)\neq\varphi(v)\) holds which is
equivalent to \(NEQ_{rgb}(u,v)\)\@.

Let \mG\ be the graph in Figure~\ref{fig:sample}. The instance of 
CSP for \tcoloring\ on \mG\ is as follows. The set of variables is the set of vertices of \mG,
and for each edge of \mG\ a constraint with relation \(NEQ_{rgb}\) is placed.
\begin{align*}
\mathcal{P}=(&D=\setof{1,2,3},V=\setof{v_1,v_2,v_3,v_4,v_5}, \\
&\conset=\{\const{(v_1,v_2),NEQ_{rgb}}, \const{(v_2,v_3),NEQ_{rgb}},\const{(v_3,v_4),NEQ_{rgb}},\\
&\const{(v_4,v_5),NEQ_{rgb}}, \const{(v_5,v_1),NEQ_{rgb}},
\const{(v_2,v_3),NEQ_{rgb}}\})
\end{align*}
\end{example}

\begin{figure}
\centering
\input{figs/sample.pdftex}
\caption{Graph \ensuremath{G} used in Example~\ref{exm:3colcsp}}
\label{fig:sample}
\end{figure}

\begin{example}[\tsat]\label{exm:3sat}
The problem of deciding if there exists a truth assignment for a 
given Boolean CNF (Conjunctive Normal Form) formula where each
clause contains exactly 3 literals is called \tsat\@.

Let \(R_1=\{(x,y,z)\mid x,y,z\in \setof{T,F}, x\lor y\lor z = T\}\), 
let \(R_2=\{(x,y,z)\mid x,y,z\in \setof{T,F}, \bar x\lor y\lor z = T\}\), 
and let \(R_3,R_4\dotsc,R_8\) be defined similarly for other combinations of literals overs
\(x,y,z\)\@. Let \(\relset_1=\setof{R_1,R_2,\dotsc, R_8}\) be a constraint language.

It is easy to see that there is one-to-one correspondence between constraints 
in a CSP(\(\relset_1\)) instance and clauses of \tsat\@. Hence,
the CSP(\(\relset_1\)) and the \tsat\ problems are the same.
\end{example}

%%%%%%%%%%%%%%%%%%%%%%%%%%%%%%%%%%%%%%%%%%%%%%%%%%%%%%%%%%%%%%%%%%%%%%%%%%%%%%%%%%%%%%%%%%%%%%%
%%%%%%%%%%%%%%%%%%%%%%%%%%%%%%%%%%%%%%%%%%%%%%%%%%%%%%%%%%%%%%%%%%%%%%%%%%%%%%%%%%%%%%%%%%%%%%%

Another type of problems defined by constraint satisfaction are
counting problems which involve finding the number of solutions for a CSP instance.
The number of solutions of a CSP instance can be used to express many computational problems
such as partition functions which are mentioned in Section~\ref{sec:appl}.

\begin{defi}[\ccsp(\mrelset)] 
For a constraint language \mrelset,
\ccsp(\mrelset) is the problem of finding the number of satisfying assignments for a
given CSP instance with \mrelset\ as the constraint language.
\end{defi}

Here are two examples of counting problems which can be expressed as \ccsp(\mrelset) for some
constraint language \mrelset\@.
\begin{example}[\ctcol]
We continue with the \(NEQ_{rgb}\) relation from Example~\ref{exm:3colcsp}\@.
The problem of finding the number of 3-Colorings of a given graph denoted by 
\ctcol, can be expressed as the \ccsp(\(NEQ_{rgb}\)) problem.
\end{example}

\begin{example}\label{exm:c3sat}
The problem of finding the number of satisfying truth assignments for a given 
Boolean CNF (Conjunctive Normal Form) formula where each clause contains exactly 3 literals
is called \ctsat\@. 

It is easy to see that with the constraint language \(\relset_1\) from Example~\ref{exm:3sat}
the \ctsat\ and the \ccsp(\(\relset_1\)) problems are the same.
\end{example} 
%%%%%%%%%%%%%%%%%%%%%%%%%%%%%%%%%%%%%%%%%%%%%%%%%%%%%%%%%%%%%%%%%%%%%%%%%%%%%%%%%%%%%%%%%%%%%%%
%%%%%%%%%%%%%%%%%%%%%%%%%%%%%%%%%%%%%%%%%%%%%%%%%%%%%%%%%%%%%%%%%%%%%%%%%%%%%%%%%%%%%%%%%%%%%%%
\section{Applications of the \ccsp} \label{sec:appl}
In this section we will show several applications of \ccsp\ and how computational problems
can be formulated with \ccsp\@. First, we will describe graph
homomorphism problem which can be viewed as a special case of CSP; next, we will relate
the number of homomorphisms between two graphs to the graph isomorphism problem;
we will also relate the number of homomorphisms between two graphs to
the graph reconstruction problem. We will also describe partition functions
and give examples of partition functions in two common models in statistical physics.

%%%%%%%%%%%%%%%%%%%%%%%%%%%%%%%%%%%%%%%%%%%%%%%%%%%%%%%%%%%%%%%%%%%%%%%%%%%%%%%%%%%%%%%%%%%%%%%
\subsection*{Graph Homomorphism}
Let \mG\ and \mH\ be two graphs. A homomorphism from \mG\ to \mH\ is a mapping \(h\) from
vertices of \mG\ to vertices of \mH\ such that for every edge \((u,v)\) of \mG,
\((h(u),h(v))\) is an edge of \mH\@. If there is homomorphism from \mG\ to \mH\ 
we shall write \(G\to H\), and \(G \not\to H\) means that there is no homomorphism from
\mG\ to \mH\@.

\begin{example}\label{exm:hom}
Let \mG\ and \mH\ be the graphs shown in Figure~\ref{fig:hom}\@. The mapping \(h\)
defined as \(h(a)=1, h(b)=2, h(c)=3, h(d)=1\)
is not a homomorphism because \((a,d)\) is mapped to \((1,1)\) which is not an 
edge of \mH\ but the mapping \(h'\) defined as \(h'(a)=1, h'(b)=2, h'(c)=1, h'(d)=2\) 
is a homomorphism because all the edges of \mG\ are mapped to edges of \mH. 
Note that a homomorphism is not necessarily surjective.

\begin{figure}[h]
\center\input{figs/hom.pdftex}
\caption{Graph \ensuremath{m} and \ensuremath{H} used in Example~\ref{exm:hom}} \label{fig:hom}
\end{figure}
\end{example}

\begin{defi}[Hom(\mH)]
For a fixed graph \mH, Hom(\mH) is the problem of deciding whether a given graph \mG\
is homomorphic to \mH\@.
\end{defi}

\begin{example}[\tcoloring]
We continue with the \tcoloring\ problem from Example~\ref{exm:3col}; however, in this example 
we express it as a graph homomorphism problem. As usual we denote the complete graph
on \(k\) vertices by \(K_k\)\@. For all complete graphs, \((u,v)\) is an edge of \(K_k\)
if an only
if \(u\neq v\)\@. This shows that \(Hom(K_k)\) is the same as
the k-coloring problem. As a special case, \(\chom(K_3)\) is the same as \ctcol\ problem.
\end{example}

Let \pname{NoRB-3-Coloring} be the \pname{3-Coloring} problem with an additional
restriction that the vertices colored blue are not allowed to be connected 
to the vertices colored red. In order to formulate this problem with graph homomorphisms,
obtain \mH\ from \(K_3\) as follows. Label the vertices with \(\setof{r,g,b}\)\@.
Delete the edge between \(r\) and \(b\). The Hom(\mH) problem is the same as
the \pname{N0RB-3-Coloring} problem.
This example illustrates that the graph homomorphism problem generalizes the graph coloring problem.
This is why
the Hom(\mH) problem is also called the \pname{H-Coloring} problem.
More details and examples on graph homomorphism 
can be found in the book ~\cite{hellbook} by Hell and Ne\v{s}et\v{r}il.

\begin{obs}\label{obs:hom}
Let \mG\ and \mH\ be two graphs.
If \mH\ contains a loop (reflexive vertex) then \(G \to H\)\@.
If \mG\ is bipartite then \(G \to H\) if and only if \mH\ is has at least one edge.
If \mH\ is bipartite then \(G \to H\) if and only if \mG\ is bipartite.
\end{obs}

Observation~\ref{obs:hom} shows that if the graph \mH\ has a loop or is bipartite,
the Hom(\mH) problem is easy; however, for other graphs the problem is of higher complexity.

\begin{theorem} [Hell and Ne\v{s}et\v{r}il 1990~\cite{pavol}]
For an undirected graph \mH, the Hom(\mH) problem is polynomial time solvable if 
\mH\ contains a loop or \mH\ is bipartite; otherwise, it is NP-complete.
\end{theorem}

\begin{defi}[\chom(\mH)]
For a fixed graph \mH,
\chom(\mH) is the problem of finding the number of homomorphisms from a given 
input graph \mG\ to \mH.
\end{defi}

Dyer and Greenhill proved the following dichotomy for the \chom(\mH)\ problem.
\begin{theorem} [Dyer and Greenhill 2000 \cite{Dyer}]
For a graph \mH, if each component of \mH\ is either a complete reflexive graph
or a complete irreflexive bipartite graph then the \chom(\mH) problem 
is polynomial time solvable; otherwise, it is \cpc\@.
\end{theorem}

Hom(\mH) and \chom(\mH) are special cases of CSP(\mrelset) and \ccsp(\mrelset),
respectively. In both cases, \mrelset\ consists of a single binary relation which is the
edge set of \mH\ with vertex set of \mH\ as the domain.
Hence, we may use Hom and \chom\ instead of
the corresponding CSP and \ccsp, respectively. 

%%%%%%%%%%%%%%%%%%%%%%%%%%%%%%%%%%%%%%%%%%%%%%%%%%%%%%%%%%%%%%%%%%%%%%%%%%%%%%%%%%%%%%%%%%%%%%%
\subsection*{Lov\'{a}sz Vectors}
The number of homomorphisms from \mG\ to \mH\ is denoted by \(\hom(G,H)\); this function
can be used to define the \emph{Lov\'{a}sz vector} which is used to characterize graphs.
Lovasz \cite{lovasz} proved that
two graphs \(H_1\) and \(H_2\) are isomorphic if and only if for any graph \mG,
\(\hom(G,H_1)=\hom(G,H_2)\). 
For an enumeration of all non-isomorphic graphs \((G_1,G_2,\dotsc)\), 
the Lovasz vector of a graph \mH\ is the infinite sequence consisting of the number of
homomorphisms from graphs in the sequence to \mH, i.e.,
\((\hom(G_1,H), \hom(G_2,H), \dotsc)\)\@. 
In other words Lovasz's theorem indicates two graphs are isomorphic if and only if
they have the same Lovasz vector.  
\begin{comment}
Lov\'{a}sz and et al. \cite{propertytesting} used the Lov\'{a}sz vector
for \emph{Edge Reconstruction} problem over graphs with sufficiently many edges, 
They also used Lov\'{a}sz vector in the {\em Property Testing} area.
In this model there is usually a huge graph and at each step only small sample of vertices
can be examined. The goal is to infer parameters of the graph using small samples of it. 
A good example of such a graph one might want to find the properties is the Internet.
\end{comment}

%%%%%%%%%%%%%%%%%%%%%%%%%%%%%%%%%%%%%%%%%%%%%%%%%%%%%%%%%%%%%%%%%%%%%%%%%%%%%%%%%%%%%%%%%%%%%%%
\subsection*{Partition Functions}
Let \mH\ be a graph with a weight function \(w: V_H \cup E_H\to \mathbb{R}\)\@.
For a given graph \mG, a configuration of \mG\ is a mapping \(\sigma\) from \(V_G\) to
\(V_H\). For each configuration \(\sigma\), the weight of the configuration is
\[w(\sigma)=\prod_{uv\in E_G}w(\sigma(u)\sigma(v))\prod_{v\in V_G}w(\sigma(v))\]
The partition function \(Z_{H,w}(G)\) is the sum of the weights for all the configurations:
\[Z_{H,w}(G)=\sum_{\sigma:V_G\to V_H}w(\sigma)\]

Note that \(\hom(G,H)=Z_{H,w}(G)\) if \(w(v)=w(e)=1\) for all \(v,e\in H\)
and \(w(e)=0\) for all \(e\not\in H\)\@. 
The \pname{H-Partition} Problem is defined as:

\pnndef%{\pname{H-Partition}}
{A graph \(G\)}
{The value of partition function \(Z_{H,w}(G)\)}

%%%%%%%%%%%%%%%%%%%%%%%%%%%%%%%%%%%%%%%%%%%%%%%%%%%%%%%%%%%%%%%%%%%%%%%%%%%%%%%%%%%%%%%%%%%%%%%
\subsection*{Sampling}
A \emph{Uniform Generation} problem involves finding a
random solution for a given instance such that for any two possible solutions
the probabilities if being chosen are the same.
An \emph{Almost Uniform Generation} involves finding 
a solution for a given instance such that
for any two possible solutions the probabilities if being chosen are approximately the same.
Jerrum et al. \cite{JVV} explained that
for practical purposes, it is impossible to distinguish
Almost Uniform Generation from Uniform Generation by experiments running in polynomial time.

Intuitively, a problem is said to be \emph{self-reducible} if solutions for an instance can be
expressed in terms of solutions for a number of smaller instances of the same problem.
Many of the combinatorial problems such as the finding the number of matchings in the graph,
finding the number of satisfying truth assignment for a CNF(Conjunctive Normal Form) or
DNF(Disjunctive Normal Form) are self-reducible.
An approximate counting problem involves finding an answer for a given
instance within an acceptable range of the actual number of solutions.
Jerrum et al. \cite{JVV} proved that 
for self-reducible problems, Almost Uniform Generation and approximate counting 
are interreducible. 

The model used in \cite{JVV} is very general, we provide a very informal description of the
proof for the \ccsp(\mrelset) problem. For solving an approximate counting using Almost
Uniform Generator we randomly generate some solutions. Since the problem is self-reducible,
it can be be expressed in terms of solutions for a number of smaller instances. We choose
one the small instances and from the solutions we have already generated measure the
ratio of solutions satisfying that cases to the total. Next, recursively solve the 
same problem and from the estimation for sub-problem we estimate the problem.

For the other side, we use a similar approach. We approximately count the number of solutions
for each sub-problem. Then branch to that problem with probability proportional to the number of
solutions. Next, we recursively generate the rest of the problem in the same manner.

\emph{Sampling} is an extension of the Uniform Generation; given a
distribution \(\pi\) over solutions, the goal is to find a solution \(\sigma \)
from distribution \(\pi\)\@.
Let's get back to graph homomorphism problem. 
The Gibbs distribution on homomorphisms from \mG\ to \mH\ is
the distribution that each homomorphism \(h\) has a probability proportional to its weight.
\[\pi_{H,w,G}(h)=\frac{w(h)}{Z_{H,w}(G)}\]

The \pname{H-GibbsSample} problem is defined as:

\pnndef%{\pname{H-GibbsSample}}
{A graph \mG}
{An H-coloring of \mG\ chosen from distribution \(\pi_{H,w,G}\)}

Dyer et al. \cite{Sampling} have proved that if \pname{H-GibbsSample} can be approximately 
sampled in polynomial time then \pname{H-Partition}
can also be approximated by a randomized algorithm in polynomial time. 
%The \chom(\mH) problem is not a self-reducible problem; however, it is very similar to
%self-reducible problems. 

%%%%%%%%%%%%%%%%%%%%%%%%%%%%%%%%%%%%%%%%%%%%%%%%%%%%%%%%%%%%%%%%%%%%%%%%%%%%%%%%%%%%%%%%%%%%%%%
\subsection*{Statistical Mechanics}
Statistical Mechanics is a branch of physics that applies probability theory to
predict the behaviour of a system at a given temperature. Usually there is a complex system
consisting of many microscopic elements. A state or configuration \(\sigma\) is an assignment of
parameters of the microscopic elements. 
The \emph{Hamiltonian} \(H(\sigma)\) is the energy of the system in state \(\sigma\)\@. 
Let \(\beta\) be the inverse (one over) temperature. 
The \emph{Partition Function} at a given temperature is defined as
\[Z=\sum_\sigma e^{-\beta H(\sigma)}\]
Given a model of the system and the temperature,
many of the important properties of the system such as the free energy, the entropy, the specific heat, and the location of phase transition can be evaluate using the the partition function.

One of the commonly used models in Statistical Mechanics is the Ising model. In the Ising mode,
the system is defined by a graph \(G(V,E)\)\@.
Each edge of the graph \((u,v)\in E\) has an {\em interaction strength} \(J_{u,v}\)\@. Each
vertex has an associated \emph{local external magnetic field} \(l_v\)\@.
A configuration of the system  is an assignment \(\sigma: V\to \setof{-1,+1}\) of spins to
vertices of \mG. Each configuration is associated with energy:
\[H(\sigma)=-\sum_{(u,v)\in E}J_{u,v}\cdot\sigma(u)\cdot\sigma(v)-\sum_{v\in V} l_v\cdot\sigma(v)\]
the partition function for the Ising model is:
\begin{align*}
Z(G,\beta,J,l)&=\sum_{\sigma:V\to\setof{-1,+1}}e^{-\beta H(\sigma)}\\
&=\sum_{\sigma:V\to\setof{-1,+1}} \prod_{(u,v)\in E}
e^{\beta J_{u,v}\sigma(u)\sigma(v)} \prod_{v\in V} 
e^{\beta l_v\sigma(v)}
\end{align*}

To avoid exponentials in the notation, let \(\lambda_{u,v}=e^{2\beta J_{u,v}}\) and 
\(\mu_{v}=e^{2\beta l_v}\). The partition function will be:

\begin{align*}
Z(G,\lambda,\mu)&=\sum_{\sigma :V\to\setof{-1,+1}} \prod_{(u,v)\in E} \lambda_{u,v}^{\frac{1}{2}
\sigma(u)\sigma(v)}\prod_{v\in V}\mu_v^{\frac{1}{2}\sigma(v)}\\
&=\prod_{(u,v)\in E}\lambda_{u,v}^{-\frac{1}{2}}\prod_{v\in V}\mu_v^{-\frac{1}{2}}\sum_{\sigma:V\to
\setof{-1,+1}}
\prod_{\substack{(u,v)\in E \\ \sigma(u)=\sigma(v)}}\lambda_{u,v}\prod_{v\in V:
\sigma(v)=+1}\mu_v
\end{align*}

The system is \emph{ferromagnetic} if each interaction energy \(J_{u,v}\) is non-negative; 
which implies for all \(u,v\) in \(V\), we have \(\lambda_{u,v} \ge 1\)\@.
A system is \emph{consistent} if
either for all \(v\), \(\mu_v\ge 1\) or for all \(v\), \(\mu_v \le 1\)\@. We will
mention results on complexity of these cases in TODO. \todo{Move the results to proper section
and fix the reference}
%Jerrum and Sinclair \cite{Jer93} have proved that the problem of finding the partition
%function for the Ising model is hard to compute.

%Ashkin-Teller model generalizes the Ising model to four possible states.
The Potts model generalizes the Ising model to \(q\) possible spins.
In the Potts model there is an underlying graph \(G(V,E)\) and 
each configuration is an assignment \(\sigma:V\to \setof{\oneto{q}}\)\@.
Each edge of the graph \((u,v)\in E\) has an interaction
strength \(J_{u,v}\) and each vertex \(v\) at state \(c\) is associated with an
external field \(h_{v,c}\)\@. The energy of a configuration $\sigma$ is:
\[H(\sigma)=-\sum_{(u,v)\in E} J_{u,v}\chi(\sigma(u),\sigma(v)) - \sum_{v\in V} 
h_{v,\sigma(v)}\]
where
\[\chi(s,s')=\begin{cases}+1, & s=s'\\-1, & otherwise\end{cases}\]
As before, let \(\lambda_{u,v}=e^{2\beta J_{u,v}}\) and
\(\mu_{u,c}=e^{\beta h_{v,c}}\)\@. The partition function for the Potts model is:
\[Z(G,\lambda,\mu)=\prod_{(u,v)\in E}\lambda_{u,v}^{-\frac{1}{2}}
\sum_{\sigma:V\to\setof{\oneto q}} \prod_{(u,v)\in E:\sigma(u)=\sigma(v)}\lambda_{u,v}
\prod_{v\in V}\mu_{v,\sigma(v)}\]

Problems of computing partition functions in the Ising and Potts models can be
reduced to a \ccsp(\mrelset) for some constraint language \mrelset; however, the reduction
may not be straightforward.
\todo{Compose a small section from this part and previous results and \cite{Goldberg2007}
The refer to that part}


%%%%%%%%%%%%%%%%%%%%%%%%%%%%%%%%%%%%%%%%%%%%%%%%%%%%%%%%%%%%%%%%%%%%%%%%%%%%%%%%%%%%%%%%%%%%%%%
%%%%%%%%%%%%%%%%%%%%%%%%%%%%%%%%%%%%%%%%%%%%%%%%%%%%%%%%%%%%%%%%%%%%%%%%%%%%%%%%%%%%%%%%%%%%%%%
%%%%%%%%%%%%%%%%%%%%%%%%%%%%%%%%%%%%%%%%%%%%%%%%%%%%%%%%%%%%%%%%%%%%%%%%%%%%%%%%%%%%%%%%%%%%%%%
\chapter{Reductions and Complexity of Counting}
In this chapter, we will mention several complexity classes for computational functions.
We will define clones and polymorphisms and specify how they are related to
the complexity of the CSP(\mrelset) problem. We will mention the major results on the complexity of 
\ccsp(\mrelset) problem. We will also define partial clones and specify
how they are related to the complexity of the \ccsp(\mrelset) problem.

Two important classes of functions we are interested in are FP and \cp\@.
FP is the class of functions computable in polynomial time and \cp\ is
the class of function that can be expressed as the
number of accepting paths of a non-deterministic polynomial time Turing machine. 

Let \(f,g:\Sigma^*\to\mathbb{N}\) be two functions. A \emph{parsimonious reduction}
from \mf\ to \mg\ is a polynomial time computable function
\(\sigma:\Sigma^*\to\Sigma^*\) such that
\(f(x)=g(\sigma(x))\) holds. 
A Turing reduction from \mf\ to \mg\ is an polynomial time algorithm
that computes \mf\ using an oracle of \mg\@.

\begin{defi}[\cpC ness] 
A problem \(f\) is \cpc\ if it is a member of \cp\ and every problem in \cp\ is 
Turing reducible to \(f\) in polynomial time.
\end{defi}

Consider the SAT and \csat\ problems defined as follows.
Note that the SAT and the \csat\ problems are generalizations of the \tsat\ and
the \ctsat\ problems
mentioned in Examples~\ref{exm:3sat}~and~\ref{exm:c3sat}\@. 

\pdef{SAT}
{A CNF formula \(\varphi\)}
{Existence of a truth assignment that satisfies \(\varphi\)}

\pdef{\csat}
{A CNF formula \(\varphi\)}
{The number of truth assignments that satisfy \(\varphi\)}

In the same way that Cook \cite{cook1971} proved that SAT is NP-complete,
Valiant \cite{Valiant1979} showed that \csat\ is \cpc\@. It is easy to see that 
Cook's reduction is a parsimonious reduction; hence, it works for counting problems
as well. 

%%%%%%%%%%%%%%%%%%%%%%%%%%%%%%%%%%%%%%%%%%%%%%%%%%%%%%%%%%%%%%%%%%%%%%%%%%%%%%%%%%%%%%%%%%%%%%%
%%%%%%%%%%%%%%%%%%%%%%%%%%%%%%%%%%%%%%%%%%%%%%%%%%%%%%%%%%%%%%%%%%%%%%%%%%%%%%%%%%%%%%%%%%%%%%%
\section{Clones}
In instances of CSP, constraints can be expressed either explicitly or implicitly by 
interactions of other constraints. Consider the following example.

\begin{example}
Let \(NEQ_{01}=\setof{(0,1),(1,0)}\) be a binary relation over the set
\(\setof{0,1}\)\@. Consider the following instance of 
CSP(\(NEQ_{01}\)) :
\[\mathcal{P}=(D=\setof{0,1},V=\setof{x,y,z},
 C=\setof{\const{NEQ_{01},(x,z)},\const{NEQ_{01}(y,z)}}\]
There is no explicit constraint with scope \((x,y)\)\@; however, there is the implicit 
constraint \(\const{EQV_{01},(x,y)}\) where \(EQV_{01}=\setof{(0,0),(1,1)}\)\@.
\end{example}

\begin{defi}[pp-definition]
Let \mrelset\ be a constraint language with domain \(D\) and 
let \(R\) be a relation with the same domain.
Relation \(R\) is said to be primitive positive (pp)-definable by \mrelset\
if \(R\) can be expressed as a predicate using relations from \mrelset,
the relation \(EQV_D\) which is the binary equality relation over \(D\),  conjunctions, and
existential quantification.
\end{defi}

\begin{example} \label{exm:ppd} 
Let \mrelset\ be a constraint language with domain \(D=\setof{a,b,c}\), 
consisting of a single relation 
\(R=\setof{(a,a,a),(b,b,b),(a,b,a),(a,b,c)}\)\@.
Relations 
\(R_1=\setof{a,b}\), 
\(R_2=\setof{(a,a),(b,b),(a,b)}\), 
and
\(R_3=\setof{(a,a,a),(a,a,c),(b,b,b)}\) are pp-definable by \mrelset\ as follows:
\begin{align*}
&R_1(x)=\exists z.R(x,x,z)\\
&R_2(x,y)=\exists z.R(x,y,z)\\
&R_3(x,y,z)=\exists s,t.R(x,t,z)\land R(z,s,y)
\end{align*}
\end{example}

The notion of  pp-definition is equivalent to elementary operators used in \cite{post41,bkkr69,lau},
and strict implementations used in \cite{madu,Trichotomy}\@.

A set of relations closed under pp-definition is called a \emph{co-clone}.
For a constraint language \mrelset, \(\clone{\relset}\) is the the least co-clone containing
\mrelset\ also called the co-clone generated by \mrelset\@.

A function \(f\) is said to be a \emph{projection} function if \(f(x_1,\dotsc,x_l)=x_l\) 
for some \(l\)\@.
A set of functions is said to be a \emph{clone} if it contains all projection
functions and is closed under superpositions (compositions). For a set of functions
\(C\), \(\clone{C}\) is the least clone containing \(C\) also called the clone generated by \(C\)\@.

A Galois connection between sets \(A\) and \(B\) is a pair of functions
\(\fA:\pow{A} \to \pow{B}\) and \(\gB:\pow{B} \to \pow{A}\) 
such that they maintain {\em antitony}, that is,
\begin{align*}
X \subseteq X' \implies \fA(X) \supseteq \fA(X') \\
Y \subseteq Y' \implies \gB(Y) \supseteq \gB(Y') \\
\end{align*}
and {\em extensivity}, that is,
\begin{align*}
X \subseteq \gB(\fA(X)) \\
Y \subseteq \fA(\gB(Y))
\end{align*}

Let \mR\ be a relation over domain \mD\ and \(f:D^n\to D\) a function of arity \mn\ over the
same domain. Function \mf\ \emph{preserves} \mR\ or is a \emph{polymorphism} of \mR\ if
for any \mn\ tuples \(\ba_1,\ba_2,\dotsc,\ba_n\) in \mR, the tuple
\(f(\ba_1,\ba_2,\dotsc,\ba_n)\) obtained by component-wise application of \mf,
is in \mR\@. The relation \mR\ in this case is said to be \emph{invariant} with respect 
to \mf\@.
The set of all functions preserving a constraint language \mrelset\ is 
denoted by \(Pol(\relset)\), the set of all relations invariant with respect to a set of
functions  \(C\) is denoted by \(Inv(C)\)\@.

Functions \(Inv\) and \(Pol\) satisfy antitony and extensivity; hence, they 
form a \emph{Galois connection} between the sets of functions 
and the set of relations. The following theorem links the clones to \(Inv\) and \(Pol\)
functions.

\begin{theorem}[Geiger \cite{gei68}, Romov et al. \cite{bkkr69}]
For any constraint language \mrelset\ and 
any set of functions \(C\), 
we have \(Inv(Pol(\relset))=\clone{\relset}\)
and
\(Pol(Inv(C))=\clone{C}\)
\@.
\end{theorem}

\begin{cor}
Since the \(Pol\) and \(Inv\) functions form a Galois connection, for any
constraint language \mrelset\ and any set of functions \(C\), 
\(Pol(\relset)\) and \(Inv(C)\) are co-clones and clones, respectively.
\end{cor}

\begin{cor}
Let \mrelset\ and \(\relset'\) be two constraint languages with the same domain.
If \(Pol(\relset) \subseteq Pol(\relset')\) then every relation from \(\relset'\)  is 
pp-definable in \mrelset\@.
\end{cor}

\begin{theorem} [Jeavons et al. 1997 \cite{Jeavons}] \label{trm:jeavons}
For a constraint language \mrelset\ and a relation \(R\) with the same domain,
if \mR\ is pp-definable in \mrelset\ then CSP(\mrelset) is polynomial-time equivalent 
to CSP(\(\relset \cup \setof R\))\@.
\end{theorem}

In order to reduce CSP(\(\relset \cup \setof R\)) to CSP(\mrelset),
for each instance \(\mathcal{P}=(D,V,C)\) of CSP(\(\relset \cup \setof R\)),
create an instance \(\mathcal{P}'=(D,V',C')\) of CSP(\mrelset) as follows.
Include all variables of \(V\) in \(V'\)\@.
For all relations \(R'\) in \mrelset, include constraints \(\const{R',\varrho}\) from \(C\)
in \(C'\)\@. For all constraints \(\const{R,\varrho}\) in \(C\), we use \(\psi\)
the pp-definition of \mR\ in \mrelset\ to replace \mR\ in \(\mathcal{P}'\);
for each existential quantifier in \(\psi\), add the quantified variables to \(V'\);
for each \(EQV_D(x,y)\) relation in \(\psi\), replace all occurrences of \(y\) with \(x\);
the rest of \(\psi\) consists of conjunction of clauses in the form \(R'(\varrho)\) where
\(R'\) is a relation in \mrelset\ and \(\varrho\) is a scope of variables from \(V'\);
for each clause \(R'(\varrho)\) in \(\psi\), add a constraint \(\const{R',\varrho}\) to \(C'\)\@.
For each satisfying assignment \(\varphi\) of \(\mathcal{P'}\),
the restriction of \(\varphi\) to \(V\) is a satisfying assignment for \(\mathcal{P}\),\@
and for each satisfying assignment \(\varphi\) of \(\mathcal{P}\),
there is at least one extension of \(\varphi\) to \(V'\) which
is a satisfying assignment for \(\mathcal{P}'\)\@.

Note that this reduction preserves the existence of an satisfying assignment
but does not necessarily preserve the number of satisfying assignments.

\begin{cor}
For any \(\relset'\) a finite subset of \(\clone{\relset}\),
the CSP(\(\relset'\)) is polynomial time reducible to  CSP(\mrelset);
consequently, the complexity of the CSP(\mrelset) only depends on the \(Pol(\relset)\)\@.
\end{cor}

The CSP dichotomy conjecture states for any constraint language \mrelset, 
the CSP(\mrelset) problem is either polynomial time solvable or NP-complete.
Bulatov and Valeriote \cite{recent08}
have a survey of results on the relation between complexity of CSP(\mrelset) and \(Pol(\relset)\).
There are more results in \cite{dur628, valeriote}\@.
The CSP dichotomy still remains an open problem. 

%%%%%%%%%%%%%%%%%%%%%%%%%%%%%%%%%%%%%%%%%%%%%%%%%%%%%%%%%%%%%%%%%%%%%%%%%%%%%%%%%%%%%%%%%%%%%%%
%%%%%%%%%%%%%%%%%%%%%%%%%%%%%%%%%%%%%%%%%%%%%%%%%%%%%%%%%%%%%%%%%%%%%%%%%%%%%%%%%%%%%%%%%%%%%%%
\section{Exact Counting}
In the previous section we mentioned the relation between polymorphisms and complexity of
the CSP(\mrelset) problem; in this section we will establish a similar connection 
between polymorphisms and complexity of \ccsp(\mrelset) and mention 
results on the complexity of \ccsp(\mrelset)\@.

Consider the following problem:

\pdef{\cdcol}
{A graph \(G\)}
{The number of 2-Colorings of \(G\)}

The \cdcol\ problem can be solved in polynomial time as follows.
For an input graph \mG, if \mG\ is not bipartite then the answer is \(0\); otherwise,
it is \(2^m\) where \(m\) is the number of connected components of \mG\@.

The \cdcol\ problem can be expressed as \ccsp(\(NEQ_{01}\)), where
\(NEQ_{01}\) is the inequality relation over a two element set, that is,
\(NEQ_{01}=\setof{(0,1),(1,0)}\)\@. 
The \cdcol\ problem is easy to solve because
the \(NEQ\) relation belongs to a family of relations called affine relations.
A relation is affine if it is expressible by a system of linear equations over a
finite fields  of size \(k\) for some \(k\)\@. The relation \(NEQ_{01}\) is an 
affine relation because it can be represented as \(x\oplus y \equiv 1 \pmod{2}\)\@.

\begin{example}
Relation \mR\ defined as \(R=\setof{(0,1),(1,0),(2,2)}\) is affine because it can be expressed as 
\(\setof{(x,y)\mid x \oplus y \equiv 1 \pmod{3}}\)\@.
\end{example}

Alternatively, affine relations are the relations invariant under functions of the form
\(f(x,y,z)=xy^{-1}z\) of a group.

For any affine relation \mR, the \ccsp(\mR) problem can be solved in polynomial time
as follows. If the system of linear equations is inconsistent
then there is no solutions; otherwise, there are \(k^m\) solutions where \(m\) is the
dimensionality of the solution space of the system of linear equations and
\(k\) is the size of the finite field used to express \mR\@.

The simplest type of constraint languages are those 
over a domain of size two. They are usually 
referred to as \emph{Boolean} constraint languages.
\ccsp(\mrelset) for Boolean \mrelset\ is often referred as Boolean \ccsp(\mrelset)\@.
Creignou and Hermann proved a dichotomy for the Boolean \ccsp(\mrelset) problem.

\begin{theorem} [Creignou and Hermann 1996 \cite{Nadia}] 
For a Boolean constraint language \mrelset,
if  \mrelset\ is affine then \ccsp(\mrelset) 
is polynomial time solvable; otherwise, it is \cpc\@.
\end{theorem}

Previously, we mentioned the dichotomy theorem for complexity of \ccsp\ in  Section~\ref{sec:appl}.

\begin{theorem} [Dyer and Greenhill 2000 \cite{Dyer}]
For a graph \mH, if each component of \mH\ is a complete reflexive graph
or a complete irreflexive bipartite graph then the \chom(\mH) problem 
is polynomial time solvable; otherwise, it is \cpc\@.
\end{theorem}

Creignou and Hermann \cite{Nadia} proved that a Boolean constraint language \mrelset\
is affine if and only if every relation \mR\ in \mrelset\
is closed under the function \(f(x,y,z)=x\oplus y \oplus z\). It is also easy to see that
a graph \mG\ is complete reflexive if and only if the edge set of \mG\ is closed under any
polymorphisim. Similarly, a bipartite graph \(G\) is complete if and only if the orientation of edges
of \(G\) from one part to other part is closed under any polymorphisim.

Hence, polymorphisims can express criteria used in both theorems. This suggests
that \(Pol(\relset)\) can express the complexity of the \ccsp(\mrelset) problem.
Suppose \(\relset_1\) and \(\relset_2\) are two constraint languages such that
every relation in \(\relset_2\) is pp-definable in \(\relset_1\); 
the pp-definition suggests a reduction from CSP(\(\relset_2\))
to CSP(\(\relset_1\))\@. However, this reduction is not usable for counting problems. 
Bulatov and Dalmau provided a reduction that relates 
the complexity of the \ccsp(\mrelset) problem to \(Pol(\relset)\)\@.

\begin{theorem}[Bulatov, Dalmau 2007 \cite{bulatov07}] 
For a constraint language \mrelset\ and a relation \(R\) with the same domain,
if \mR\ is pp-definable in \mrelset\ then \ccsp\((\relset)\) is polynomial-time equivalent 
to \ccsp\((\relset \cup \setof R)\)\@.
\end{theorem}

This theorem is analogous to Theorem~\ref{trm:jeavons}. In the same manner Jeavons's theorem
links complexity of CSP(\mrelset) to clones, this theorem
links complexity of the \ccsp(\mrelset) problems to clones. This connection
led to a dichotomy for the \ccsp(\mrelset) problem.

Let \mrelset\ be a constraint language with domain \mD\ and 
let \mR\ be a \(k\)-ary relation pp-definable in \mrelset\@.
A \emph{congruence} of \mR\ is a \(2k\)-ary relation \(Q\) which is also pp-definable in \mrelset\
and satisfying the following conditions:
\begin{inparaenum}[(a)]
\item \(Q\) can be viewed as a binary relation on \mR, i.e., \(Q \subseteq R^2\);
\item \(Q\) viewed as a binary relation on \mR\ is an equivalence relation.
\end{inparaenum}


Now, let \(Q,Q_1,Q_2\) be congruences of \mR\ such that
\(Q\subseteq Q_1,Q_2\). Let \(A_1,\dotsc,A_m\) and \(B_1,\dotsc,B_n\) be
the equivalence classes of \(Q_1\) and \(Q_2\), respectively. 
\(M(R;Q_1,Q_2;Q)\) denotes a matrix where \(M_{ij}\) is the number of \(Q\)-classes in
\(A_i\cap B_j\). 

A constraint language \mrelset\ is said to be \emph{congruence
 singular} if for any pp-definable relation \mR\ in \mrelset\ and any congruences
\(Q,Q_1,Q_2\) of \mR\ with \(Q\subseteq Q_1,Q_2\), the
\emph{row rank} of  matrix \(M(R;Q_1,Q_2;Q)\) equals
the number of classes of the smallest equivalence relation containing
both \(Q_1\) and \(Q_2\)\@. 

\begin{theorem}[Bulatov 2008 \cite{Bulatov}]
For a constraint language \mrelset,
the \ccsp(\mrelset) problem is polynomial time solvable if 
\mrelset\ is congruence singular; otherwise, it is \cpc\@.
\end{theorem}

Bulatov proved a dichotomy for all \mrelset\ but the decidability of being congruence
singular remained open until Richerby and Dyer \cite{DyerR10} 
proved that being congruence singular can be verified in polynomial time.

\begin{theorem} [Dyer and Richerby 2010 \cite{DyerR10}]
For a constraint language \mrelset, checking if \mrelset\ is congruence singular is in NP.
\end{theorem}

%%%%%%%%%%%%%%%%%%%%%%%%%%%%%%%%%%%%%%%%%%%%%%%%%%%%%%%%%%%%%%%%%%%%%%%%%%%%%%%%%%%%%%%%%%%%%%%
%%%%%%%%%%%%%%%%%%%%%%%%%%%%%%%%%%%%%%%%%%%%%%%%%%%%%%%%%%%%%%%%%%%%%%%%%%%%%%%%%%%%%%%%%%%%%%%
\section{Weak Co-clones}
In the previous section we showed the connection between polymorphisms and complexity 
of the CSP(\mrelset) and \ccsp(\mrelset) problems. The existence of a similar relationship
between polymorphism and complexity of approximate counting is still an open
question. However, there are weaker tools that can be used for the approximate counting
problems. Removing the existential quantifier from pp-definitions limits the 
reduction to parsimonious reductions which preserve approximation.


For a domain \mD, a set of relations closed under conjunctions and
containing \(EQV_D\), the binary equality relation over \mD,
is a weak co-clone. For a constraint language \mrelset, 
\(\wclone{\relset}\) is the least weak co-clone containing \mrelset,
also called the weak co-clone generated by \mrelset\@.

For a partial function \(f\), the set of all tuples from \(D^n\) on which \(f\) is
defined is called the \emph{domain} of \(f\) and denoted by \(\dom(f)\).
A set of partial functions \(C\) is said to be \emph{down-closed} if
for every function \(f\) in \(C\), \(C\) contains any function \(f'\) such that \(\dom(f')\subseteq\dom(f)\) and \(f'(a_1,a_2,\dotsc,a_n)=f'(a_1,a_2,\dotsc,a_n)\)
for every tuple \((a_1,a_2,\dotsc,a_n)\in \dom(f')\)\@.
A down-closed set of partial functions, 
containing all projections and closed under 
superpositions is called a \emph{partial clone}. 

Let \mR\ be a relation with domain \(D\) and \(f:D^n\to D\) be a partial function of arity \(n\)
over the same domain.
Function \mf\ is a \emph{partial polymorphism} for \mR,
if for any \mn\ tuples \(\ba_1,\ba_2,\dotsc,\ba_n\) in \mR\, 
if  \((\ba_1,\ba_2,\dotsc,\ba_n) \in dom(f)\) then 
\(f(\ba_1,\ba_2,\dotsc,\ba_n)\) is also in \mR\@. Relation \mR\ in this case is said to
be \emph{invariant} with respect to \mf\@.
The set of all partial polymorphisms of a constraint language \mrelset\ is denoted by
\(pPol(\relset)\) and the set of all relations invariant under a set of partial functions
\(C\) is denoted by \(Inv(C)\)\@. Note that the \(Inv\) function for partial functions is
an extension of the \(Inv\) function for total functions.

\begin{theorem} [Creignou et al. \cite{madu}] \label{trm:partial}
Let \(\relset_1\),\(\relset_2\) be two sets of relations over the same domain; if 
\(\relset_2\) is a finite subset of \(\wclone{\relset_1}\) then
\ccsp(\(\relset_2\)) is parsimoniously reducible to \ccsp(\(\relset_1\))\@.
\end{theorem}

Fleischer and Rosenberg \cite{Rosenberg} proved that for any constraint language \mrelset\
and any set of functions \(C\),
we have \(Inv(pPol(\relset))=\wclone{\relset}\) and
\(pPol(Inv(C))=\wclone{C}\)\@. On the other hand, \(pPol\) and \(Inv\) functions 
maintain extensivity and antitony; hence, they form Galois connection between 
sets of relations and partial functions. Due to properties of Galois connections,
for any constraint language \mrelset\ and any set of functions \(C\), the 
sets \(Inv(C)\) and \(pPol(\relset)\) are a weak co-clone and a partial clone, respectively.

Theorem \ref{trm:partial} indicates that weak co-clones are
in charge of complexity of approximation of the \ccsp(\mrelset) problem.
The relation between weak co-clones and partial polymorphisms,
implies that the complexity of approximation of the \ccsp(\mrelset) problem 
depends on \(pPol(\relset)\)\@.

%%%%%%%%%%%%%%%%%%%%%%%%%%%%%%%%%%%%%%%%%%%%%%%%%%%%%%%%%%%%%%%%%%%%%%%%%%%%%%%%%%%%%%%%%%%%%%%
%%%%%%%%%%%%%%%%%%%%%%%%%%%%%%%%%%%%%%%%%%%%%%%%%%%%%%%%%%%%%%%%%%%%%%%%%%%%%%%%%%%%%%%%%%%%%%%
%%%%%%%%%%%%%%%%%%%%%%%%%%%%%%%%%%%%%%%%%%%%%%%%%%%%%%%%%%%%%%%%%%%%%%%%%%%%%%%%%%%%%%%%%%%%%%%
\chapter{Approximate Counting}
In this chapter we formally define FPRAS, the complexity class which
is considered as the efficient computational model for counting problems.
We define AP-reductions and give a classification of computational problems 
with respect to AP-reductions. We use the definition of FPRAS from \cite{counting}\@. 

A \emph{randomized approximation scheme} (RAS) for a function \(f:\Sigma^*\to\nat\)
is a probabilistic algorithm that for an input \((x,\eps)\in\Sigma^*\times(0,1)\) 
where \(x\) is an instance of \mf\ and \(\eps\) is the error tolerance, 
produces an integer random variable \(z\) such that 
\[Pr\left(\left|\frac{z-f(x)}{f(x)}\right|\le \epsilon\right) \ge \frac{3}{4}.\]
A randomized approximation scheme is said to be \emph{fully polynomial} if it runs in time \(poly(|x|,\frac{1}{\eps})\)\@. 
The phrase ``fully polynomial randomized approximation scheme'' is usually abbreviated to 
\emph{FPRAS}\@. The complexity class FPRAS is referred to problems which have FPRAS\@.

Note that there is no significance in the constant \(\frac{3}{4}\) in the definition,
other than being in \((\frac{1}{2},1)\) interval. 
Jerrum et al. \cite{JVV} proved that any success probability 
greater than \(\frac{1}{2}\) can be improved to \(1-\delta\) for any desired \(\delta\) by
\(O(\log \delta^{-1})\) trials of algorithm and taking the median of the results.

APX is another complexity class that is regarded as an efficient computational model for
optimization problems.
For a function \(f:\Sigma^*\to\nat\) and a constant factor \(\alpha\),
\mf\ has an \(\alpha\)-APX if there is an algorithm that
takes \(x\in\Sigma^*\) as an input and in time \(poly(|x|)\) produces an integer \(z\)
such that \(\frac{1}{\alpha}\le\frac{z}{f(x)}\le \alpha\). In the same manner,
poly-APX and log-APX are algorithms that given an input
\(x\), in time \(poly(|x|)\) find a solution with an approximation ratio of \(poly(|x|)\)
and \(\log(|x|)\), respectively. There is no results on \ccsp(\mrelset) with
APX schemas. With the next lemma, I explain why APX schemas are not used for \ccsp(\mrelset) problems.

\begin{lemma}[Hedayaty 2012]
For a constraint language \mrelset, if there is a poly-APX for
the \ccsp(\mrelset) then there is an FPRAS for this problem.
\end{lemma}

\begin{proof}
Let \(\apxalg\) be an algorithm and \(T\) be a polynomial function such
that for any instance \(\probi\)
of \ccsp(\mrelset), we have 
\[\frac{1}{T(|\probi|)} \le \frac{\apxalg(\probi)}{\#\probi} \le T(|\probi|)\]
Choose \(k\) a sufficiently large number whose value
will be determined later. Let \(\probi'\) be \(k\) copies of \(\probi\)\@.
We know that \(\#\probi'=\#\probi^k\); \(\sqrt[k]{\apxalg(\probi')}\) is 
an (\(1+\eps\))-approximation for \(\probi\)\@ if \(1+\eps > \sqrt[k]{(T(|\probi|\cdot k)}\)\@.
For that, it is sufficient for \(k\) to be greater than
\(\oneoeps\cdot \log T(|\probi|\cdot k)\)\@.
Since the function \(T\) is polynomial in \(|\probi|\), \(k\) is also polynomially 
bounded by \(|\mathcal{P}|\) and \(\oneoeps\)\@.
\end{proof}

\begin{defi}[AP-reduction]
For any two functions \mf\ and \mg, 
an \emph{approximation-preserving reduction}(AP-reduction for short)
from \mf\ to \mg\ is a probabilistic algorithm \(\apxalg\) that for an input 
\((x,\eps) \in \Sigma^*\times(0,1)\) where \(x\) is an instance of \mf\ and \(\eps\) 
is an error tolerance, using \(\apxalg'\) produces an integer random variable \(z\)
satisfying the following  conditions: \begin{inparaenum}[(i)] \item \(\apxalg'\) takes 
an input in the form \((w,\delta) \in \Sigma^*\times(0,1)\) where \(w\) is an instance of \mg\
and \(\delta\) is the error tolerance, \item if \(\apxalg'\) meets the specification for
being a RAS for \mg\ then \(\apxalg\) meets the specifications for being a RAS for \mf, \item
\(\apxalg\) runs in \(poly(|x|,\oneoeps)\) time.
\end{inparaenum}
\end{defi}

If an approximation-preserving reduction from \mf\ to \mg\ exists we write \(f \aple g\)
and say \mf\ is \emph{AP-reducible} to \mg. If \(f \aple g\) and \(g \aple f\)
then we say that  \mf\ and \mg\ are \emph{AP-interreducible} and write \(f \apeq g\)\@. 


%%%%%%%%%%%%%%%%%%%%%%%%%%%%%%%%%%%%%%%%%%%%%%%%%%%%%%%%%%%%%%%%%%%%%%%%%%%%%%%%%%%%%%%%%%%%%%%
%%%%%%%%%%%%%%%%%%%%%%%%%%%%%%%%%%%%%%%%%%%%%%%%%%%%%%%%%%%%%%%%%%%%%%%%%%%%%%%%%%%%%%%%%%%%%%%
\section{Polynomial Time Solvable Problems}
A very few non-trivial combinatorial problems involving counting 
can be solved in polynomial time.
Some of the problems are reducible to determinant such as
the problem of finding the number of spanning trees of a graph and 
the problem of finding the number of perfect matchings in a planar graph.
Bulatov \cite{Bulatov} showed that for only congruence singular \mrelset, the \ccsp(\mrelset)
problem  is polynomial time solvable. In this section we show several
examples of counting problems which an approximate solution can be obtained in polynomial time.

There are some artificial examples of \ccsp(\mrelset) problems that can be approximated 
in polynomial time. 

\begin{example} \label{exp:poly}
For the graph \mH\ shown in Figure~\ref{fig:approxible},
we show that the \chom(\mH) problem is in FPRAS\@. Suppose the input is 
\((G,\eps)\) where \mG\ is a connected graph with \mn\ vertices and \(\eps\)
is the error tolerance.
We know that 
\(4^n\le hom(G,H) \le 4^n+3^n\)\@.
For \(\eps > (\frac{3}{4})^n\), take \(4^n\) as an approximation, the error ratio will  be
\[\frac{\hom(G,H)-4^n}{\hom(G,H)} \le \frac{3^n}{\hom(G,H)} \le (\frac{3}{4})^n \le \eps\]
For \(\eps < (\frac{3}{4})^n\), we have \(\oneoeps>(\frac{4}{3})^n \);
hence, \((\oneoeps)^{\log_{\frac{4}{3}}3} > 3^n\); consequently, \(3^n\) is polynomial 
in terms of \(\oneoeps\); which means the number of 3-colorings of \mG\ can be evaluated in 
\(poly(n,\oneoeps)\) time. 
\(hom(G,H)\) is \(4^n\) plus the number of 3-colorings of \mG\@.

\begin{figure}[h]
\center\input{figs/approxible.pdftex}
\caption{Graph \ensuremath{G} used in Example~\ref{exm:poly}}
\label{fig:approxible}
\end{figure}
\end{example}

Another interesting counting problem which can be approximated in polynomial time is 
the \pname{\#Dual-SAT} problem defined as :

\pnndef %{\pname{\#Dual-SAT}}
{A Boolean DNF (Disjunctive Normal Form) formula \(\varphi\)}
{The number of satisfying assignments for \(\varphi\)}

There are different algorithms to solve this problem.
Luby and Veli\v{c}kovic \cite{Luby} provided a deterministic algorithm for
approximating the probability of a random assignment being satisfying. 
Madras et al. \cite{Madras} provided a Monte-Carlo algorithm
for this problem. Jerrum et al. \cite{JVV} provided an algorithm,depicted in Algorithm~\ref{alg:dnf},
for sampling the \pname{Dual-SAT} problem. 
\begin{algorithm}[h]
\begin{algorithmic}
\WHILE {true}
	\STATE  select a clause \(C\) randomly proportional to the size of the clause
	\STATE  select an assignment \(A\) satisfying \(C\) uniformly random
	\STATE  \(N\)=\#clauses satisfied by \(A\)
	\STATE with probability 1/N: output \(A\) and halt
\ENDWHILE
\end{algorithmic}
\caption{An algorithm for sampling \pname{Dual-SAT}}\label{alg:dnf}
\end{algorithm}

Since the \pname{\#Dual-SAT} problem is self-reducible, Algorithm~\ref{alg:dnf} can
be used to find an approximate solution for \pname{\#Dual-SAT}\@.

Other common technique used to approximate counting problems is the Markov chain
Monte-Carlo algorithms. Consider the following problems:

\pdef{\pname{\#Match}}
{A graph \mG}
{The number of matchings in \mG}

\newcommand{\ldkcol}{\#\pname{LowDegree}-k-\pname{Coloring}}
\pdef{\ldkcol}
{A graph \mG\ such that \(2\Delta(G)+1\le k\), where \(\Delta(G)\)
is the maximum degree in \mG}
{The number of k-Colorings of \mG}

Jerrum and Sinclair \cite{Jerrum96} used Markov chain Monte-Carlo technique 
to approximate the \pname{\#Match} problem; Jerrum \cite{Jerrum} used the same technique 
to approximate the \ldkcol\ problem for graphs 
with bounded degree. We give a brief description of the Markov chain Monte Carlo algorithm
and show how these two problems are solved.
Consider a Markov chain with each state representing a solution of the problem.
The transitions are to the same solution with probability of \(\frac{1}{2}\) 
and to a similar solution with probability \(\frac{1}{2}\).

There is usually a sequence of instances (the length of sequence is polynomial) such that
the first element of the sequence is the instance  that we are trying to solve
and the last element is a trivial instance.
The ratio of answers between two consecutive elements 
is estimated by sampling along a random walk.
The final answer is the product of the ratios and the answer for trivial instance. 

In order for Markov chain Monte-Carlo to work, the chain must be ergodic and rapidly mixing.
Mixing time is the minimum length of a random walk such that the required
quantity is measured with desired approximation and if mixing
time is polynomial then the chain is said to be rapidly mixing.

Instead of \pname{\#Match}, the following problem which is a weighted version of
the \pname{\#Match} problem is solved:

\pdef{Monomer-dimer}
{A graph \mG\ and a positive real number \(\lambda\)}
{The partition function \(Z_G(\lambda)\) which is the sum of
\(\lambda^{|M|}\) for all matchings \(M\) of \mG}

Note that \(Z_G(1)\) is the number of matchings in \mG\ and \(Z_G(0)=1\) because
for the empty matching \(M\), \(\lim_{\lambda\to 0^+}\lambda^{|M|}=1\).
The \pname{\#Match} problem is solved as follows.
A suitable sequence \(\bl\) of real numbers beginning with one and ending
with zero is chosen. The ratios \(Z_G(\bl_i)/Z_G(\bl_{i+1})\) are estimated 
by sampling along a random walk on the Markov chain. \(Z_G(1)\) is the product of
all the ratios.

The \ldkcol\ problem is solved as follows. Let \mG\ be a graph with \mn\ vertices
and \mm\ edges. Let \(\bG\) be a sequence of subgraphs of \mG\ where
\(\bG_0=G\) and \(\bG_{i+1}\) is obtained from \(\bG_i\) by removing a single edge. 
The last element \(\bG_m\) is an empty graph which is k-colorable in \(k^n\) ways.
The ratio \(Z(G_i)/Z(G_{i+1})\) is estimated
by sampling along a random walk on the Markov chain. \(Z(G)\) is the product of all the
ratios and the number of k-colorings of \(\bG_m\).


%%%%%%%%%%%%%%%%%%%%%%%%%%%%%%%%%%%%%%%%%%%%%%%%%%%%%%%%%%%%%%%%%%%%%%%%%%%%%%%%%%%%%%%%%%%%%%%
%%%%%%%%%%%%%%%%%%%%%%%%%%%%%%%%%%%%%%%%%%%%%%%%%%%%%%%%%%%%%%%%%%%%%%%%%%%%%%%%%%%%%%%%%%%%%%%
\section{Problems Hard To Approximate} \label{sec:hard}
In this section we study a set of counting problems that regarding complexity
of approximation are the hardest problems in \cp, in other words every
problem in \cp\ is AP-reducible to them.
Existence of an FPRAS for any of these problems implies NP=RP\@. The following theorem 
points us to the first set of problems in this complexity class.

\begin{theorem} [Goldberg et al. \cite{Leslie03}]\label{theorem:trichotomy}
For any NP-complete decision problem, the corresponding counting problem is complete
for \cp\ with respect to AP-reducibility.
\end{theorem}

\begin{cor}
\csat, \ctsat, and \ctcol\ are complete for \cp\ with respect to AP-reducibility.
\end{cor}

Another important problem AP-interreducible with the \csat\ problem is the \cis\ problem defined as:

\pnndef%{\cis}
{A graph \mG}
{The number of independent sets in \mG}

Although the decision version of the \cis\ problem is a trivial problem, Goldberg et al.
\cite{Leslie03} proved that the \cis\ problem is AP-interreducible to the \csat\ problem. 
Finding AP-reductions from many problems from the \cis\ problem is more straightforward than
finding reductions from the problems that the decision version is NP-complete.

\begin{figure}[h]
\centering 
\subfigure[\ensuremath{Wr_0}]{\input{figs/Wr0.pdftex}\label{fig:Wr0}}\hspace{4.78cm}
\subfigure[\ensuremath{Wr_1}]{\input{figs/Wr1.pdftex}\label{fig:Wr1}}\\
\subfigure[\ensuremath{Wr_2}]{\input{figs/Wr2.pdftex}\label{fig:Wr2}}\hspace{4cm}
\subfigure[\ensuremath{Wr_3}]{\input{figs/Wr3.pdftex}\label{fig:Wr3}}
\caption{Wrench graphs Family}
\label{fig:wrench}
\end{figure}

\begin{example}
Consider the binary relations \(OR\) and \(NAND\) on domain \(\setof\zo\) defined as
\(OR=\setof{(0,1),(1,0),(1,1)}\) and \(NAND=\setof{(0,0),(0,1),(1,0)}\)\@.
The \cis\ problem can be expressed by the \ccsp(\(OR\)) or \ccsp(\(NAND\)) problem;
hence, the \ccsp(\(OR\)) and \ccsp(\(NAND\)) problems are both AP-interreducible with the
\csat\ problem.

Although, the \dsat\ problem (the decision problem) is polynomial time
solvable, the \cdsat\ problem can express \ccsp(\(OR\)) and
\ccsp(\(NAND\)); hence, the \cdsat\ problem 
is also AP-interreducible with the \csat\ problem.
\end{example}

\begin{example}[\sc\#Wrench-Col] \label{exm:wrench}
Let \(Wr_q\) be the graph with vertex set \(V_q=\setof{a,b,c_1,c_2,\cdots,c_q}\)
and edge set \(E_q=\setof{\setof{a,b},\setof{b,b}} \cup 
\setof{\setof{b,c_i},\setof{c_i,c_i}:1\le i\le q}\)\@.
Graphs \(Wr_0\), \(Wr_1\), \(Wr_2\), and \(Wr_3\) are shown in Figure~\ref{fig:wrench}.
The \chom(\(Wr_q\)) problem is referred to as the \#q-{\sc Wrench-Col} problem. 
Goldberg et al. \cite{Leslie03} proved that for \(q\neq 2\), the  \#q-{\sc Wrench-Col} problem
is AP-interreducible with the \csat\ problem.
\end{example}

\begin{example} [\#q-\pname{Particle-WR-Configs}] \label{exm:particles}
For \(q \ge 1\), homomorphisms to \(S^*_q\) where
\(S^*_q\) is a q-leaf star with loops at all the \(q+1\)
vertices are configuration in the q-particle Widom-Rowlinson model.
The \#q-\pname{Particle-WR-Configs} problem is defined as the \chom(\(S^*_q\)) problem.
Graphs \(S^*_1\), \(S^*_2\), \(S^*_3\), and \(S^*_4\) are shown in Figure~\ref{fig:wrconfigs}.
 
The \chom(\(S^*_1\)) problem is polynomial time solvable.
We will consider the \chom(\(S^*_2\)) and \chom(\(S^*_3\)) problems later;
for \(q \ge 4\), the \chom(\(S^*_q\)) problem is proven to be AP-interreducible
with the \csat\ problem \cite{Leslie03}.
\end{example}

\begin{figure}[h]
\centering 
\subfigure[\ensuremath{S^*_1}]{\input{figs/wr1.pdftex}\label{fig:ss1}}\hspace{5cm}
\subfigure[\ensuremath{S^*_2}]{\input{figs/wr2.pdftex}\label{fig:ss2}}\\
\subfigure[\ensuremath{S^*_3}]{\input{figs/wr3.pdftex}\label{fig:ss3}}\hspace{5cm}
\subfigure[\ensuremath{S^*_4}]{\input{figs/wr4.pdftex}\label{fig:ss4}}
\caption{Particle-WR-Configs}
\label{fig:wrconfigs}
\end{figure}

Other examples of problems hard to approximate are Potts model
and antiferromagnetic Ising model which are defined in Section~\ref{sec:appl}. 
Jerrum and Goldberg \cite{Goldberg2007} showed these problems are 
AP-interreducible with the \csat\ problem.

%%%%%%%%%%%%%%%%%%%%%%%%%%%%%%%%%%%%%%%%%%%%%%%%%%%%%%%%%%%%%%%%%%%%%%%%%%%%%%%%%%%%%%%%%%%%%%%
%%%%%%%%%%%%%%%%%%%%%%%%%%%%%%%%%%%%%%%%%%%%%%%%%%%%%%%%%%%%%%%%%%%%%%%%%%%%%%%%%%%%%%%%%%%%%%%
\section{Problems AP-interreducible With \cbis}
The proof used to show that the \csat\ problem is AP-reducible to the \cis\ problem does
not work if the input for the \cis\ problem is limited to bipartite graphs.
The version of the \cis\ problem with input limited to bipartite graphs is called
the \cbis\ problem and is defined as follows:

\pnndef%{\cbis}
{A bipartite graph \mG}
{The number of independent sets in \mG}

The following problems AP-interreducible with \cbis:

\begin{figure}[h]
\centering
\subfigure[\ensuremath{\vec{P}^*_2}]{\input{figs/ds.pdftex}\label{fig:ds}}\hfill 
\subfigure[\ensuremath{P_4}]{\input{figs/p4.pdftex}\label{fig:p4}}\hfill
\subfigure[Oriented \ensuremath{P_4}] {\input{figs/dp4.pdftex}\label{fig:dp4}}\\
%\subfloat[$P^*_4$]{\input{figs/ps4.pdftex}\label{fig:ps4}}\hfill
\subfigure[\ensuremath{\vec{P}^*_3}]{\input{figs/dps3.pdftex}\label{fig:dps3}}\hfill
\subfigure[\ensuremath{P^*_k}]{\input{figs/psk.pdftex}\label{fig:psk}}
\caption{Some of the problems AP-interreducible with \cbis}
\label{fig:bisred}
\end{figure} 

\pdef{\cds}
{A partial order \((P,\preceq)\)}
{The number of down-sets in \(P\)}

\pdef{\pname{\#1P1N-SAT}}  
{A Boolean CNF formula \(\varphi\) with clauses of size one or two such that
there is at most one negative and at most one positive literal per clause}
{The number of satisfying assignments of \(\varphi\)}

Note that the number of down sets in a partial order equals
the number of anti-chains in the same partial order. 

For many graphs, the \chom(\mH) problem is also AP-interreducible with the \cbis\ problem.
For example, the \chom(\mH) problem is AP-interreducible with the \cbis\ problem
if \mH\ is any of the graphs \(\vec{P}^*_2\), \(P_4\), \(\vec{P}_4\), \(\vec{P}^*_3\),
or \(P^*_k\)(\(k\ge 3\)), which are shown in Figure \ref{fig:bisred}\@.
 
There are also problems from statistical physics that are AP-interreducible with the \cbis\
problem. The Ising model is described in Section~\ref{sec:appl}.
The \pname{\#2-Wrench-Coloring}
and \pname{\#2-Particles-WR-Configs} problems are described in Examples~\ref{exm:wrench}~
and~\ref{exm:particles}, respectively.

\pdef{\pname{Ferromagnetic Ising}}
{A graph \mG, inverse temperature \(\beta\), interaction strengths \(J\) such that
\(J_{u,v} > 0\), local external magnetic field \(l\)}
{The partition function \(Z(G,\beta,J,l)\)}

\pdef{\pname{\#2-Wrench-Coloring}}
{A graph \mG}
{The number of homomorphisms from \mG\ to \(WR_2\)}

\pdef{\pname{\#2-Particles-WR-Configs}}
{A graph \mG}
{The number of homomorphisms from \mG\ to \(S^*_2\)}


%%%%%%%%%%%%%%%%%%%%%%%%%%%%%%%%%%%%%%%%%%%%%%%%%%%%%%%%%%%%%%%%%%%%%%%%%%%%%%%%%%%%%%%%%%%%%%%
%%%%%%%%%%%%%%%%%%%%%%%%%%%%%%%%%%%%%%%%%%%%%%%%%%%%%%%%%%%%%%%%%%%%%%%%%%%%%%%%%%%%%%%%%%%%%%%
\section{Other Difficult Problems}
Goldberg et al.~\cite{Trichotomy} proved an approximation trichotomy for the Boolean
\ccsp(\mrelset) problem. A Boolean relation \mR\ is \emph{monotone} if \mR\ is closed under
\(\land\) and \(\lor\) operators; a Boolean constraint languages \mrelset\ is monotone if 
every relation in \mrelset\ is monotone.


\begin{theorem}[Dyer, Goldberg, Jerrum, 2007 \cite{Trichotomy}]
For a constraint language \mrelset,
if \mrelset\ is affine then \ccsp(\mrelset) is polynomial time solvable; otherwise,
if \mrelset\ is monotone then it is AP-interreducible to the \cbis\ problem; otherwise,
it is AP-interreducible with the \csat\ problem.
\end{theorem}

Despite the approximation trichotomy for the Boolean \ccsp(\mrelset) problem,
the non-Boolean \ccsp(\mrelset) problem seems to have more complexity classes.
So far, our knowledge on complexity classes of the \ccsp(\mrelset) problem 
is very limited. We are not aware if the approximation
complexity classes for this problem are finite or even countable.
We know a few problems that are only proven to be harder than
the \cbis\ problem \cite{Leslie03} and currently, 
we have no knowledge on how hard these problems are.

Here are some of these problems. \pname{\#3-Particles-WR-Configs} was introduced in
Example~\ref{exm:particles}.

\pdef{\pname{\#3-Particles-WR-Configs}}
{A graph \mG}
{The number of homomorphisms from \mG\ to \(Wr_3\)}

\pdef
{\pname{\#Bipartite} q-\pname{Coloring} for \(q\ge 3\)}
{A bipartite graph \mG}
{The number of q-Colorings of \mG}
%%%%%%%%%%%%%%%%%%%%%%%%%%%%%%%%%%%%%%%%%%%%%%%%%%%%%%%%%%%%%%%%%%%%%%%%%%%%%%%%%%%%%%%%%%%%%%%
%%%%%%%%%%%%%%%%%%%%%%%%%%%%%%%%%%%%%%%%%%%%%%%%%%%%%%%%%%%%%%%%%%%%%%%%%%%%%%%%%%%%%%%%%%%%%%%
%%%%%%%%%%%%%%%%%%%%%%%%%%%%%%%%%%%%%%%%%%%%%%%%%%%%%%%%%%%%%%%%%%%%%%%%%%%%%%%%%%%%%%%%%%%%%%%
%%%%%%%%%%%%%%%%%%%%%%%%%%%%%%%%%%%%%%%%%%%%%%%%%%%%%%%%%%%%%%%%%%%%%%%%%%%%%%%%%%%%%%%%%%%%%%%
%%%%%%%%%%%%%%%%%%%%%%%%%%%%%%%%%%%%%%%%%%%%%%%%%%%%%%%%%%%%%%%%%%%%%%%%%%%%%%%%%%%%%%%%%%%%%%%
%%%%%%%%%%%%%%%%%%%%%%%%%%%%%%%%%%%%%%%%%%%%%%%%%%%%%%%%%%%%%%%%%%%%%%%%%%%%%%%%%%%%%%%%%%%%%%%
%%%%%%%%%%%%%%%%%%%%%%%%%%%%%%%%%%%%%%%%%%%%%%%%%%%%%%%%%%%%%%%%%%%%%%%%%%%%%%%%%%%%%%%%%%%%%%%
%%%%%%%%%%%%%%%%%%%%%%%%%%%%%%%%%%%%%%%%%%%%%%%%%%%%%%%%%%%%%%%%%%%%%%%%%%%%%%%%%%%%%%%%%%%%%%%
%%%%%%%%%%%%%%%%%%%%%%%%%%%%%%%%%%%%%%%%%%%%%%%%%%%%%%%%%%%%%%%%%%%%%%%%%%%%%%%%%%%%%%%%%%%%%%%
%%%%%%%%%%%%%%%%%%%%%%%%%%%%%%%%%%%%%%%%%%%%%%%%%%%%%%%%%%%%%%%%%%%%%%%%%%%%%%%%%%%%%%%%%%%%%%%
%%%%%%%%%%%%%%%%%%%%%%%%%%%%%%%%%%%%%%%%%%%%%%%%%%%%%%%%%%%%%%%%%%%%%%%%%%%%%%%%%%%%%%%%%%%%%%%
%\nocite{*}
