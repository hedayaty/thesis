\chapter{Conclusion and Future Work}
\section{Conclusion}
We formally defined the problem \ccsp(\mrelset), mentioned some of its applications; mentioned the 
major results on complexity of the problem \ccsp(\mrelset). We defined the
class FPRAS which is considered the efficient computational model for finding approximate solutions
for counting problems.
We mentioned FPRAS algorithms for several problems such as simple \ccsp(\mrelset),
\pname{\#Dual-SAT},\pname{\#Match}, and \pname{\#LowDegree-}k\pname{-Coloring}.
The problem \pname{\#Dual-SAT} is solved by sampling and the problems \pname{\#Match} and 
\pname{\#LowDegree-}k\pname{-Coloring} are solved by Markov chain Monte-Carlo method. 

We defined AP-reductions which are used to classify approximation counting problems.
We introduced the problem \cbis\ and the class of the problems AP-interreducible with
the problem \cbis\@. We mentioned several well-known examples of this class such as
the  problems \cdsp\ and \pname{\#1P1N-SAT}\@. Computation of the partition function for 
the Ising model, where the model is ferromagnetic, also belongs to this class.

We also mentioned that some of the problems are hard to approximate. There are
no polynomial time approximation algorithm for any of the problems in this class, unless \(RP=NP\)\@.
Note that although according to current knowledge these problems are not efficiently solvable,
approximating these problems is still easier than exactly solving them.
Some of the well known problems in this 
class are the problems \cdsat, \ctsat, \csat, \cisp, and \ctcol\@. Computation
of the partition function for 
the Ising model, where the model is anti-ferromagnetic, also belongs to this class.

We showed that by limiting the input for the problems \ccsp(\mrelset) to connected structures,
the problems do not become easier. We showed that if there is linear transformation 
between the number of solutions for two counting problems and 
both problems always have solutions, then those two problems are AP-interreducible.
We generalized the pinning theorem from \cite{Trichotomy}\@. 

We introduced the maximization technique for AP-reductions. The sets of relations closed
under maximization form sets of relations similar to relational clones.

We extended the notion of \emph{monotone} relations from Boolean domains to 
general domains. We proved that for any monotone constraint language \mrelset,
the problem \ccsp(\mrelset) is AP-reducible to the problem \cbis\@.
We also extended the notion of monotone relations to bipartite graphs 
and proved that for any monotone bipartite graph \mH, the problem \chom(\mH)  
is also AP-reducible to to the problem \cbis\@.

We proved that for any reflexive oriented graph \mH, the problem \cbis\ is
AP-reducible to the problem \chom(\mH).
Despite these results, finding a necessary and sufficient condition for \mrelset\ such that 
the problem \ccsp(\mrelset) is AP-interreducible with the problem \cbis\ remains open.

\begin{figure}
\centering
\subfigure[\ensuremath{N_5}]{\input{figs/n5.pdftex}\label{fig:n5}}\hfill
\subfigure[\ensuremath{M_5}]{\input{figs/m5.pdftex}\label{fig:m5}}\hfill
\subfigure[\ensuremath{C_6}]{\input{figs/c6.pdftex}\label{fig:c6}}
\caption{Example graphs for which complexity of the \chom(\ensuremath{H}) is unknown}
\label{fig:unknown}
\end{figure}

We investigated monotone reflexive graphs. We proved that a retract of
a product of proper interval graphs is a monotone graph.

We expect that there exists a trichotomy for the problem \ccsp(\mrelset) over 3-element sets
similar to the trichotomy for the problem \ccsp(\mrelset) over 2-element sets.
However, the common belief is that there is no such a trichotomy in general for
the problem \ccsp(\mrelset).
For example, consider the graphs \(N_5\), \(M_5\), and \(C_6\) shown in Figure~\ref{fig:unknown}.
The problems \chom(\(N_5\)), \chom(\(M_5\)), and \chom(\(C_6\)) are harder than
the problem \cbis; however, they are not expected to be as hard as the problem \csat\@.

\section{Future Work}
We studied the complexity of approximately solving 
the problem \ccsp(\mrelset) for maximal partial clones on
3-element sets \cite{mvl}, we also studied complexity of approximately solving 
the problem \chom(\mH) for graphs with 3 vertices. However, in both categories,
there are several problems whose complexity of approximately solving them is still unknown.

Valiant and Vazirani \cite{valvaz} proved that with a SAT oracle, there is FPRAS for any 
problem in \cp\@. With the recent growth of practical usage of SAT-solvers, this theorem
potentially provides a practical approach to approximately solve the problem \ccsp(\mrelset).

The Markov chain Monte-Carlo method is used approximately solve several counting problems.
This method may approximately solve the problem \ccsp(\mrelset) for some \mrelset\@.

We defined the monotone constraint languages as the constraint languages that
are closed under the meet and join operators of a distributive lattice.
\(N_5\) and \(M_5\), shown in Figure~\ref{fig:unknown}, appear on non-distributive 
lattices. Proving that the problems \chom(\(N_5\)) and \chom(\(M_5\)) are AP-reducible to
the problem \cbis\ may extend our result to
constraint languages closed under meet and join operators of a general lattice.

We proved that for any reflexive oriented graph \mH, the problem \cbis\ is
AP-reducible to the problem \chom(\mH). We can also prove that for some
oriented graph with possible loop, the problem \cbis\ is AP-reducible to 
the problem \chom(\mH). We believe for a more general family of digraphs
with possible loops, the problem \cbis\ is AP-reducible to the problem \chom(\mH).