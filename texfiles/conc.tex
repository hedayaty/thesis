\chapter{Conclusion and Future Works}
\section{Conclusion}
We formally defined the \ccsp(\mrelset) problem, mentioned some of its applications; mentioned the 
results and the dichotomy on complexity of the \ccsp(\mrelset) problem. We defined the
FPRAS class which is considered as the efficient computational model for finding approximate solutions
for counting problems.
We mentioned FPRAS algorithms for some simple \ccsp(\mrelset) problems, also 
for the \pname{\#Dual-SAT},\pname{\#Match}, and \pname{\#LowDegree-}k\pname{-Coloring} problems.
The \pname{\#Dual-SAT} problem is solved by sampling and the \pname{\#Match} and \pname{\#LowDegree-}k\pname{-Coloring} problems are solved by the Markov chain Monte-Carlo technique. 

We defined the AP-reductions which are the reductions used to classify approximation
counting problems.
We introduced the \cbis\ problem and the class of the problems AP-interreducible with the \cbis\ problem.
We mentioned several well-known examples of this class such as
the \cds\ and \pname{\#1P1N-SAT} problems. Computation of the partition function for 
the Ising model, where the model is ferromagnetic also belongs to this class.

We also mentioned that some of the problems are hard to approximate. There is
no polynomial time approximation algorithm for problems in this class, unless \(RP=NP\)\@.
Note that although according to current knowledge these problems are not efficiently solvable,
approximating these problems is still easier than exactly solving them.
Some of the well known problems in this 
class are the \cdsat, \ctsat, \csat, \cis, and \ctcol\ problems. Computation of the partition function for 
the Ising model, where the model is anti-ferromagnetic also belongs to this class.

We showed that by limiting the input for the \ccsp(\mrelset) problems to connected structures,
the problems do not become easier. We showed that if there is linear transformation 
between the number of solutions for two counting problems and 
both problems always have solutions, then those two problems are AP-interreducible.
We extended the pinning 
theorem \cite{Trichotomy}\@. 

We introduced the maximization technique for AP-reductions. The sets of relations closed
under maximization form sets of relations similar to relational clones.
In future, better understanding these sets provides a better intuition on the complexity of 
the \ccsp(\mrelset) problem.

Finally, we focused on AP-reductions involving the \cbis\ problem.
We extended the notion of \emph{monotone} relations from Boolean domains to 
multi-valued domains. We proved that the \ccsp(\mrelset) problems 
over monotone constraint languages
are AP-reducible to the \cbis\ problem. We also introduced a family of bipartite graphs
called \emph{bimonotone} graphs. We proved that the \chom(\mH) problems 
for bimonotone graphs are also AP-reducible to to the \cbis\ problem.

We also introduced the \emph{\RBA} relations and proved that the \cbis\ problem is
AP-reducible to the \ccsp(\mR) problems where \mR\ is an \RBA\ relation.
Despite these results, characterizing of constraint languages for which 
the \ccsp(\mrelset) problems are AP-interreducible with the \cbis\ problem remains open.

\begin{figure}
\centering
\subfigure[\ensuremath{N_5}]{\input{figs/n5.pdftex}\label{fig:n5}}\hfill
\subfigure[\ensuremath{M_5}]{\input{figs/m5.pdftex}\label{fig:m5}}\hfill
\subfigure[\ensuremath{C_6}]{\input{figs/c6.pdftex}\label{fig:c6}}
\caption{Several graphs for which complexity of the \chom(\ensuremath{H}) is unknown}
\label{fig:unknown}
\end{figure}

We expect that there exists a trichotomy similar to the Boolean \ccsp(\mrelset) problems
for the \ccsp(\mrelset) problems over 3-element and 4-element sets.
However, the common belief is that there is no such a trichotomy in general for
the \ccsp(\mrelset) problems.
For example, consider the graphs \(N_5\), \(M_5\), and \(C_6\) shown in Figure~\ref{fig:unknown}.
The \chom(\(N_5\)), \chom(\(M_5\)), and \chom(\(C_6\)) problems are harder than
the \cbis\ problem; however, they are expected to be as hard as the \csat\ problem.

\section{Future Works}
We studied the complexity of approximately solving 
the \ccsp(\mrelset) problem for maximal partial clones on
3-element sets \cite{mvl}, we also studied complexity of approximately solving 
the \chom(\mH) problem for graphs with 3 vertices. However, in both categories,
there are few problems whose complexity of approximately solving them is still unknown.

Valiant and Vazirani \cite{valvaz} proved that with a SAT oracle, there is FPRAS for any 
problem in \cp\@. With the recent growth of practical usage of SAT-solvers, this theorem
has the potential to provide a practical approach to solve the \ccsp(\mrelset) problem.

The Markov chain Monte-Carlo is a useful technique to find approximate solutions for
several counting problems. There might be some families of constraint languages for 
which the Markov chain Monte-Carlo method may be used to approximately solve the 
\ccsp(\mrelset) problem over them.

We defined the monotone constraint languages as the constraint languages that
are closed under the meet and join operators from a distributive lattice.
\(N_5\) and \(M_5\) shown in Figure~\ref{fig:unknown} always appear on non-distributive 
lattices. Proving that the \chom(\(N_5\)) and \chom(\(M_5\)) problem are AP-reducible to
the \cbis\ problem may be a break through by extending  our result to
constraint languages closed under meet and join operators from any lattice.

We introduced \RBA\ graphs and proved that the \cbis\ problem is AP-reducible to
the \chom(\mH) problem where \mH\ is an \RBA\ graph. We partially extended the result
for specific graphs; however, we expect this result to be extendable from the \RBA\ graphs
to general graphs with at least 2 asymmetric reflexive vertices.