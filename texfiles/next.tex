\chapter{Characterization of Monotone Graphs}
In Chapter~\ref{chp:results} we showed that for monotone relations and monotone
bipartite graphs the problem \ccsp(\mH) is AP-reducible to the \cbis\ problem. In this chapter we will
inspect\todo{inspect?} the monotone reflexive graphs; although the arguments are
analogous for bipartite graphs we will not cover them in this work.

\emph{Interval graphs} and \emph{proper interval graphs}
are two interesting families of reflexive graphs that are play a key role in complexity of 
list homomorphism and mincost-homomorphism problems. The are interested in these two families
because of their connection with monotone graphs. An interval graph is defined as follows.

\begin{defi}
A graph \(G=(V,E)\) is said be to an interval graph if every vertex \(v \in V\)
corresponds to an interval \(i(v)\) such that \(uv \in E\) if and only if the intervals
\(i(u)\) and \(i(v)\) intersect.
\end{defi}

\begin{figure} [h]
\hfill
\subfigure[\ensuremath{G_1}]{\input{figs/claw.pdftex}\label{fig:clawint}}\hfill 
\subfigure[\ensuremath{I_1}]{\input{figs/intv.pdftex}\label{fig:int}}\hfill 
\end{figure}

The graph \(G_1\) in Figure~\ref{fig:clawint} referred as \emph{claw} is an interval graph
and \(I_1\) in Figure~\ref{fig:int} is the set of intervals corresponding to \(G_1\)\@.
The list homomorphism problem is defined as:

\begin{defi} [L-Hom]
Let \mH\ be a fixed graph. Given an input graph \mG\ and for each vertex \(v\) of \mG\
a \emph{list} \(L(v) \subseteq V(H)\) the \emph{list homomorphism} problem, denoted
as L-Hom(\mH), is the problem of  deciding whether or not
there exists a homomorphism \(h: G\to H\) such that for each vertex \(v\) of \mG\ 
we have \(h(v)\in L(v)\)\@.
\end{defi}

Theorem~\ref{thm:lhom} by Feder and Hell shows the connection between the list homomorphism
problem and the interval graphs.

\begin{theorem} [Feder and Hell 1996 \cite{listhom}] \label{thm:lhom}
For any fixed reflexive graph \mH, the problem L-Hom(\mH) is polynomial time solvable if
\mH\ is an interval graph and otherwise, it is NP-complete.
\end{theorem}

One of the ways to define a proper interval graph is as follows.

\begin{defi}
A graph \(G=(V,E)\) is said to be a proper interval graph if every vertex \(v \in V\)
corresponds to an interval such that no interval is a proper subset of any other interval
\(uv \in E\) if and only if the intervals \(i(u)\) and \(i(v)\) intersect.
\end{defi}

Alternately the proper interval graphs are defined as the graphs which can be represented 
by intervals of unit length.

\begin{figure}[h]
\hfill
\subfigure[\ensuremath{G_2}]{\input{figs/ps4.pdftex}\label{fig:pintg}}\hfill 
\subfigure[\ensuremath{I_2}]{\input{figs/pintv.pdftex}\label{fig:pint}}\hfill 
\end{figure}

The graph \(G_2\) in Figure~\ref{fig:pintg} is a proper interval graph and
\(I_2\) in Figure~\ref{fig:pint} is the set of intervals corresponding to \(G_2\)\@.
The min-cost homomorphism problem is defined as:

\begin{defi} [MinHom]
Let \mH\ be a fixed graph. Given an input graph \mG\ and a \emph{cost} function
\(c: V(G) \times V(H) \to R^+\) the \emph{min-cost homomorphism} problem,
denoted as MinHom(\mH), is the problem of finding a homomorphism 
\(h:G\to H\) such that \(\sum_{v\in G} c(v, h(v))\) is minimum.
\end{defi}

The cost function \(c\) indicates\todo{indicates?} the cost of assigning each vertex of \mG\ 
to a specific vertex in \mH\@. 

Theorem~\ref{thm:minhom} shows the connection between the min-cost homorphism problem 
and the proper interval graphs.

\begin{theorem} [TODO] \label{thm:minhom}
For any fixed reflexive graph \mH, the problem MinHom(\mH) is polynomial time solvable if
\mH\ is a proper interval graph and otherwise, it is NP-complete.
\end{theorem}

Interval graphs and proper interval graph can be characterized by polymorphisms. 
These characterization shows their relation with monotone relations. 
Before giving the characterization we remind the definition of semilattice.

\begin{defi} [Semilattice]
A \emph{semilattice} \(L=(X,\wedge,\vee)\) is a binary operators \(\wedge\) on \(X\) which is
\begin{itemize}
\item idempotent: \(x \wedge x = x\),
\item commutative: \(x \wedge y = y \wedge x\),
\item associative: \(x \wedge (y \wedge z) = (x \wedge y) \wedge z\).
\end{itemize}
\end{defi}

\begin{theorem}[TODO] \label{thm:semimin}
A reflexive graph \mH\ is an interval graph if and only if there exists
a semilattice \(L=(V(H), \wedge)\) such that \(\wedge\) is a polymorphism for \mH\@.
\end{theorem}

\begin{cor} \label{cor:intmon}
Every reflexive monotone graph is an interval graph.
\end{cor}

\begin{theorem} [TODO] \label{thm:minmax}
A reflexive graph \mH\ is a proper interval graph if there exists an ordering of 
vertices of \mH\ such that \emph{min} and \emph{max} defined by the ordering are
polymorphisms for \mH\@.
\end{theorem}

\begin{cor} \label{cor:pintmon}
Every reflexive proper interval graph is a monotone graph.
\end{cor}

Theorems~\ref{thm:semimin}~and~\ref{thm:minmax} indicate similarities between
monotone graphs and (proper) interval graphs. However, monotone graphs are not
equivalent with (proper) interval graphs.

\begin{example} \ref{exm:diff}
The graph \(H_1\) in Figure~\ref{fig:intnotmon} is 
an interval graph; however, \(H_1\) is not a monotone graph.
The graph \(H_2\) in Figure~\ref{fig:monnotpint} is a monotone graph;
however, the vertices \(a\), \(b\), \(c\), and \(d\) forms a claw; hence,
\(H_2\) is not a proper interval graph. 

\begin{figure}[h]
\hfill
\subfigure[\ensuremath{H_1}]{\input{figs/claw.pdftex}\label{fig:intnotmon}}\hfill 
\subfigure[\ensuremath{H_2}]{\input{figs/lattice.pdftex}\label{fig:monnotpint}}\hfill 
\end{figure}
\end{example}


Note that for (proper) interval interval graphs, if \(H\) is not a (proper) interval graph
and it is an induced subgraph of \(H'\) then \(H'\) is not a (proper) interval graph, as well.
However, in the Example~\ref{exm:diff} \(H_1\) is not a monotone graph and it is an induced 
subgraph of \(H_2\); regardless, \(H_2\) is a monotone graph.

\section{Finding monotone graphs}
In this section we suggest several ways to build monotone reflexive graphs. We are aware that
these constructions do not cover all the monotone graphs and we will 
give examples of the monotone graph we can not build using this method.
By Corollary~\ref{cor:pintmon}, every proper interval graph is a monotone graph.
This gives us some starting point.

\begin{lemma} \label{lem:monotone:con}
A reflexive graph \mG\ is monotone if and only if each component of \mG\ is monotone.
\end{lemma}

\begin{proof}
TODO
\end{proof}

Using Lemma~\ref{lem:monotone:con}, we can build a monotone graph
by building each connected component of it. From here, we restrict ourselves to
reflexive and connected monotone graph. Assume all the monotone graphs are
connected and reflexive, unless explicitly specified.

When a lattice \(L=(X,\wedge,\vee)\) is viewed as a partial order \((X,\preceq)\), a \emph{cover}
is a pair \((a,b)\) such that
\[a \preceq x \preceq b \Rightarrow a = x or b = x\]

\begin{lemma}
Let \mG\ be a monotone graph with the lattice ordering \(L\), then every cover
if \(L\) is an edge of \mG\@.
\end{lemma}

\begin{proof}
TODO
\end{proof}

TODO Define \(\times\) on graphs and lattices\@.
\begin{lemma}
For any monotone graphs \(G_1\) and \(G_2\), \(G_1 \times G_2\) is monotone.
\end{lemma}

\begin{proof}
TODO
\end{proof}

TODO define retraction on graphs and lattices
\begin{lemma}
If \(r\) a retraction from \(L\) to \(L'\) then
TODO
\end{lemma}

\begin{proof}
TODO
\end{proof}