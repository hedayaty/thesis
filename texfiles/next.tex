\chapter{Characterization of Monotone Graphs}
In Chapter~\ref{chp:results}, we showed that for monotone relations and monotone
bipartite graphs the problem \ccsp(\mH) is AP-reducible to the \cbis\ problem. In this chapter,
we will inspect the monotone reflexive graphs; although the arguments are
analogous for bipartite graphs we will not cover them in this work.

\emph{Interval graphs} and \emph{proper interval graphs}
are two interesting families of reflexive graphs that play a key role in complexity of 
list homomorphism and mincost-homomorphism problems. We are interested in these two families
because of their connections with monotone graphs. An interval graph is defined as follows:

\begin{defi}
A graph \(G=(V,E)\) is said be to an interval graph if every vertex \(v \in V\)
corresponds to an interval \(i(v)\) such that \(uv \in E\) if and only if the intervals
\(i(u)\) and \(i(v)\) intersect.
\end{defi}

\begin{figure} [h]
\hfill
\subfigure[\ensuremath{G_1}]{\input{figs/claw.pdftex}\label{fig:clawint}}\hfill 
\subfigure[\ensuremath{I_1}]{\input{figs/intv.pdftex}\label{fig:int}}\hfill 
\end{figure}

The graph \(G_1\) in Figure~\ref{fig:clawint} referred as \emph{claw} is an interval graph
and \(I_1\) in Figure~\ref{fig:int} is the set of intervals corresponding to \(G_1\)\@.
The list homomorphism problem is defined as:

\begin{defi} [L-Hom]
Let \mH\ be a fixed graph. Given an input graph \mG\ and for each vertex \(v\) of \mG,
a \emph{list} \(L(v) \subseteq V(H)\) the \emph{list homomorphism} problem, denoted
as L-Hom(\mH), is the problem of  deciding whether or not
there exists a homomorphism \(h: G\to H\) such that for each vertex \(v\) of \mG\ 
we have \(h(v)\in L(v)\)\@.
\end{defi}

Theorem~\ref{thm:lhom} by Feder and Hell shows the connection between the list homomorphism
problem and the interval graphs.

\begin{theorem} [Feder and Hell 1996 \cite{listhom}] \label{thm:lhom}
For any fixed reflexive graph \mH, the problem L-Hom(\mH) is polynomial time solvable if
\mH\ is an interval graph and otherwise, it is NP-complete.
\end{theorem}

One way to define a proper interval graph is as follows:

\begin{defi}
A graph \(G=(V,E)\) is said to be a proper interval graph if every vertex \(v \in V\)
corresponds to an interval such that no interval is a proper subset of any other interval
and \(uv \in E\) if and only if the intervals \(i(u)\) and \(i(v)\) intersect.
\end{defi}

Alternately the proper interval graphs are defined as the graphs which can be represented 
by intervals of unit length.

\begin{figure}[h]
\hfill
\subfigure[\ensuremath{G_2}]{\input{figs/ps4.pdftex}\label{fig:pintg}}\hfill 
\subfigure[\ensuremath{I_2}]{\input{figs/pintv.pdftex}\label{fig:pint}}\hfill 
\end{figure}

The graph \(G_2\) in Figure~\ref{fig:pintg} is a proper interval graph and
\(I_2\) in Figure~\ref{fig:pint} is the set of intervals corresponding to \(G_2\)\@.
The min-cost homomorphism problem is defined as:

\begin{defi} [MinHom]
Let \mH\ be a fixed graph. Given an input graph \mG\ and a \emph{cost} function
\(c: V(G) \times V(H) \to R^+\) the \emph{min-cost homomorphism} problem,
denoted as MinHom(\mH), is the problem of finding a homomorphism 
\(h:G\to H\) such that \(\sum_{v\in G} c(v, h(v))\) is minimized.
\end{defi}

The cost function \(c\) indicates\todo{indicates?} the cost of assigning each vertex of \mG\ 
to a specific vertex in \mH\@. 

Theorem~\ref{thm:minhom} shows the connection between the min-cost homorphism problem 
and the proper interval graphs.

\begin{theorem} [TODO] \label{thm:minhom}
For any fixed reflexive graph \mH, the problem MinHom(\mH) is polynomial time solvable if
\mH\ is a proper interval graph and otherwise, it is NP-complete.
\end{theorem}

We show that (proper) interval graphs can be characterized by polymorphisms. 
These characterizations show the relation between (proper) interval graphs and monotone relations. 
Before giving the characterization we remind the definition of semilattice.

\begin{defi} [Semilattice]
A \emph{semilattice} \(L=(X,\wedge,\vee)\) is a binary operators \(\wedge\) on \(X\) which is
\begin{itemize}
\item idempotent: \(x \wedge x = x\),
\item commutative: \(x \wedge y = y \wedge x\),
\item associative: \(x \wedge (y \wedge z) = (x \wedge y) \wedge z\).
\end{itemize}
\end{defi}

\begin{theorem}[TODO] \label{thm:semimin}
A reflexive graph \mH\ is an interval graph if and only if there exists
a semilattice \(L=(V(H), \wedge)\) such that \(\wedge\) is a polymorphism for \mH\@.
\end{theorem}

\begin{cor} \label{cor:intmon}
Every reflexive monotone graph is an interval graph.
\end{cor}

\begin{theorem} [TODO] \label{thm:minmax}
A reflexive graph \mH\ is a proper interval graph if there exists an ordering of 
vertices of \mH\ such that \emph{min} and \emph{max} defined by the ordering are
polymorphisms for \mH\@.
\end{theorem}

\begin{cor} \label{cor:pintmon}
Every reflexive proper interval graph is a monotone graph.
\end{cor}

Theorems~\ref{thm:semimin}~and~\ref{thm:minmax} indicate similarities between
monotone graphs and (proper) interval graphs. However, monotone graphs are not
equivalent with (proper) interval graphs.

\begin{example} \label{exm:diff}
The graph \(H_1\) in Figure~\ref{fig:intnotmon} is 
an interval graph; however, \(H_1\) is not a monotone graph.
The graph \(H_2\) in Figure~\ref{fig:monnotpint} is a monotone graph;
however, the vertices \(a\), \(b\), \(c\), and \(d\) forms a claw; hence,
\(H_2\) is not a proper interval graph. 

\begin{figure}[h]
\hfill
\subfigure[\ensuremath{H_1}]{\input{figs/claw.pdftex}\label{fig:intnotmon}}\hfill 
\subfigure[\ensuremath{H_2}]{\input{figs/lattice.pdftex}\label{fig:monnotpint}}\hfill 
\end{figure}
\end{example}


For any (proper) interval graph \mH, each induced sub-graph of \mH\ is a 
(proper) interval graph. However, in the Example~\ref{exm:diff} \(H_1\)
is not a monotone graph and it is an induced 
sub-graph of \(H_2\); while, \(H_2\) is a monotone graph.

\section{Finding monotone graphs}
In this section we suggest several ways to generate monotone reflexive graphs.
There are monotone graphs that can not be generated by the following methods,
we will give several examples of such graphs.
By Corollary~\ref{cor:pintmon}, every proper interval graph is a monotone graph.
Thus, proper interval graphs can be used as a base to generate monotone graphs.

\begin{lemma} \label{lem:monotone:con}
A reflexive graph \mG\ is monotone if and only if each component of \mG\ is monotone.
\end{lemma}

\begin{proof}
If \mG\ is a monotone graph with lattice order \(L\), each connected component of \mG\ is
closed under same meet and join operators defined by \(L\)\@. 

If \(G_1,G_2,\dotsc,G_k\), the connected components of \mG, are
are monotone graphs with lattice orders \(L_1,L_2,\dotsc,L_k\),
\mG\ is closed under any lattice \(L\) based on union of \(L_i\)s
and choosing any arbitrary total (or even lattice order) between the
lattices. The meet and join of any two edge from the same component is in
\mG\ by hypothesis and meet and join of any two edges from different components 
will be a loop which is in \mg\ because \mG\ is reflexive. 
\end{proof}

Lemma~\ref{lem:monotone:con} indicates that any monotone graph can be generated
from its connected components. From here, we restrict ourselves to
reflexive and connected monotone graph. All the monotone graphs are
assumed to be connected and reflexive, unless explicitly specified.

A lattice \(L=(X,\wedge,\vee)\) can be viewed as a partial order;
\(\bot\) and \(\top\) denote the smallest and largest elements in the \(L\) 
respectively; \((X,\preceq)\), a \emph{cover} is a pair \((a,b)\) such that
\[a \preceq x \preceq b \Rightarrow a = x\ \mathrm{or}\ b = x\]

\begin{lemma}
Let \mG\ be a monotone graph with lattice ordering \(L\),
for every vertex \(v\), there is descending path from
\(v\) to \(\bot\)\@.
\end{lemma}

\begin{proof}
Since \mG\ is connected there a path \(u_0=u,u_1,u_2,\dotsc,u_k=\bot\) from \(u\) to 
\(\bot\) in \mG\@. Let \(u'_0=u_0\) and \(u'_i = u_i\wedge u'_{i-1}\)\@.
Since \mG\ is reflexive \(u'_0u'_0=uu \in G\)\@.
\[u'_0u'_0, u_0u_1 \in G \Rightarrow u'_0u'_1 \in G\]
inductively we have
\[u'_{i-1}u'_i, u_iU_{i+1} \in G \Rightarrow u'_iu'_{i+1} \in G\]
This implies that the path \(u'_0=u,u'_1,\dotsc,u'_k=\bot\) is a descending path from 
\(u\) to \(\bot\)\@.
\end{proof}

\begin{cor}
For every element \(u \neq \bot\),
there is an element \(x\) such that \(ux \in G\) and \(x \preceq u\)\@.
\end{cor}

\begin{lemma}
Let \mG\ be a monotone graph with the lattice ordering \(L\), then every cover
if \(L\) is an edge of \mG\@.
\end{lemma}

\begin{proof}
Let \(a,b\) be a cover in \(L\) such that \(a \preceq b\)\@.
Since \(a \preceq b\), we have \(b \neq \bot\); thus,
there is \(x\) such that \(bx \in G\) and \(x\preceq b\)\@.
If \(x=a\), the proof is complete; otherwise, since \((a,b)\) is a cover, we have \(x\preceq a\).
\[bx,aa \in G \Rightarrow ba \in G\]\@
\end{proof}

Next, we want to show that monotone graphs are closed under Cartesian products.
Remember that the Cartesian product on graphs is defined as follows:
\begin{defi} [Cartesian product on graphs]
For graphs \(G_1=(V_1,E_1\) and \(G_2=(V_2,E_2)\), \(G = G_1 \times G_2\)
is the \emph{Cartesian product} of \(G_1\) and \(G_2\) if \(G=(V,E)\)
such that \(V=V_1 \times V_2\) and \((u_1,u_2)(v_1,v_2) \in E\)
if and only if \(u_1v_1 \in E_1\) and \(u_2v_2 \in E_2\)\@.
\end{defi}

We need the Cartesian product of the lattices as well,
since lattices are partial orders we will simply define the Cartesian product
for partial orders.
\begin{defi} [Cartesian product on partial orders]
For partial orders \(P_1(X_1,\preceq)\) and \(P_2=(X_2,\preceq)\), 
\(P= P_1 \times P_2\) is the \emph{Cartesian product} of \(P_1\) and \(P_2\)
if \(P=(X,\preceq\) such that \(X=X_1\times X_2\) and \((x_1,x_2) \preceq (y_1,y_2)\)
if and only if \(x_1\preceq y_1\) and \(x_2\preceq y_2\)\@.
\end{defi}

\begin{rem}
Let \(L_1=(X_1,\wedge,\vee)\) and \(L_2=(X_2,\wedge,\vee)\) be two lattices.
For \(x_1,y_1 \in X_1\) and \(x_2,y_2\in X_2\), we have :
\[(x_1,x_2)\wedge (y_1,y_2) = (x_1\wedge y_1, x_2 \wedge y_2) \]
and \[(x_1,x_2)\vee (y_1,y_2) = (x_1\vee y_1, x_2 \vee y_2).\]
\end{rem}

Lemma~\ref{lem:pro} shows that we can generate monotone relations from product of monotone relations.

\begin{lemma}\ref{lem:prod}
For any monotone graphs \(G_1\) and \(G_2\), \(G_1 \times G_2\) is also a monotone graph.
\end{lemma}

\begin{proof}
For any two edges \(e = (u_1,u_2)(v_1,v_2)\) and \(e' =(u'_1,u'_2)(v'_1,v'_2)\) of \(G_1\times G_2\),
the edges \(u_1v_1\) and \(u'_1v'_1\) are edges of \(G_1\) and
edges \(u_2v_2\) and \(u'_2v'_2\) are edges of \(G_2\)\@.
Since \(G_1\) and \(G_2\) are monotone,
\[e_1 = u_1v_1 \wedge u'_1v'_1\]
and
\[e_2 = u_2v_2 \wedge u'_2v'_2\]
are edges of \(G_1\) and \(G_2\), respectively.
\begin{align*}
e \wedge e' &=& ((u_1,u_2)\wedge(u'_1,u'_2))((v_1,v_2)\wedge(v'_1,v'_2)) \\
&=& (u_1\wedge u'_1,u_2 \wedge u'_2)(v_1\wedge v'_1,v_2 \wedge v'_2) \\
&=& (u_1\wedge u'_1)(v_1\wedge v'_1) \wedge (u_2\wedge u'_2)(v_2\wedge v'_2) \\
&=& e_1 \wedge e_2 \\
&\in& G_1 \times G_2 \\
\end{align*}
Analogously, \(e \vee e'\) is also an edge of \(G_1\times G_2\)\@.
\end{proof}

An other operation that preserves monotone graphs is retraction. Here is a reminder 
for retraction.
\begin{defi} [Retraction]
A homomorphism from \mG\ to a subset of \mG\ is called a retraction.
\end{defi}

\begin{lemma}
Let \mG\ be a monotone graph with lattice order \(L\)\@. For a retraction \(r\) of \(L\)
we have:
\begin{itemize}
\item \(r\) is a retraction of \mG
\item \(r(G)\) is a monotone graph
\end{itemize}
\end{lemma}

\begin{proof}
TODO
\end{proof}

\begin{lemma}
Let \(G=(V,E)\) be a monotone graph with lattice order \(L=(V,\wedge, \vee)\)\@.
For vertices \(a\), \(b\), \(c\), and \(d\) of \(V\) if
\(b \vee c  = a\) and \(b \wedge c = d\), then 
\(ad \in E\)\@.
\end{lemma}

\begin{proof}
\end{proof}

TODO give arguments of that this construction do not generate all monotone graphs.