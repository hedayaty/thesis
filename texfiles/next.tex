\chapter{Characterization of Monotone Graphs}
In Chapter~\ref{chp:results}, we showed that for monotone relations and monotone
bipartite graphs the problem \ccsp(\mH) is AP-reducible to the problem \cbis\@. In this chapter,
we will inspect monotone reflexive graphs; although the arguments are
analogous for bipartite graphs we will not cover bipartite graphs in this part.
\(u \sim v\) indicates there is an edge between vertices \(u\) and \(v\);
\(u \not sim v\) indicates there is not an edge between vertices \(u\) and \(v\)\@.

\emph{Interval graphs} and \emph{proper interval graphs}
are two interesting families of reflexive graphs that play a key role in complexity of 
list homomorphism and mincost-homomorphism problems. We are interested in these two families
because of their connection with monotone graphs. 
An interval graph is defined as follows:

\begin{defi}
A graph \(G=(V,E)\) is said be to an \emph{interval graph} if \mG\ can
be represented by a family of closed intervals such that
each vertex \(v \in V\) corresponds to an interval \(I_v\) and \(u \sim v\)
if and only if the intervals \(I_u\) and \(I_v\) intersect.
\end{defi}

\begin{figure} [h]
\center
\subfigure[\ensuremath{G_1}]{\input{figs/claw.pdftex}\label{fig:claw}} \hspace{2cm}
\subfigure[\ensuremath{I_1}]{\input{figs/intv.pdftex}\label{fig:int}} \hspace{2cm}
\subfigure[\ensuremath{G_2}]{\input{figs/cs4.pdftex}\label{fig:nonint}}
\caption{Interval graph \ensuremath{G_1}, intervals \ensuremath{\mathcal{I}_1}, and non-interval graph \ensuremath{G_2}}
\end{figure}

The graph \(G_1\) in Figure~\ref{fig:claw}, called \emph{claw}, is an interval graph
and \(\mathcal{I}_1\) in Figure~\ref{fig:int} is a possible family of intervals representing \(G_1\)\@.
The graph \(G_2\) in Figure~\ref{fig:nonint} is not an interval graph. In order to show that \(G_2\) is not
an interval graph, suppose \(\{I_a, I_b, I_c, I_d\}\) is a family of intervals representing \(G_2\)\@.
Without loss of generality, suppose \(I_a\) is the interval with the smallest starting point.
Since \(a\not \sim d\), \(I_d\) must appear after \(I_a\)\@. Since \(b\sim a\), \(b\sim d\), \(c\sim a\), and \(c\sim d\),
\(I_b\) and \(I_c\) must both intersect \(I_a\) and \(I_d\); for that, they must both include the interval between
\(I_a\) and \(I_d\); this implies that \(b\sim c\) which is a contradiction.

One of the problems that relate homomorphisms to interval graphs is the list homomorphism problem.
The list homomorphism problem is defined as:

\begin{defi} [L-Hom]
Let \mH\ be a fixed graph. Given an input graph \mG\ and for each vertex \(v\) of \mG,
a \emph{list} \(L(v) \subseteq V(H)\) the \emph{list homomorphism} problem for \mH, denoted
as L-Hom(\mH), is the problem of  deciding whether or not
there exists a homomorphism \(h: G\to H\) such that for each vertex \(v\) of \mG\ 
we have \(h(v)\in L(v)\)\@.
\end{defi}

Theorem~\ref{thm:lhom} shows a connection between the list homomorphism
problem and interval graphs.

\begin{theorem} [Feder and Hell 1996 \cite{listhom}] \label{thm:lhom}
For any fixed reflexive graph \mH, the problem L-Hom(\mH) is polynomial time solvable if
\mH\ is an interval graph and otherwise, it is NP-complete.
\end{theorem}

In addition to interval graphs, we shall also focus on proper interval graphs.
Proper interval graphs are defined as follows:

\begin{defi}
A graph \(G=(V,E)\) is said be to a \emph{proper interval graph} if \mG\ can
be represented by a family of closed intervals such that 
each vertex \(v \in V\) corresponds to an interval \(I_v\), 
\(I_u \not\subseteq I_v\) for any two vertices that \(u \neq v\), and
\(u \sim v\) if and only if the intervals \(I_u\) and \(I_v\) intersect.
\end{defi}

\begin{figure}[h]
\center
\subfigure[\ensuremath{G_3}]{\input{figs/ps4l.pdftex}\label{fig:pintg}}\hspace{4cm}
\subfigure[\ensuremath{I_2}]{\input{figs/pintv.pdftex}\label{fig:pint}}%\hspace{2cm}
\caption{Proper interval graph \ensuremath{G_2} and intervals \ensuremath{\mathcal{I}_2}}\label{fig:pintgi}
\end{figure}

The graph \(G_3\) in Figure~\ref{fig:pintg} is a proper interval graph and
\(\mathcal{I}_2\) in Figure~\ref{fig:pint} is a possible set of intervals representing \(G_3\)\@.
The graph \(G_1\) in Figure~\ref{fig:claw} is not a proper interval graph. In order to show that
\(G_1\) is not a proper interval graph, suppose \(\{I_a, I_b, I_c, I_d\}\) is a family of intervals representing \(G_1\)\@.
Intervals \(I_b\), \(I_c\), and \(I_d\) must intersect \(I_a\) and not intersect each other; two of them can intersect \(I_a\)
in the beginning and the end of \(I_a\), the third interval can not intersect \(I_a\) without intersecting the other two; hence, we have a contradiction.

One of the problems that relate homomorphisms to proper interval graphs is the mincost homomorphism problem.
The min-cost homomorphism problem is an optimization problem defined as follows:

\begin{defi} [MinHom]
Let \mH\ be a fixed graph. Given an input graph \mG\ and a \emph{cost} function
\(c: V(G) \times V(H) \to \mathbb{R}^+\) the \emph{min-cost homomorphism} problem,
denoted as MinHom(\mH), is the problem of finding a homomorphism 
\(h:G\to H\) such that \(\sum_{v\in G} c(v, h(v))\) is minimized.
\end{defi}

The value \(c(u,v)\) represents the cost of assigning the vertex \(u\) of \mG\ to the vertex \(v\) of \mH\@.
Theorem~\ref{thm:minhom} shows a connection between the min-cost homomorphism problem 
and proper interval graphs.

\begin{theorem} [Gutin et al. 2008 \cite{Gutin}] \label{thm:minhom}
For any fixed reflexive graph \mH, the problem MinHom(\mH) is polynomial time solvable if
\mH\ is a proper interval graph and otherwise, it is NP-complete.
\end{theorem}

We repeat the definition of \emph{monotone} relations from Chapter~\ref{chp:results}
for reflexive graphs. Before monotone graphs, we repeat the definition of distributive lattice from Chapter~\ref{chp:results}\@.

\begin{defi} [Distributive Lattice]
A \emph{distributive lattice} \(L=(X,\wedge,\vee)\) is set \(X\) with a pair of binary operators \(\wedge\)
and \(\vee\) on \(X\) that are:
\begin{itemize}
\item idempotent: \(x \wedge x = x\) and \(x \vee x = x\),
\item commutative: \(x \wedge y = y \wedge x\) and \(x \vee y = y \vee x\),
\item associative: \(x \wedge (y \wedge z) = (x \wedge y) \wedge z\) and 
\(x \vee (y \vee z) = (x \vee y) \vee z\),
\item absorptive: \(x \wedge (x\vee y) = x\) and \(x \vee (x \wedge y) = x\),
\item distributive: \(x \wedge (y \vee z) = (x \wedge y) \vee (x \wedge z)\) and
\(x \vee (y \wedge z) = (x \vee y) \wedge (x \vee z)\)
\end{itemize}
\end{defi}

A lattice \(L=(X,\wedge,\vee)\) can be viewed as a partial order \(P=(X,\preceq)\);
\[ a\wedge b \preceq a,b, \preceq a\vee b\]
\(\bot\) and \(\top\) denote the smallest and largest elements in \(L\), respectively; in other words:
\[x \preceq \bot \Rightarrow x = \bot \]
\[\top \preceq x \Rightarrow x = \top \]
alternately,
\[a \wedge \bot = \bot, a \vee \bot = a\]
\[a \vee \top = a, a \vee \top = \top\]
a \emph{cover} is a pair \((a,b)\) such that \(a \neq b\) and:
\[a \preceq x \preceq b \Rightarrow a = x\ \mathrm{or}\ b = x\]
A \emph{cover graph} of a partial order is a directed graph such that there is
an edge from \(a\) to \(b\) if \((a,b)\) is a cover in the partial order.


\begin{defi} [Monotone reflexive graph]
Let \(G=(V,E)\) be a reflexive graph; \mG\ is \emph{monotone} if there exists distributive lattice
\(L=(V, \wedge, \vee)\) such that
when \(a\sim b\) and \(c \sim d\), we have \((a \wedge c) \sim (b \wedge d)\) and \((a \vee c) \sim (b \vee d)\)\@.
\end{defi}

We also provide an alternate characterization for (proper) interval graphs.
These characterizations show the relation between (proper) interval graphs and monotone relations. 
A total order is a partial order \(\preceq\) such that for every \(a, b\) either \(a \preceq b\)
or \(b \preceq a\) holds. For every total order, we can define \(\wedge\) and \(\vee\) operators as:
\[a \preceq b \Rightarrow a \wedge b = a, a \vee b = b\]
A function \(f\) that \(f(x_1,x_2,\dotsc,x_k)\in \{x_1,x_2,\dotsc,x_k\}\) is called
a \emph{conservative} function. Hence, conservative \(\wedge\)` and \(\vee\) operators
represent a total order.

\begin{theorem} [\cite{hellbook}] \label{thm:semimin}
A reflexive graph \(G=(V,E)\) is an interval graph if and only if there exists
a total order \(T=(V,\wedge,\vee)\) such that
when \(a\sim b\) and \(c \sim d\), we have \((a \wedge c) \sim (b \wedge d)\)\@.
\end{theorem}

\begin{proof}
Let \(G=(V,E)\) be a graph and let \(T=(V,\wedge,\vee)\) be a total order such that
when \(a\sim b\) and \(c \sim d\), we have \((a \wedge c) \sim (b \wedge d)\)\@.
We show that \mG\ is an interval graph. For \(x\in V\), let \(o(x) = |\{y| y\preceq x\}|\)\@.

Take \(\mathcal{I}\) as follows:
\[I_v = [o(v), \max_{y\sim x}\{o(y)\}]\]
Let \(a, b\in V\) such that \(a \preceq b\)\@. If \(a\sim b\), we have:
\[o(a) \le o(b) \Rightarrow o(b) \in [o(a), o(b)] \subseteq I_a \Rightarrow I_a \cap I_b \neq \emptyset\]
and if \(I_a \cap I_b \neq \emptyset\), we have:
\[o(a) \le o(b) \Rightarrow \exists x: a \sim x, o(b) \le o(x) \Rightarrow b\sim b, a\sim x \Rightarrow (a \wedge b) \sim (b \sim x) \Rightarrow a \sim b.\]
Hence, \(\mathcal{I}\) represent \mG\@.

Now, let \(G=(V,E)\) be a graph and let \(\mathcal{I}\) be a family of intervals such that
each vertex \(v \in V\) corresponds to an interval \(I_v\) and \(u \sim v\)
if and only if the intervals \(I_u\) and \(I_v\) intersect. We show that there is a
total order \(T=(V,\wedge,\vee)\) such that
when \(a\sim b\) and \(c \sim d\), we have \((a \wedge c) \sim (b \wedge d)\)\@.

Take \(a\preceq b\) if \(I_a\) starts \emph{before} \(I_b\) or if \(I_a\) and \(I_b\)
start together and \(I_a\) ends \emph{after} \(I_b\)\@. 
Let \(a,b,c,d \in V\) such that \(a \sim b\), \(c \sim d\) holds.
Without loss of generality assume \(a \preceq b\), \(c \preceq d\), and \(a \preceq c\)\@. 
If \(b \preceq d\), we have:
\[a \wedge c = a, b \wedge d = b, a \sim b \Rightarrow  (a \wedge c) \sim (b \wedge d).\]
If \(d \preceq b\), then 
\[o(a) \le o(c) \le o(d), o(d) \le o(b), I_a \cap I_b \neq \emptyset \Rightarrow a\sim d;\]
and 
\[a \wedge c = a, b \wedge d = d, a\sim d \Rightarrow (a \wedge c) \sim (b \wedge d).\]

\end{proof}

\begin{theorem} [Hell and Rafiey 2010 Unpublished] \label{thm:minmax}
A reflexive graph \(G=(V,E)\) is a proper interval graph if and only if there exists
a total order \(T=(V,\wedge,\vee)\) such that
when \(a\sim b\) and \(c \sim d\), we have \((a \wedge c) \sim (b \wedge d)\) and \((a \vee c) \sim (b \vee d)\)\@.
\end{theorem}

\begin{cor} \label{cor:pintmon}
Every reflexive proper interval graph is a monotone graph.
\end{cor}

Theorems~\ref{thm:semimin}~and~\ref{thm:minmax} indicate similarities between
monotone graphs and (proper) interval graphs. However, monotone graphs are not
equivalent to (proper) interval graphs.

\begin{example} \label{exm:diff}
The graph \(G_1\) in Figure~\ref{fig:claw} is an interval graph; however, we show that \(G_1\) is not a monotone graph.
Suppose that \(G_1\) is monotone. If \(\top = a\), then
\[a \sim b, c \sim a \Rightarrow (a \wedge c) \sim (b \wedge a) \Rightarrow c \sim b\]
Analogously, if \(\bot = a\), then \(b \sim c\)\@. Hence, \(a\) can not be \(\top\) to \(\bot\)\@.
Without loss of generality suppose \(\top = b\) and \(\bot = c\); if \(a \preceq d\)
\[a \sim b, d \sim a \Rightarrow (a \vee d) \sim (b \vee a) \Rightarrow d \sim b\]
Analogously, if \(d \preceq a\), the \(d \sim c\)\@. The last possible case is 
\(a \wedge d = c\) and \(a \vee d = b\). In this case:
\[a \sim d, d \sim d \Rightarrow (a \wedge d) \sim (d \wedge d) \Rightarrow c \sim d\]
Hence, \(G_1\) is not monotone. 

Now consider the graph \(H_1\) in Figure~\ref{fig:monnotpint};
\(H_1\) is not a proper interval graph since the vertices \(a\), \(b\), \(c\), and \(d\) form a claw;
\(H_1\) is not an interval graph either since the vertices \(e\), \(f\), \(g\), and \(h\) induce a subgraph isomorphic to the graph \(G_2\);
however, we show that \(H_1\) is a monotone graph in Example~\ref{exm:prod}.

\begin{figure}[h]
\center
\input{figs/lattice.pdftex}
\caption{Graph \ensuremath{H_1}} \label{fig:monnotpint}
\end{figure}
\end{example}

For any (proper) interval graph \mH, each induced subgraph of \mH\ is a 
(proper) interval graph. However, some monotone graph have induced subgraphs that are not monotone.
In Example~\ref{exm:diff}, we show that \(H_1\) is a monotone graph while \(G_1\) and \(G_2\), induced subgraphs of \(H_1\), are
not monotone.

\section{Generating monotone graphs}
In this section, we provide several ways to generate monotone reflexive graphs.
We attempt to provide an algorithm to generate all reflexive monotone graphs;
however, there exist monotone graphs that can not be generated by the our algorithm,
we provide an example of such graphs. This section is based on an unpublished
manuscript by P. Hell and M. Siggers (2009)\@. In this work, I have modified the
statements of the Lemme and provided alternating proofs to be more consistent with
rest of the thesis. The idea for the proofs are mostly the same.

By Corollary~\ref{cor:pintmon}, every proper interval graph is a monotone graph.
Hence, proper interval graphs can be used as a base to generate monotone graphs.
We first show that in order to find out if a graph is monotone, we can 
investigate individual connected components.

\begin{lemma} \label{lem:monotone:con}
A reflexive graph \mG\ is monotone if and only if each connected component of \mG\ is monotone.
\end{lemma}

\begin{proof}
Let \mG\  be a graph with connected components \(G_1,G_2,\dotsc,G_k\);
if each \(G_i\) is a monotone graph with lattice \(L_i\),
then \mG\ is monotone with any lattice \(L\) based on union of \(L_i\)s
and choosing any arbitrary total ordering between the \(L_i\)s\@.
For any two edges \(e_1\) and \(e_2\) from the same component,
by hypothesis \(e_1\wedge e_2\) and \(e_1\vee e_2\) are
edges of the same component.
For any two edge \(e_1\) and \(e_2\) from different components,
\(e_1\wedge e_2\) and \(e_1\vee e_2\) are both loops and since \mG\ is reflexive,
they are edges of \mG\@.

Let \(G=(V,E)\) be a monotone graph with lattice \(L=(V,\wedge,\vee)\) and connected components \(G_1,G_2,\dotsc,G_k\);
we show that each \(G_i\) is a monotone graph. Before that, we show that each \(G_i\) is closed under \(L\)\@.
In other words, we have: \[x,y\in G_i \Rightarrow x\wedge y, x\vee y \in G_i.\]

Let \(x, y\in G_i\);
since \(x,y\) are in the same connected component of \mG, there is a path \(P=v_0=x,v_1,v_2,\dotsc,v_k=y\)
from \(x\) to \(y\) in \mG\@. Let \(v'_0 = v_0\) and \(v'_j = v'_{j-1} \wedge v_j\)\@.
\[v'_0 \sim v'_0, v_0\sim v_1 \Rightarrow (v'_0 \wedge v_0) \sim (v'_0 \wedge v_1) \Rightarrow v'_0 \sim v'_1;\]
by induction we have:
\[v'_j \sim v'_{j-1}, v_{j+1} \sim v_j \Rightarrow (v'_j \wedge v_{j+1}) \sim (v'_{j-1} \wedge v_j) \Rightarrow v'_{j+1} \sim v'_j.\]
This shows that there is a vertex \(z'=v'_{k+1}\) in \(G_i\) such that \(x,y\preceq z'\)\@.
Let \(z = a\wedge b\); if \(z' \prec z\), then there is some \(j\) such that \(v'_j \prec z \prec v'_{j+1}\) and 
\[v'_j\sim v'_{j+1}, z \sim z \Rightarrow (v'_j \wedge z) \sim (v'_{j+1} \wedge z) \Rightarrow z \sim v'_{j+1} \Rightarrow z \in G_i.\]
Hence, \(G_i\) is closed under \(\wedge\) operator; similarly, \(G_i\) is also closed under \(\vee\) operator.
For any edge \(e_1,e_2\in G_i\), we have \(e_1\wedge e_2\in G_i\) and \(e_1\vee e_2 \in G_i\); hence, \(G_i\) is a monotone graph.
\end{proof}

Lemma~\ref{lem:monotone:con} indicates that any monotone graph can be generated
from its connected components. From here, all the monotone graphs are
assumed to be connected and reflexive, unless explicitly specified.

\begin{lemma}
Let \mG\ be a monotone graph with lattice  \(L\),
for every vertex \(v\), there is descending path from
\(v\) to \(\bot\)\@.
\end{lemma}

\begin{proof}
Since \mG\ is connected, there a path \(u_0=u,u_1,u_2,\dotsc,u_k=\bot\) from \(u\) to 
\(\bot\) in \mG\@. Let \(u'_0=u_0\) and \(u'_i = u_i\wedge u'_{i-1}\)\@.
Since \mG\ is reflexive \(u\sim u \Rightarrow u'_0\sim u'_0\)\@.
\[u'_0\sim u'_0, u_0\sim u_1 \Rightarrow u'_0\sim u'_1\]
by induction we have:
\[u'_{i-1}\sim u'_i, u_i\sim u_{i+1} \Rightarrow u'_i\sim u'_{i+1} \]
This implies that \(W=u'_0=u,u'_1,\dotsc,u'_k=\bot\) is a descending walk from 
\(u\) to \(\bot\)\@. \(W\) may contain loops (not cycles because \(W\) is descending);
by removing the loops, we get a descending path from \(u\) to \(\bot\)\@.
\end{proof}

\begin{cor}
Let \(G=(V,E)\) be a monotone graph with lattice \(L=(V,\preceq)\); 
For every element \(u \neq \bot\),
there is an element \(x \preceq u\) such that \(u\sim x\)\@.
\end{cor}

\begin{lemma}
Let \mG\ be a monotone graph with the lattice \(L\), then every cover
of \(L\) is an edge of \mG\@.
\end{lemma}

\begin{proof}
Let \(a,b\) be a cover in \(L\) such that \(a \preceq b\)\@.
\[a \neq b, a \preceq b \Rightarrow b\neq \bot \Rightarrow \exists x \preceq b: b\sim x\]
If \(x=a\), the proof is complete; otherwise, since \((a,b)\) is a cover, we have \(x\preceq a\).
\[b\sim x,a\sim a \Rightarrow (b \vee a) \sim (x \vee a) \Rightarrow b\sim a\]\@
\end{proof}

\begin{cor}
Let \(G=(V,E)\) be a monotone graph with lattice \(L=(V, \preceq)\);
for every pair of vertices that \(u\preceq v\), there is descending path from
\(v\) to \(u\) in \mG\@. 
\end{cor}

\begin{lemma}
Let \mG\ be a monotone graph and let
\(a\), \(b\), \(c\), and \(d\) be vertices of \mG\ such that \(b \wedge c = d\) and \(b \vee c = a\);
if \(b\sim d, c\sim d\) or \(a\sim b, a\sim c\), all edges between 
\(a, b, c, d\) are present in \mG\@.
\end{lemma}

\begin{proof}
If \(b\sim d, c\sim d\) we have:
\begin{eqnarray*}
b\sim d, c\sim d \Rightarrow (b \vee c) \sim (d \vee d) \Rightarrow a\sim d\\
b\sim d, d\sim c \Rightarrow (b \vee d) \sim (d \vee c) \Rightarrow b\sim c\\
b\sim c, c\sim d \Rightarrow (b \vee c) \sim (c \vee d) \Rightarrow a\sim c\\
b\sim c, d\sim b \Rightarrow (b \vee d) \sim (c \vee b) \Rightarrow b\sim a\\
\end{eqnarray*}
Analogously, if \(a\sim b, a\sim c\), all the edges above must be present in \mG\@.
\end{proof}

Next, we show that monotone graphs are closed under Cartesian product.
The Cartesian product on graphs is defined as follows:
\begin{defi} [Cartesian product on graphs]
For graphs \(G_1=(V_1,E_1)\) and \(G_2=(V_2,E_2)\), \(G = G_1 \times G_2\)
is the \emph{Cartesian product} of \(G_1\) and \(G_2\) if \(G=(V,E)\)
such that \(V=V_1 \times V_2\) and \((u_1,u_2)\sim (v_1,v_2)\) in \(E\)
if and only if \(u_1\sim v_1\) in \(E_1\) and \(u_2\sim v_2\) in \(E_2\)\@.
\end{defi}

We need the Cartesian product of the lattices, as well;
since lattices are partial orders we simply define the Cartesian product
for partial orders.
\begin{defi} [Cartesian product on partial orders]
For partial orders \(P_1= (X_1,\preceq)\) and \(P_2=(X_2,\preceq)\), 
\(P= P_1 \times P_2\) is the \emph{Cartesian product} of \(P_1\) and \(P_2\)
if \(P=(X,\preceq)\) such that \(X=X_1\times X_2\) and \((x_1,x_2) \preceq (y_1,y_2)\)
if and only if \(x_1\preceq y_1\) and \(x_2\preceq y_2\)\@.
\end{defi}

\begin{rem}
Let \(L_1=(X_1,\wedge,\vee)\) and \(L_2=(X_2,\wedge,\vee)\) be two lattices.
For \(x_1,y_1 \in X_1\) and \(x_2,y_2\in X_2\), we have :
\[(x_1,x_2)\wedge (y_1,y_2) = (x_1\wedge y_1, x_2 \wedge y_2) \]
and \[(x_1,x_2)\vee (y_1,y_2) = (x_1\vee y_1, x_2 \vee y_2).\]
\end{rem}

Lemma~\ref{lem:prod} will show that we can generate monotone graph from product of other
monotone graphs.

\begin{lemma}\label{lem:prod}
For any monotone graphs \(G_1\) and \(G_2\), \(G_1 \times G_2\) is a monotone graph.
\end{lemma}

\begin{proof}
For any two edges \(e = (u_1,u_2)(v_1,v_2)\) and \(e' =(u'_1,u'_2)(v'_1,v'_2)\) of \(G_1\times G_2\),
the edges \(u_1v_1\) and \(u'_1v'_1\) are edges of \(G_1\) and
edges \(u_2v_2\) and \(u'_2v'_2\) are edges of \(G_2\)\@.
Since \(G_1\) and \(G_2\) are monotone,
\[e_1 = u_1v_1 \wedge u'_1v'_1\]
and
\[e_2 = u_2v_2 \wedge u'_2v'_2\]
are edges of \(G_1\) and \(G_2\), respectively.
\begin{align*}
e \wedge e' &= ((u_1,u_2)\wedge(u'_1,u'_2))((v_1,v_2)\wedge(v'_1,v'_2)) \\
&= (u_1\wedge u'_1,u_2 \wedge u'_2)(v_1\wedge v'_1,v_2 \wedge v'_2) \\
&= (u_1\wedge u'_1)(v_1\wedge v'_1) \times (u_2\wedge u'_2)(v_2\wedge v'_2) \\
&= e_1 \times e_2 \\
&\in G_1 \times G_2 \\
\end{align*}
Analogously, \(e \vee e'\) is also an edge of \(G_1\times G_2\)\@. Hence, \(G_1\times G_2\)
is a monotone graph.
\end{proof}

\begin{example}\label{exm:prod}
The graph \(H_2\) in Figure~\ref{fig:pint2} is a proper interval graph because it is
closed under \(\wedge\) and \(\vee\) operators defined by the order \(a \preceq b \preceq c\)\@.
By Corollary~\ref{cor:pintmon}, every proper interval graph is monotone; hence, \(H_2\) is a monotone graph.
The graph \(H_1\) in Figure~\ref{fig:monnotpint} is \(H_2\times H_2\); thus, \(H_1\) is a monotone graph.
\begin{figure}
\center
\input{figs/ps3.pdftex}
\caption{Graph \ensuremath{H_2}}\label{fig:pint2}
\end{figure}
\end{example}

An other operation that preserves monotone graphs is retraction. Retraction is defined as:

\begin{defi} [Retraction]
Let \mH\ be a subset of \mG; a \emph{retraction} of \mG\ to \mH\ is a 
homomorphism \(r: G\to H\) such that for every \(v\in V(H)\) we have
\(r(v) = v\)\@.
\end{defi}
When there is a \emph{retraction} from \mH\ to \mG, equivalently we say
\mH\ is a \emph{retract} of \mG, and \mG\ \emph{retracts} to \mH\@.
Note that, in the definition of retraction whether \mH\ is an induced subgraph or not 
is not significant. Now, we show that retractions preserve monotone relations.

\begin{lemma}
Let \mG\ be a monotone graph; for any retraction \(r\) over \mG, the graph
\(r(G)\) is monotone.
\end{lemma}

\begin{proof}
Let \(G=(V,E)\) be a monotone graph with lattice \(L=(V, \wedge,\vee)\) and
let \(r\) be a retraction from \mG\ to \(H=(U,E')\)\@. We claim
\mH\ is monotone with lattice \(M=(U,\wedge_M, \vee_M)\) where
\(\wedge_M\) and \(\vee_M\) are defined as:
\begin{eqnarray*}
u \wedge_M v &= r(u \wedge v) \\
u \vee_M v &= r(u \vee v) \\
\end{eqnarray*}
For any two edges \(u_1 \sim v_1\) and \(u_2 \sim v_2\) in \mH,
we have \(u_1 \wedge_M u_2 = r(u_1 \wedge u_2)\) and \(v_1 \wedge_M v_2 = r(v_1 \wedge v_2)\);
since \mG\ is monotone, \((u_1 \wedge u_2) \sim (v_1 \wedge v_2)\) in \mG;
since \(r\) is a retraction, \(r(u_1 \wedge u_2) \sim r(v_1 \wedge v_2)\) in \mH\@.
\end{proof}

So far we presented several ways in which we can generate monotone graphs.
Algorithm~\ref{prc:genmon} combines them to generate monotone graphs.
\begin{algorithm}
\begin{enumerate}
\item Take \(m\) proper interval graphs \(G_1,G_2,\dotsc,G_m\)\@.
\item Let \(G' = G_1 \times G_2 \times \dotsb G_m\)\@.
\item Use a retraction \(r:G' \to G^r\) to generate \(G^r\)
\item Output \(G^r\)\@.
\end{enumerate}
\caption{Generating monotone relations}\label{prc:genmon}
\end{algorithm}

By the Lemme~\ref{lem:prod}~and~\ref{lem:retract}, Algorithm~\ref{prc:genmon} generates a monotone graph.
Next, we try to find out if Algorithm~\ref{prc:genmon} generates all monotone graphs.

Here, We need to define more parameters for partial orders.
The \emph{length} of a partial order \(P\), \(l(P)\), is the length of maximum chain in
the partial order. The \emph{width} of a partial order \(P\),
\(w(P)\), is the size of maximum unti-chain the partial order.
Note that a proper interval graph is a monotone graph with a lattice of width \(1\)\@.

We will use the Duffus and Rival Theorem to decompose a distributive lattice into 
product of chains.

\begin{theorem} [Duffus and Rival 1983 \cite{DR83}] \label{thm:DR}
Let \(L\) be a sub-lattice of a distributive lattice \(L'\) with
\(l(L)=l(L')\)\@. Then the cover graph of \(L\) is a retract of the cover graph of 
\(L'\)\@.
\end{theorem}

\begin{cor}
Every lattice of width \(m\), is a result of retraction of product of
\(m\) chains.
\end{cor}

\begin{lemma} \label{lem:retract}
Let \mG\ be a monotone graph with lattice \(L\)\@.
For a retraction \(r\) of the cover graph of \(L\), we have \(r(G)\) is a monotone graph
\end{lemma}

\begin{proof}
Let \(G=(V,E)\) be a monotone graph with lattice \(L=(V, \wedge,\vee)\) and
let \(r\) be a retraction from \(L\) to \(M =(U, \wedge, \vee)\)\@.
Take \mH\ as the subgraph of \mG\ induced by \(U\)\@.
We claim that \mH\ is a monotone graph with lattice \(M\)\@.
For every two vertices \(x,y\in U\), 
\(x\wedge y,x\vee y\in U\) because \(M\) is a lattice. For every two edges \(e_1\) and \(e_2\) of
\mH, \(e_1\wedge e_2,e_1\vee e_2\in G\) because  \mG\ is monotone and
\(e_1\wedge e_2,e_1\vee e_2\in H\) because \mH\ is an induced subgraph of \mG\@. 
\end{proof}

Now, we want generate a given monotone graph \mG\ with lattice \(L\),
using Algorithm~\ref{prc:genmon}. Let \(m\) be the width of the \(L\)\@.
Take \(m\) chains \(C_1,C_2,\dotsc,C_m\) of the same length as \(L\)\@.
Let \(G_i\) be a proper interval graph. 
Let \(L'\) be the product of the \(C_i\)s and let \(G'\) be the product of \(G_i\)s.

By Theorem~\ref{thm:DR}, there is a retraction \(r\) from the cover graph of \(L'\) to the cover graph of \(L\)\@.
Apply the retraction \(r\) over \(G'\) and generate \(G^r = r(G')\)\@.
It would be very nice if \(G^r\) was isomorphic with \mG\@. Unfortunately, \mG\ is a subset of \(G^r\)\@.

For some graphs \(G^r\), removing some of the edges from \(G^r\) results a connected 
monotone graph with the same lattice as \(G^r\)\@.
For example, consider the graph \(G'\) in
Figure~\ref{fig:erm1}\@. By applying a retraction, we get the
graph \(G^r\) in Figure~\ref{fig:erm2}\@. The graph
\mG\ in Figure~\ref{fig:erm3} is also a monotone graph with the same lattice 
as \(G^r\)\@. Thus, Algorithm~\ref{prc:genmon} can not generate \mG\@. 

\begin{figure}
\hfill
\subfigure[\ensuremath{G'}]{\input{figs/edgeremove1.pdftex}\label{fig:erm1}}\hfill 
\subfigure[\ensuremath{G^r}]{\input{figs/edgeremove2.pdftex}\label{fig:erm2}}\hfill 
\subfigure[\ensuremath{G}]{\input{figs/edgeremove3.pdftex}\label{fig:erm3}}\hfill 
\caption{Graphs \ensuremath{G'}, \ensuremath{G^r}, and \ensuremath{G}}
\end{figure}

The only remaining challenge is given \(G^r\) and \(L\), 
finding the set of edges \(E_1\) of \(G^r\) such that 
\(G^r - E\) is connected and monotone with lattice \(L\)\@.
