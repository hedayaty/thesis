\chapter{Characterization of Monotone Graphs}
In Chapter~\ref{chp:results} we showed that for monotone relations and monotone
bipartite graphs the problem \ccsp(\mH) is AP-reducible to the \cbis\ problem. In this chapter we will
characterize the monotone reflexive graphs; although the arguments are
analogous for bipartite graphs we will not cover them in this work.

\emph{Interval graphs} and \emph{proper interval graphs}
are two interesting families of reflexive graphs that are used to characterize graphs that the
list-homomorphism and mincost-homomorphism problems are polynomial time solvable for them.
Because of the connection between monotone graphs and these two graphs we will briefly introduce them
here.

\begin{defi}
A graph \(G=(V,E)\) is said to an interval graph if there is a pair of functions
\(s:V\to R\) and \(e: V\to R\) such that for every vertex \(v \in V\) 
we have \(s(v) < e(v)\) and \(uv \in E\) if and only if the intervals
\((s(u),e(u))\) and \((s(v),e(v))\) intersect.
\end{defi}

\begin{figure}
\center\input{figs/intv.pdftex}
\caption{Interval graph \ensuremath{G} and intervals \ensuremath{I}}
\label{fig:intv}
\end{figure}

Figure~\ref{fig:intv} shows an interval graph \mG\ and the set of 
corresponding intervals \(I\)\@.

\begin{defi} [L-Hom]
Let \mH\ be a fixed graph. Given an input graph \mG\ and for each vertex \(v\) of \mG\
a \emph{list} \(L(v) \subseteq V(H)\) the \emph{list-homomorphism} problem, denoted
as L-Hom(\mH), is the problem of  deciding whether or not
there exists a homomorphism \(h: G\to H\) such that for each vertex \(v\) of \mG\ 
we have \(h(v)\in L(v)\)\@.
\end{defi}

\begin{theorem} [Feder and Hell 1996 \cite{listhom}] \label{thm:lhom}
For any fixed reflexive graph \mH, the problem L-Hom(\mH) is polynomial time solvable if
\mH\ is an interval graph and otherwise, it is NP-complete.
\end{theorem}

\begin{defi}
A graph \(G=(V,E)\) is said to a proper interval graph if there is a pair of functions
\(s:V\to R\) and \(e: V\to R\) such that for every vertex \(v \in V\) 
we have \(e(v) - s(v) = 1\) and \(uv \in E\) if and only if the intervals
\((s(u),e(u))\) and \((s(v),e(v))\) intersect.
\end{defi}

\begin{figure}
\center\input{figs/pintv.pdftex}
\caption{Proper interval graph \ensuremath{G} and intervals \ensuremath{I}}
\label{fig:pintv}
\end{figure}

Figure~\ref{fig:pintv} shows a proper interval graph \mG\ and the set of corresponding 
intervals \(I\)\@.

\begin{defi} [MinHom]
Let \mH\ be a fixed graph. Given an input graph \mG\ and a \emph{cost} function
\(c: V(G) \times V(H) \to R\) the \emph{min-cost homomorphism} problem,
denoted as MinHom(\mH), is the problem of finding a homomorphism 
\(h:G\to H\) such that \(\sum_{v\in G} c(v, h(v))\) is minimum.
\end{defi}

\begin{theorem} [TODO] \label{thm:minhom}
For any fixed reflexive graph \mH, the problem MinHom(\mH) is polynomial time solvable if
\mH\ is a proper interval graph and otherwise, it is NP-complete.
\end{theorem}

\begin{defi} [Semilattice]
A \emph{semilattice} \(L=(X,\wedge,\vee)\) is a binary operators \(\wedge\) on \(X\) which is
\begin{itemize}
\item idempotent: \(x \wedge x = x\),
\item commutative: \(x \wedge y = y \wedge x\),
\item associative: \(x \wedge (y \wedge z) = (x \wedge y) \wedge z\),
\end{itemize}
\end{defi}

\begin{theorem}[TODO] \label{thm:semimin}
A reflexive graph \mH\ is an interval graph if and only if there exists
a semilattice \(L=(V(H), \wedge)\) such that \(\wedge\) is a polymorphism for \mH\@.
\end{theorem}

\begin{theorem} [TODO] \label{thm:minmax}
A reflexive graph \mH\ is a proper interval graph if there exists an ordering of 
vertices of \mH\ such that \emph{min} \and \emph{max} defined by the ordering are
polymorphisms for \mH\@.
\end{theorem}

\begin{cor}
Every reflexive proper interval graph is a monotone graph and
every reflexive monotone graph is an interval graph.
\end{cor}

\section{Reflexive Graphs}
Provide a similar argument for reflexive graphs. Show that monotone graphs are interval graphs.
Show that proper interval graphs are monotone(trivial). Give example of interval graphs that
are not monotone, also give examples of monotone graphs which are not proper interval graphs.


\section{Characterization of Monotone Graphs}
TODO write some introduction about this section. We will give a characterization of
monotone graphs. We restrict this section to non-directed graphs. Also for simplicity,
we restrict to connected graphs as well. By Theorems~\ref{thrm:pavol}~and~\ref{thrm:npcp},
\mH\ is monotone AP-reducible to \csat\ unless \mH\ is a reflexive or bipartite graph.	
In this work we only cover reflexive graphs.

Discussion about interval and proper interval graph.

\begin{example}
\mH\ is a monotone graph; however, \mH\ is not a proper interval graph.
\end{example}

\begin{example}
\mH\ is a interval graph; however, \mH\ is not a monotone graph.
\end{example}

\begin{defi} [Cover Graph]
Also mention \emph{Hess Diagram}.
\end{defi}

\begin{defi} [Length of a Lattice]
\end{defi}


Let \(L\)be a distributive lattice of length \(k\) and let \(h\) be  a homomorphism from a 
k-Cube \(Q\) to \(\mathcal{H}(L)\)\@.

TODO Continue ...
