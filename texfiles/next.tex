\chapter{Characterization of Monotone Graphs}
In Chapter~\ref{chp:results}, we showed that for monotone relations and monotone
bipartite graphs the problem \ccsp(\mH) is AP-reducible to the \cbis\ problem. In this chapter,
we will inspect the monotone reflexive graphs; although the arguments are
analogous for bipartite graphs we will not cover them in this work.

\emph{Interval graphs} and \emph{proper interval graphs}
are two interesting families of reflexive graphs that play a key role in complexity of 
list homomorphism and mincost-homomorphism problems. We are interested in these two families
because of their connections with monotone graphs. An interval graph is defined as follows:

\begin{defi}
A graph \(G=(V,E)\) is said be to an interval graph if every vertex \(v \in V\)
corresponds to an interval \(i(v)\) such that \(uv \in E\) if and only if the intervals
\(i(u)\) and \(i(v)\) intersect.
\end{defi}

\begin{figure} [h]
\hfill
\subfigure[\ensuremath{G_1}]{\input{figs/claw.pdftex}\label{fig:clawint}}\hfill 
\subfigure[\ensuremath{I_1}]{\input{figs/intv.pdftex}\label{fig:int}}\hfill 
\caption{Interval graph \ensuremath{G_1} and corresponding intervals \ensuremath{I_1}}
\end{figure}

The graph \(G_1\) in Figure~\ref{fig:clawint} referred as \emph{claw} is an interval graph
and \(I_1\) in Figure~\ref{fig:int} is the set of intervals corresponding to \(G_1\)\@.
The list homomorphism problem is defined as:

\begin{defi} [L-Hom]
Let \mH\ be a fixed graph. Given an input graph \mG\ and for each vertex \(v\) of \mG,
a \emph{list} \(L(v) \subseteq V(H)\) the \emph{list homomorphism} problem, denoted
as L-Hom(\mH), is the problem of  deciding whether or not
there exists a homomorphism \(h: G\to H\) such that for each vertex \(v\) of \mG\ 
we have \(h(v)\in L(v)\)\@.
\end{defi}

Theorem~\ref{thm:lhom} by Feder and Hell shows the connection between the list homomorphism
problem and the interval graphs.

\begin{theorem} [Feder and Hell 1996 \cite{listhom}] \label{thm:lhom}
For any fixed reflexive graph \mH, the problem L-Hom(\mH) is polynomial time solvable if
\mH\ is an interval graph and otherwise, it is NP-complete.
\end{theorem}

One way to define a proper interval graph is as follows:

\begin{defi}
A graph \(G=(V,E)\) is said to be a proper interval graph if every vertex \(v \in V\)
corresponds to an interval such that no interval is a proper subset of any other interval
and \(uv \in E\) if and only if the intervals \(i(u)\) and \(i(v)\) intersect.
\end{defi}

Alternately the proper interval graphs are defined as the graphs which can be represented 
by intervals of unit length.

\begin{figure}[h]
\hfill
\subfigure[\ensuremath{G_2}]{\input{figs/ps4.pdftex}\label{fig:pintg}}\hfill 
\subfigure[\ensuremath{I_2}]{\input{figs/pintv.pdftex}\label{fig:pint}}\hfill 
\caption{Proper interval graph \ensuremath{G_2} and corresponding intervals
\ensuremath{I_2}}\label{fig:pintgi}
\end{figure}

The graph \(G_2\) in Figure~\ref{fig:pintg} is a proper interval graph and
\(I_2\) in Figure~\ref{fig:pint} is the set of intervals corresponding to \(G_2\)\@.
The min-cost homomorphism problem is defined as:

\begin{defi} [MinHom]
Let \mH\ be a fixed graph. Given an input graph \mG\ and a \emph{cost} function
\(c: V(G) \times V(H) \to \mathbb{R}^+\) the \emph{min-cost homomorphism} problem,
denoted as MinHom(\mH), is the problem of finding a homomorphism 
\(h:G\to H\) such that \(\sum_{v\in G} c(v, h(v))\) is minimized.
\end{defi}

The cost function \(c\) indicates\todo{indicates?} the cost of assigning each vertex of \mG\ 
to a specific vertex in \mH\@. 

Theorem~\ref{thm:minhom} shows the connection between the min-cost homorphism problem 
and the proper interval graphs.

\begin{theorem} [TODO] \label{thm:minhom}
For any fixed reflexive graph \mH, the problem MinHom(\mH) is polynomial time solvable if
\mH\ is a proper interval graph and otherwise, it is NP-complete.
\end{theorem}

We show that (proper) interval graphs can be characterized by polymorphisms. 
These characterizations show the relation between (proper) interval graphs and monotone relations. 
Before giving the characterization we remind the definition of semilattice.

\begin{defi} [Semilattice]
A \emph{semilattice} \(L=(X,\wedge,\vee)\) is a binary operators \(\wedge\) on \(X\) which is
\begin{itemize}
\item idempotent: \(x \wedge x = x\),
\item commutative: \(x \wedge y = y \wedge x\),
\item associative: \(x \wedge (y \wedge z) = (x \wedge y) \wedge z\).
\end{itemize}
\end{defi}

\begin{theorem}[TODO] \label{thm:semimin}
A reflexive graph \mH\ is an interval graph if and only if there exists
a semilattice \(L=(V(H), \wedge)\) such that \(\wedge\) is a polymorphism for \mH\@.
\end{theorem}

\begin{cor} \label{cor:intmon}
Every reflexive monotone graph is an interval graph.
\end{cor}

\begin{theorem} [TODO] \label{thm:minmax}
A reflexive graph \mH\ is a proper interval graph if there exists an ordering of 
vertices of \mH\ such that \emph{min} and \emph{max} defined by the ordering are
polymorphisms for \mH\@.
\end{theorem}

\begin{cor} \label{cor:pintmon}
Every reflexive proper interval graph is a monotone graph.
\end{cor}

Theorems~\ref{thm:semimin}~and~\ref{thm:minmax} indicate similarities between
monotone graphs and (proper) interval graphs. However, monotone graphs are not
equivalent with (proper) interval graphs.

\begin{example} \label{exm:diff}
The graph \(H_1\) in Figure~\ref{fig:intnotmon} is 
an interval graph; however, \(H_1\) is not a monotone graph.
The graph \(H_2\) in Figure~\ref{fig:monnotpint} is a monotone graph;
however, the vertices \(a\), \(b\), \(c\), and \(d\) forms a claw; hence,
\(H_2\) is not a proper interval graph. 

\begin{figure}[h]
\hfill
\subfigure[\ensuremath{H_1}]{\input{figs/claw.pdftex}\label{fig:intnotmon}}\hfill 
\subfigure[\ensuremath{H_2}]{\input{figs/lattice.pdftex}\label{fig:monnotpint}}\hfill 
\caption{Graphs \ensuremath{H_1} and \ensuremath{H_2}}
\end{figure}
\end{example}


For any (proper) interval graph \mH, each induced subgraph of \mH\ is a 
(proper) interval graph. However, in the Example~\ref{exm:diff} \(H_1\)
is not a monotone graph and it is an induced 
subgraph of \(H_2\); while, \(H_2\) is a monotone graph.

\section{Finding monotone graphs}
In this section we suggest several ways to generate monotone reflexive graphs.
There are monotone graphs that can not be generated by the following methods,
we will give several examples of such graphs.
By Corollary~\ref{cor:pintmon}, every proper interval graph is a monotone graph.
Hence, proper interval graphs can be used as a base to generate monotone graphs.

\begin{lemma} \label{lem:monotone:con}
A reflexive graph \mG\ is monotone if and only if each component of \mG\ is monotone.
\end{lemma}

\begin{proof}
If \mG\ is a monotone graph with lattice order \(L\), each connected component of \mG\ is
closed under same meet and join operators defined by \(L\)\@. 

If \(G_1,G_2,\dotsc,G_k\), the connected components of \mG, are
monotone graphs with lattice orders \(L_1,L_2,\dotsc,L_k\),
then \mG\ is closed under any lattice \(L\) based on union of \(L_i\)s
and choosing any arbitrary total (or even lattice) order between the
lattices. For any two edges \(e_1\) and \(e_2\) from the same component,
by hypothesis \(e_1\wedge e_2\) and \(e_1\vee e_2\) are
edges of the same component.
For any two edge \(e_1\) and \(e_2\) from different components,
\(e_1\wedge e_2\) and \(e_1\vee e_2\) are both loops and since \mG\ is reflexive,
they are edges of \mG\@.
\end{proof}

Lemma~\ref{lem:monotone:con} indicates that any monotone graph can be generated
from its connected components. From here, all the monotone graphs are
assumed to be connected and reflexive, unless explicitly specified.

A lattice \(L=(X,\wedge,\vee)\) can be viewed as a partial order \(P=(X,\preceq)\);
\(\bot\) and \(\top\) denote the smallest and largest elements in the \(L\), 
respectively; in other words:
\[x \preceq \bot \Rightarrow x = \bot \]
\[\top \preceq x \Rightarrow x = \top \]
a \emph{cover} is a pair \((a,b)\) such that:
\[a \preceq x \preceq b \Rightarrow a = x\ \mathrm{or}\ b = x\]
A \emph{cover graph} of a partial order is a directed graph such that there is
an edge from \(a\) to \(b\) if \((a,b)\) is a cover in the partial order.

\begin{lemma}
Let \mG\ be a monotone graph with lattice ordering \(L\),
for every vertex \(v\), there is descending path from
\(v\) to \(\bot\)\@.
\end{lemma}

\begin{proof}
Since \mG\ is connected, there a path \(u_0=u,u_1,u_2,\dotsc,u_k=\bot\) from \(u\) to 
\(\bot\) in \mG\@. Let \(u'_0=u_0\) and \(u'_i = u_i\wedge u'_{i-1}\)\@.
Since \mG\ is reflexive \(u'_0u'_0=uu \in G\)\@.
\[u'_0u'_0, u_0u_1 \in G \Rightarrow u'_0u'_1 \in G\]
inductively we have
\[u'_{i-1}u'_i, u_iu_{i+1} \in G \Rightarrow u'_iu'_{i+1} \in G\]
This implies that \(W=u'_0=u,u'_1,\dotsc,u'_k=\bot\) is a descending walk from 
\(u\) to \(\bot\)\@. Some of the vertices may appear several times in a row in \(W\),
by merging them we get a descending path from \(u\) to \(\bot\)\@.
\end{proof}

\begin{cor}
For every element \(u \neq \bot\),
there is an element \(x \preceq u\) such that \(ux \in G\)\@.
\end{cor}

\begin{lemma}
Let \mG\ be a monotone graph with the lattice ordering \(L\), then every cover
of \(L\) is an edge of \mG\@.
\end{lemma}

\begin{proof}
Let \(a,b\) be a cover in \(L\) such that \(a \preceq b\)\@.
\[a \preceq b, b\neq \bot \Rightarrow \exists x \preceq b: bx \in G\]
If \(x=a\), the proof is complete; otherwise, since \((a,b)\) is a cover, we have \(x\preceq a\).
\[bx,aa \in G \Rightarrow (bx \vee aa) = ba \in G\]\@
\end{proof}

\begin{cor}
Let \mG\ be a monotone graph with lattice ordering \(L\);
for every pair of vertices that \(u\preceq v\), there is descending path from
\(v\) to \(u\)\@. 
\end{cor}

Next, we want to show that monotone graphs are closed under Cartesian products.
Remember that the Cartesian product on graphs is defined as follows:
\begin{defi} [Cartesian product on graphs]
For graphs \(G_1=(V_1,E_1)\) and \(G_2=(V_2,E_2)\), \(G = G_1 \times G_2\)
is the \emph{Cartesian product} of \(G_1\) and \(G_2\) if \(G=(V,E)\)
such that \(V=V_1 \times V_2\) and \((u_1,u_2)(v_1,v_2) \in E\)
if and only if \(u_1v_1 \in E_1\) and \(u_2v_2 \in E_2\)\@.
\end{defi}

We need the Cartesian product of the lattices as well,
since lattices are partial orders we simply define the Cartesian product
for partial orders.
\begin{defi} [Cartesian product on partial orders]
For partial orders \(P_1(X_1,\preceq)\) and \(P_2=(X_2,\preceq)\), 
\(P= P_1 \times P_2\) is the \emph{Cartesian product} of \(P_1\) and \(P_2\)
if \(P=(X,\preceq)\) such that \(X=X_1\times X_2\) and \((x_1,x_2) \preceq (y_1,y_2)\)
if and only if \(x_1\preceq y_1\) and \(x_2\preceq y_2\)\@.
\end{defi}

\begin{rem}
Let \(L_1=(X_1,\wedge,\vee)\) and \(L_2=(X_2,\wedge,\vee)\) be two lattices.
For \(x_1,y_1 \in X_1\) and \(x_2,y_2\in X_2\), we have :
\[(x_1,x_2)\wedge (y_1,y_2) = (x_1\wedge y_1, x_2 \wedge y_2) \]
and \[(x_1,x_2)\vee (y_1,y_2) = (x_1\vee y_1, x_2 \vee y_2).\]
\end{rem}

Lemma~\ref{lem:prod} will show that we can generate monotone graph from product of other
monotone graphs.

\begin{lemma}\label{lem:prod}
For any monotone graphs \(G_1\) and \(G_2\), \(G_1 \times G_2\) is also a monotone graph.
\end{lemma}

\begin{proof}
For any two edges \(e = (u_1,u_2)(v_1,v_2)\) and \(e' =(u'_1,u'_2)(v'_1,v'_2)\) of \(G_1\times G_2\),
the edges \(u_1v_1\) and \(u'_1v'_1\) are edges of \(G_1\) and
edges \(u_2v_2\) and \(u'_2v'_2\) are edges of \(G_2\)\@.
Since \(G_1\) and \(G_2\) are monotone,
\[e_1 = u_1v_1 \wedge u'_1v'_1\]
and
\[e_2 = u_2v_2 \wedge u'_2v'_2\]
are edges of \(G_1\) and \(G_2\), respectively.
\begin{align*}
e \wedge e' &= ((u_1,u_2)\wedge(u'_1,u'_2))((v_1,v_2)\wedge(v'_1,v'_2)) \\
&= (u_1\wedge u'_1,u_2 \wedge u'_2)(v_1\wedge v'_1,v_2 \wedge v'_2) \\
&= (u_1\wedge u'_1)(v_1\wedge v'_1) \wedge (u_2\wedge u'_2)(v_2\wedge v'_2) \\
&= e_1 \wedge e_2 \\
&\in G_1 \times G_2 \\
\end{align*}
Analogously, \(e \vee e'\) is also an edge of \(G_1\times G_2\)\@. Hence, \(G_1\times G_2\)
is a monotone graph.
\end{proof}

An other operation that preserves monotone graphs is retraction. Here is a reminder 
of retraction.
\begin{defi} [Retraction]
Let \mH\ be a subset of \mG\@. A \emph{retraction} of \mG\ to \mH\ is a 
homomorphism \(r: G\to H\) such that for every \(v\in V(H)\) we have
\(r(v) = v\)\@.
A homomorphism from \mG\ to a subset of \mG\ is called a retraction.
\end{defi}
When there is a \emph{retraction} from \mH\ to \mG, equivalently we say
\mH\ is a \emph{retract} of \mG, and \mG\ \emph{retracts} to \mH\@.
Note that, in the definition of retraction whether \mH\ is an induced subgraph or not 
is not significant.

\begin{lemma} \label{lem:retract}
Let \mG\ be a monotone graph with lattice order \(L\)\@.
For \(r\) a retraction of the cover graph of \(L\), we have \(r(G)\) is a monotone graph
\end{lemma}

\begin{proof}
Let \(G=(V,E)\) be a monotone graph with lattice order \(L=(V, \wedge,\vee)\) and
let \(r\) be a retraction from \(L\) to \(M =(U, \wedge, \vee)\)\@.
Take \mH\ as the induced subgraph of \mG\ by \(U\)\@.
We claim that \mH\ is a monotone graph with ordering \(M\)\@.
For every two vertex \(x,y\in U\), since \(M\) is a lattice
\(x\wedge y,x\vee y\in U\)\@. For every two edges \(e_1\) and \(e_2\) of
\mH, since \mG\ is monotone \(e_1\wedge e_2,e_1\vee e_2\in G\)
and since \mH\ is an induced subgraph of \mG, they are edges of 
\mH, as well.
\end{proof}

So far we presented several ways in which we can generate monotone graphs.
These techniques can be combined to create monotone graphs.
We start by a number of proper interval graphs, take their product and apply a retraction.
By the Lemme~\ref{lem:prod}~and~\ref{lem:retract}, this process generates a monotone graph.
Next, we try to find out if we can generate all the monotone graph with this process.

We need to define other parameters for partial orders as well.
The \emph{length} of a partial order \(P\), \(l(P)\), is the length of maximum chain in
the partial order. The \emph{width} of a partial order \(P\),
\(w(P)\), is the size of maximum unti-chain the partial order.

\begin{theorem} [Duffus and Rival 1983 \cite{DR83}] \label{thm:DR}
Let \(L\) be a sub-lattice of a distributive lattice \(L'\) with
\(l(L)=l(L')\)\@. Then the cover graph of \(L\) is a retract of the cover graph of 
\(L'\)\@.
\end{theorem}

\begin{cor}
Every lattice of width \(m\), is a result of retraction of product of
\(m\) chains.
\end{cor}

Now we want generate a given monotone graph \mG\ with lattice ordering \(L\),
with the previous process. Let \(m\) be the width of the \(L\)\@.
Take \(m\) chains \(C_1,C_2,\dotsc,C_m\) and \(m\) proper interval graphs
\(G_1,G_2,\dotsc,G_m\) such that each \(C_i\) is the ordering for
\(G_i\)\@.
Let \(L'\) be the product of the \(C_i\)s and let \(G'\)
be the product of \(G_i\)s.

By Theorem~\ref{thm:DR}, there is a retraction \(r\) from the
cover graph of \(L\) to the cover graph of \(L'\)\@.
Let \(G_r\) be \(r(G')\) by \(r\)\@.
It would be nice if by proper choices of \(G_i\)s, \(G_r\)
was isomorphic with \mG\@. Unfortunately, this is not the case.
For some graphs \(G_r\), we can remove some of the edges and
the resulting graph will be a monotone graph and we can not generate those monotone
graphs with the given process. For example, consider the graph \(G'\) in
Figure~\ref{fig:erm1}\@. By applying a retraction, we get the
graph \(G_r\) in Figure~\ref{fig:erm2}\@. The graph
\mG\ in Figure~\ref{fig:erm3} is also a monotone graph; however
\mG\ can not be generated from the previous process\@.

\begin{figure}
\hfill
\subfigure[\ensuremath{G'}]{\input{figs/edgeremove1.pdftex}\label{fig:erm1}}\hfill 
\subfigure[\ensuremath{G_r}]{\input{figs/edgeremove2.pdftex}\label{fig:erm2}}\hfill 
\subfigure[\ensuremath{G}]{\input{figs/edgeremove3.pdftex}\label{fig:erm3}}\hfill 
\caption{Graphs \ensuremath{G'}, \ensuremath{G_r}, and \ensuremath{G}}
\end{figure}
