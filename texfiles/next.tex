\chapter{Characterization of Monotone Graphs}
In Chapter~\ref{chp:results}, we showed that for monotone relations and monotone
bipartite graphs the problem \ccsp(\mH) is AP-reducible to the problem \cbis\@. In this chapter,
we will inspect the monotone reflexive graphs; although the arguments are
analogous for bipartite graphs we will not cover bipartite graphs in this part.

\emph{Interval graphs} and \emph{proper interval graphs}
are two interesting families of reflexive graphs that we shall see they play a key role in complexity of 
list homomorphism and mincost-homomorphism problems. We are interested in these two families
because of their connections with monotone graphs. An interval graph is defined as follows:

\begin{defi}
A graph \(G=(V,E)\) is said be to an interval graph if \mG\ can
be represented by a family of closed intervals such that
each vertex \(v \in V\) corresponds to an interval \(I_v\) and \(u \sim v \in E\)
if and only if the intervals \(I_u\) and \(I_v\) intersect.
\end{defi}

\begin{figure} [h]
\hspace{1cm}
\subfigure[\ensuremath{G_1}]{\input{figs/claw.pdftex}\label{fig:clawint}} \hspace{2cm}
\subfigure[\ensuremath{I_1}]{\input{figs/intv.pdftex}\label{fig:int}}\hspace{2cm}
\subfigure[\ensuremath{G_2}]{\input{figs/cs4.pdftex}\label{fig:nonint}}
\caption{Interval graph \ensuremath{G_1}, intervals \ensuremath{\mathcal{I}_1}, and non-interval graph \ensuremath{G_2}}
\end{figure}

The graph \(G_1\) in Figure~\ref{fig:clawint}, called \emph{claw}, is an interval graph
and \(\mathcal{I}_1\) in Figure~\ref{fig:int} is a possible family of intervals representing \(G_1\)\@.
The graph \(G_2\) in Figure~\ref{fig:nonint} is not an interval graph. In order to show that \(G_2\) is not
an interval graph, suppose there is family \(\{I_a, I_b, I_c, I_d\}\) of intervals representing \(G_2\)\@.
Without loss of generality, suppose \(I_a\) is the interval with the smallest starting point.
Since \(a\not \sim d\), \(I_d\) must appear after \(I_a\)\@. Since \(b\sim a\), \(b\sim d\), \(c\sim a\), and \(c\sim d\),
\(I_b\) and \(I_c\) mush intersect both \(I_a\) and \(I_d\); for that, they both must include the interval between
\(I_a\) and \(I_d\); this implies \(b\sim c\) which is a contradiction.


The list homomorphism problem is defined as:

\begin{defi} [L-Hom]
Let \mH\ be a fixed graph. Given an input graph \mG\ and for each vertex \(v\) of \mG,
a \emph{list} \(L(v) \subseteq V(H)\) the \emph{list homomorphism} problem, denoted
as L-Hom(\mH), is the problem of  deciding whether or not
there exists a homomorphism \(h: G\to H\) such that for each vertex \(v\) of \mG\ 
we have \(h(v)\in L(v)\)\@.
\end{defi}

Theorem~\ref{thm:lhom} shows a connection between the list homomorphism
problem and interval graphs.

\begin{theorem} [Feder and Hell 1996 \cite{listhom}] \label{thm:lhom}
For any fixed reflexive graph \mH, the problem L-Hom(\mH) is polynomial time solvable if
\mH\ is an interval graph and otherwise, it is NP-complete.
\end{theorem}

In addition to interval graphs we shall also focus on proper interval graphs.
Proper interval graphs are defined as follows:

\begin{defi}
A graph \(G=(V,E)\) is said be to an interval graph if \mG\ can
be represented by a family of closed intervals such that
no interval is a proper subset of another interval,
each vertex \(v \in V\) corresponds to an interval \(I_v\), and \(u \sim v \in E\)
if and only if the intervals \(I_u\) and \(I_v\) intersect.
\end{defi}

\begin{figure}[h]
\hspace{1cm}
\subfigure[\ensuremath{G_3}]{\input{figs/ps4.pdftex}\label{fig:pintg}}\hspace{2cm}
\subfigure[\ensuremath{I_2}]{\input{figs/pintv.pdftex}\label{fig:pint}}\hspace{2cm}
\subfigure[\ensuremath{G_4}]{\input{figs/claw.pdftex}\label{fig:nonpint}}
\caption{Proper interval graph \ensuremath{G_2}, intervals \ensuremath{\mathcal{I}_2}, and non-interval graph \ensuremath{G_4}}\label{fig:pintgi}
\end{figure}

The graph \(G_3\) in Figure~\ref{fig:pintg} is a proper interval graph and
\(\mathcal{I}_2\) in Figure~\ref{fig:pint} is a possible set of intervals representing \(G_3\)\@.
The graph \(G_4\) in Figure~\ref{fig:nonpint} is not a proper interval graph. In order to show that
\(G_4\) is not a proper interval graph, suppose there is family \(\{I_a, I_b, I_c, I_d\}\) of intervals representing \(G_4\)\@.
Intervals \(I_b\), \(I_c\), must \(I_d\) must intersect \(I_a\) and not intersect each other. Two of them can intersect \(I_a\)
in the beginning and the end of \(I_a\), the third interval can not intersect \(I_a\) and we have a contradiction.

The min-cost homomorphism problem is an optimization problem defined as follows:

\begin{defi} [MinHom]
Let \mH\ be a fixed graph. Given an input graph \mG\ and a \emph{cost} function
\(c: V(G) \times V(H) \to \mathbb{R}^+\) the \emph{min-cost homomorphism} problem,
denoted as MinHom(\mH), is the problem of finding a homomorphism 
\(h:G\to H\) such that \(\sum_{v\in G} c(v, h(v))\) is minimized.
\end{defi}

The value \(c(u,v)\) represents the cost of assigning the vertex \(u\) of \mG\ to the vertex \(v\) of \mH\@.
Theorem~\ref{thm:minhom} shows a connection between the min-cost homomorphism problem 
and proper interval graphs.

\begin{theorem} [Gutin et al. 2008 \cite{Gutin}] \label{thm:minhom}
For any fixed reflexive graph \mH, the problem MinHom(\mH) is polynomial time solvable if
\mH\ is a proper interval graph and otherwise, it is NP-complete.
\end{theorem}

We repeat the definition of \emph{monotone} relations from Chapter~\ref{chp:results}
for reflexive graphs.

\begin{defi} [Monotone reflexive graph]
Let \(G=(V,E)\) be a reflexive graph; \mG\ is \emph{monotone} if there exists distributive lattice order
\(L=(V, \wedge, \vee)\) such that
when \(a\sim b\) and \(c \sim d\), we have \(a \wedge c \sim b \wedge d\) and \(a \vee c \sim b \vee d\)\@.
\end{defi}

We also provide an alternate characterization for (proper) interval graphs.
These characterizations show the relation between (proper) interval graphs and monotone relations. 
A total order is a partial order that for every \(a, b\) we have either \(a \preceq b\)
or \(b \preceq a\)\@. For every total order we can define \(\wedge\) and \(\vee\) operators as:
\[a \preceq b \Rightarrow a \wedge b = a, a \vee b = b\]
Hence, we can represent a total order by \(\wedge\) and \(\vee\) operators.

\begin{theorem}[TODO] \label{thm:semimin}
A reflexive graph \(G=(V,E)\) is an interval graph if and only if there exists
a total order \(T=(V,\wedge,\vee))\) such that
when \(a\sim b\) and \(c \sim d\), we have \(a \wedge c \sim b \wedge d\)\@.
\end{theorem}

\begin{lemma} \label{lem:intmon}
Every reflexive monotone graph is an interval graph.
\end{lemma}

\begin{proof}
TODO. It is not trivial.
\end{proof}

\begin{theorem} [TODO] \label{thm:minmax}
A reflexive graph \(G=(V,E)\) is a proper interval graph if and only if there exists
a total order \(T=(V,\wedge,\vee))\) such that
when \(a\sim b\) and \(c \sim d\), we have \(a \wedge c \sim b \wedge d\) and \(a \vee c \sim b \vee d\)\@.
\end{theorem}

\begin{cor} \label{cor:pintmon}
Every reflexive proper interval graph is a monotone graph.
\end{cor}

Theorems~\ref{thm:semimin}~and~\ref{thm:minmax} indicate similarities between
monotone graphs and (proper) interval graphs. However, monotone graphs are not
equivalent with (proper) interval graphs.

\begin{example} \label{exm:diff}
The graph \(H_1\) in Figure~\ref{fig:intnotmon} is 
an interval graph; however, \(H_1\) is not a monotone graph.\todo{Prove \(H_1\) is not monotone and \(H_2\) is monotone}
The graph \(H_2\) in Figure~\ref{fig:monnotpint} is a monotone graph;
however, the vertices \(a\), \(b\), \(c\), and \(d\) forms a claw; hence,
\(H_2\) is not a proper interval graph. 

\begin{figure}[h]
\hspace{2cm}
\subfigure[\ensuremath{H_1}]{\input{figs/claw.pdftex}\label{fig:intnotmon}}\hspace{4cm}
\subfigure[\ensuremath{H_2}]{\input{figs/lattice.pdftex}\label{fig:monnotpint}}
\caption{Graphs \ensuremath{H_1} and \ensuremath{H_2}}
\end{figure}
\end{example}


For any (proper) interval graph \mH, each induced subgraph of \mH\ is a 
(proper) interval graph. However, in the Example~\ref{exm:diff} \(H_1\)
is not a monotone graph and it is an induced 
subgraph of \(H_2\); while, \(H_2\) is a monotone graph.

\section{Generating monotone graphs}
In this section we suggest several ways to generate monotone reflexive graphs.
There exist monotone graphs that can not be generated by the following methods,
we will give several examples of such graphs. This section is based on an unpublished
manuscript by P. Hell and M. Siggers (2009)\@. In this work, I have modified the
statements of the Lemme and provided alternating proofs to be more consistent with
rest of the thesis. The idea for the proofs are mostly the same.

By Corollary~\ref{cor:pintmon}, every proper interval graph is a monotone graph.
Hence, proper interval graphs can be used as a base to generate monotone graphs.

\begin{lemma} \label{lem:monotone:con}
A reflexive graph \mG\ is monotone if and only if each component of \mG\ is monotone.
\end{lemma}

\begin{proof}
If \mG\ is a monotone graph with lattice order \(L\), each connected component of \mG\ is
closed under same meet and join operators defined by \(L\)\@.

If \(G_1,G_2,\dotsc,G_k\), the connected components of \mG, are
monotone graphs with lattice orders \(L_1,L_2,\dotsc,L_k\),
then \mG\ is closed under any lattice \(L\) based on union of \(L_i\)s
and choosing any arbitrary total (or even distributive lattice) ordering between the
lattices. For any two edges \(e_1\) and \(e_2\) from the same component,
by hypothesis \(e_1\wedge e_2\) and \(e_1\vee e_2\) are
edges of the same component.
For any two edge \(e_1\) and \(e_2\) from different components,
\(e_1\wedge e_2\) and \(e_1\vee e_2\) are both loops and since \mG\ is reflexive,
they are edges of \mG\@.
\end{proof}

Lemma~\ref{lem:monotone:con} indicates that any monotone graph can be generated
from its connected components. From here, all the monotone graphs are
assumed to be connected and reflexive, unless explicitly specified.

A lattice \(L=(X,\wedge,\vee)\) can be viewed as a partial order \(P=(X,\preceq)\);
\(\bot\) and \(\top\) denote the smallest and largest elements in the \(L\), 
respectively; in other words:
\[x \preceq \bot \Rightarrow x = \bot \]
\[\top \preceq x \Rightarrow x = \top \]
a \emph{cover} is a pair \((a,b)\) such that:
\[a \preceq x \preceq b \Rightarrow a = x\ \mathrm{or}\ b = x\]
A \emph{cover graph} of a partial order is a directed graph such that there is
an edge from \(a\) to \(b\) if \((a,b)\) is a cover in the partial order.

\begin{lemma}
Let \mG\ be a monotone graph with lattice ordering \(L\),
for every vertex \(v\), there is descending path from
\(v\) to \(\bot\)\@.
\end{lemma}

\begin{proof}
Since \mG\ is connected, there a path \(u_0=u,u_1,u_2,\dotsc,u_k=\bot\) from \(u\) to 
\(\bot\) in \mG\@. Let \(u'_0=u_0\) and \(u'_i = u_i\wedge u'_{i-1}\)\@.
Since \mG\ is reflexive \(u\sim u \Rightarrow u'_0\sim u'_0\)\@.
\[u'_0\sim u'_0, u_0\sim u_1 \Rightarrow u'_0\sim u'_1\]
inductively we have
\[u'_{i-1}\sim u'_i, u_i\sim u_{i+1} \Rightarrow u'_i\sim u'_{i+1} \]
This implies that \(W=u'_0=u,u'_1,\dotsc,u'_k=\bot\) is a descending walk from 
\(u\) to \(\bot\)\@. Some of the vertices may appear several times consecutively in \(W\),
by merging them we get a descending path from \(u\) to \(\bot\)\@.
\end{proof}

\begin{cor}
Let \(G=(V,E)\) be a monotone graph with lattice ordering \(L=(V,\preceq)\); 
For every element \(u \neq \bot\),
there is an element \(x \preceq u\) such that \(u\sim x\)\@.
\end{cor}

\begin{lemma}
Let \mG\ be a monotone graph with the lattice ordering \(L\), then every cover
of \(L\) is an edge of \mG\@.
\end{lemma}

\begin{proof}
Let \(a,b\) be a cover in \(L\) such that \(a \preceq b\)\@.
\[a \preceq b, b\neq \bot \Rightarrow \exists x \preceq b: b\sim x\]
If \(x=a\), the proof is complete; otherwise, since \((a,b)\) is a cover, we have \(x\preceq a\).
\[b\sim x,a\sim a \Rightarrow (b \vee a) \sim (x \vee a) \Rightarrow b\sim a\]\@
\end{proof}

\begin{cor}
Let \(G=(V,E)\) be a monotone graph with lattice ordering \(L=(V, \preceq)\);
for every pair of vertices that \(u\preceq v\), there is descending path in \mG\ from
\(v\) to \(u\)\@. 
\end{cor}

\begin{lemma}
Let \mG\ be a monotone graph and let
\(a\), \(b\), \(c\), and \(d\) such that \(b \wedge c = d\) and \(b \vee c = a\)
If \(b\sim d, c\sim d\) or \(a\sim b, a\sim c\), we have all edges between 
\(a, b, c, d\) are present in \mG\@.
\end{lemma}

\begin{proof}
If \(b\sim d, c\sim d\) we have:
\begin{eqnarray*}
b\sim d, c\sim d \Rightarrow (b \vee c) \sim (d \vee d) \Rightarrow a\sim d\\
b\sim d, d\sim c \Rightarrow (b \vee d) \sim (d \vee c) \Rightarrow b\sim c\\
b\sim c, c\sim d \Rightarrow (b \vee c) \sim (c \vee d) \Rightarrow a\sim c\\
b\sim c, d\sim b \Rightarrow (b \vee d) \sim (c \vee b) \Rightarrow b\sim a\\
\end{eqnarray*}
Analogously, if \(a\sim b, a\sim c\) all the above edge must be in \mG\@.
\end{proof}

Next, we show that monotone graphs are closed under Cartesian products.
The Cartesian product on graphs is defined as follows:
\begin{defi} [Cartesian product on graphs]
For graphs \(G_1=(V_1,E_1)\) and \(G_2=(V_2,E_2)\), \(G = G_1 \times G_2\)
is the \emph{Cartesian product} of \(G_1\) and \(G_2\) if \(G=(V,E)\)
such that \(V=V_1 \times V_2\) and \((u_1,u_2)(v_1,v_2) \in E\)
if and only if \(u_1v_1 \in E_1\) and \(u_2v_2 \in E_2\)\@.
\end{defi}

We need the Cartesian product of the lattices as well,
since lattices are partial orders we simply define the Cartesian product
for partial orders.
\begin{defi} [Cartesian product on partial orders]
For partial orders \(P_1(X_1,\preceq)\) and \(P_2=(X_2,\preceq)\), 
\(P= P_1 \times P_2\) is the \emph{Cartesian product} of \(P_1\) and \(P_2\)
if \(P=(X,\preceq)\) such that \(X=X_1\times X_2\) and \((x_1,x_2) \preceq (y_1,y_2)\)
if and only if \(x_1\preceq y_1\) and \(x_2\preceq y_2\)\@.
\end{defi}

\begin{rem}
Let \(L_1=(X_1,\wedge,\vee)\) and \(L_2=(X_2,\wedge,\vee)\) be two lattices.
For \(x_1,y_1 \in X_1\) and \(x_2,y_2\in X_2\), we have :
\[(x_1,x_2)\wedge (y_1,y_2) = (x_1\wedge y_1, x_2 \wedge y_2) \]
and \[(x_1,x_2)\vee (y_1,y_2) = (x_1\vee y_1, x_2 \vee y_2).\]
\end{rem}

Lemma~\ref{lem:prod} will show that we can generate monotone graph from product of other
monotone graphs.

\begin{lemma}\label{lem:prod}
For any monotone graphs \(G_1\) and \(G_2\), \(G_1 \times G_2\) is a monotone graph.
\end{lemma}

\begin{proof}
For any two edges \(e = (u_1,u_2)(v_1,v_2)\) and \(e' =(u'_1,u'_2)(v'_1,v'_2)\) of \(G_1\times G_2\),
the edges \(u_1v_1\) and \(u'_1v'_1\) are edges of \(G_1\) and
edges \(u_2v_2\) and \(u'_2v'_2\) are edges of \(G_2\)\@.
Since \(G_1\) and \(G_2\) are monotone,
\[e_1 = u_1v_1 \wedge u'_1v'_1\]
and
\[e_2 = u_2v_2 \wedge u'_2v'_2\]
are edges of \(G_1\) and \(G_2\), respectively.
\begin{align*}
e \wedge e' &= ((u_1,u_2)\wedge(u'_1,u'_2))((v_1,v_2)\wedge(v'_1,v'_2)) \\
&= (u_1\wedge u'_1,u_2 \wedge u'_2)(v_1\wedge v'_1,v_2 \wedge v'_2) \\
&= (u_1\wedge u'_1)(v_1\wedge v'_1) \wedge (u_2\wedge u'_2)(v_2\wedge v'_2) \\
&= e_1 \wedge e_2 \\
&\in G_1 \times G_2 \\
\end{align*}
Analogously, \(e \vee e'\) is also an edge of \(G_1\times G_2\)\@. Hence, \(G_1\times G_2\)
is a monotone graph.
\end{proof}

An other operation that preserves monotone graphs is retraction. Here is a definition
of retraction:

\begin{defi} [Retraction]
Let \mH\ be a subset of \mG\@. A \emph{retraction} of \mG\ to \mH\ is a 
homomorphism \(r: G\to H\) such that for every \(v\in V(H)\) we have
\(r(v) = v\)\@.
A homomorphism from \mG\ to a subset of \mG\ is called a retraction.
\end{defi}
When there is a \emph{retraction} from \mH\ to \mG, equivalently we say
\mH\ is a \emph{retract} of \mG, and \mG\ \emph{retracts} to \mH\@.
Note that, in the definition of retraction whether \mH\ is an induced subgraph or not 
is not significant.

\begin{lemma} \label{lem:retract}
Let \mG\ be a monotone graph with lattice order \(L\)\@.
For \(r\) a retraction of the cover graph of \(L\), we have \(r(G)\) is a monotone graph
\end{lemma}

\begin{proof}
Let \(G=(V,E)\) be a monotone graph with lattice order \(L=(V, \wedge,\vee)\) and
let \(r\) be a retraction from \(L\) to \(M =(U, \wedge, \vee)\)\@.
Take \mH\ as the induced subgraph of \mG\ by \(U\)\@.
We claim that \mH\ is a monotone graph with ordering \(M\)\@.
For every two vertex \(x,y\in U\), since \(M\) is a lattice
\(x\wedge y,x\vee y\in U\)\@. For every two edges \(e_1\) and \(e_2\) of
\mH, since \mG\ is monotone \(e_1\wedge e_2,e_1\vee e_2\in G\)
and since \mH\ is an induced subgraph of \mG, they are edges of 
\mH, as well.
\end{proof}

So far we presented several ways in which we can generate monotone graphs.
These techniques can be combined to create monotone graphs.
We start by a number of proper interval graphs, take their product and apply a retraction.
By the Lemme~\ref{lem:prod}~and~\ref{lem:retract}, this process generates a monotone graph.
Next, we try to find out if we can generate all the monotone graph with this process.

We need to define other parameters for partial orders as well.
The \emph{length} of a partial order \(P\), \(l(P)\), is the length of maximum chain in
the partial order. The \emph{width} of a partial order \(P\),
\(w(P)\), is the size of maximum unti-chain the partial order.

\begin{theorem} [Duffus and Rival 1983 \cite{DR83}] \label{thm:DR}
Let \(L\) be a sub-lattice of a distributive lattice \(L'\) with
\(l(L)=l(L')\)\@. Then the cover graph of \(L\) is a retract of the cover graph of 
\(L'\)\@.
\end{theorem}

\begin{cor}
Every lattice of width \(m\), is a result of retraction of product of
\(m\) chains.
\end{cor}

Now we want generate a given monotone graph \mG\ with lattice ordering \(L\),
with the previous process. Let \(m\) be the width of the \(L\)\@.
Take \(m\) chains \(C_1,C_2,\dotsc,C_m\) and \(m\) proper interval graphs
\(G_1,G_2,\dotsc,G_m\) such that each \(C_i\) is the ordering for
\(G_i\)\@.
Let \(L'\) be the product of the \(C_i\)s and let \(G'\)
be the product of \(G_i\)s.

By Theorem~\ref{thm:DR}, there is a retraction \(r\) from the
cover graph of \(L\) to the cover graph of \(L'\)\@.
Let \(G^r\) be \(r(G')\) by \(r\)\@.
It would be nice if by proper choices of \(G_i\)s, \(G^r\)
was isomorphic with \mG\@. Unfortunately, this is not the case.

For some graphs \(G^r\), we can remove some of the edges and
the resulting graph will be a monotone graph and we can not generate those monotone
graphs with the given process. For example, consider the graph \(G'\) in
Figure~\ref{fig:erm1}\@. By applying a retraction, we get the
graph \(G^r\) in Figure~\ref{fig:erm2}\@. The graph
\mG\ in Figure~\ref{fig:erm3} is also a monotone graph; however
\mG\ can not be generated from the previous process\@.


\begin{figure}
\hfill
\subfigure[\ensuremath{G'}]{\input{figs/edgeremove1.pdftex}\label{fig:erm1}}\hfill 
\subfigure[\ensuremath{G^r}]{\input{figs/edgeremove2.pdftex}\label{fig:erm2}}\hfill 
\subfigure[\ensuremath{G}]{\input{figs/edgeremove3.pdftex}\label{fig:erm3}}\hfill 
\caption{Graphs \ensuremath{G'}, \ensuremath{G^r}, and \ensuremath{G}}
\end{figure}

Only remaining challenge is deciding which of the \(G^r\) are mandatory 
and must be present in \mG\ and which of the edges are optional.
We already proved that the edges of the lattice are mandatory.
Our guess is that, the edge \(ad \in G^r\) is mandatory if and only
all the pairs of vertices \(b,c \in G'\) such that \(b\wedge c = d\)
and \(b \vee c = a\) (assuming \(d \preceq a\)) are present in \mG\@.
