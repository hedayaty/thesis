%% Copyright 1998 Pepe Kubon
%%
%% `abstract.tex' --- abstract for thes-full.tex, thes-short-tex from
%%                    the `csthesis' bundle
%%
%% You are allowed to distribute this file together with all files
%% mentioned in READ.ME.
%%
%% You are not allowed to modify its contents.
%%

%%%%%%%%%%%%%%%%%%%%%%%%%%%%%%%%%%%%%%%%%%%%%%%%%
%
%       Abstract 
%
%%%%%%%%%%%%%%%%%%%%%%%%%%%%%%%%%%%%%%%%%%%%%%%%

\prefacesection{Abstract}
Constraint satisfactions is a framework to express combinatorial problems.
\ccsp\ is the problem of finding the number of solutions for a constraint satisfaction problem
instance. In this work, we study complexity of approximately solving the \ccsp\@. We
provide several techniques for approximation preserving reductions
among counting problems. Most of this work is around the \cbis,
the problem of finding the number of independent sets in a bipartite graph.

We prove that approximately solving \ccsp\ over relations, we call them \emph{monotone},
is not harder than \cbis\@. We also prove that approximately
solving \ccsp\ over relations, we call them \emph{RBA} (reflexive oriented asymetric), is harder than \cbis\@.
Finally, we invetigate monotone relations in reflexive graphs.

\vspace{1cm}

\noindent \textbf{Keywords:} \ccsp, Approximation, AP-reduction, FPRAS







