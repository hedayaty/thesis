%% Copyright 1998 Pepe Kubon
%%
%% `abstract.tex' --- abstract for thes-full.tex, thes-short-tex from
%%                    the `csthesis' bundle
%%
%% You are allowed to distribute this file together with all files
%% mentioned in READ.ME.
%%
%% You are not allowed to modify its contents.
%%

%%%%%%%%%%%%%%%%%%%%%%%%%%%%%%%%%%%%%%%%%%%%%%%%%
%
%       Abstract 
%
%%%%%%%%%%%%%%%%%%%%%%%%%%%%%%%%%%%%%%%%%%%%%%%%

\prefacesection{Abstract}
Constraint satisfactions is a powerful framework to express many combinatorial problems. 
\ccsp\ is the problem of finding the number of solutions for a constraint satisfaction problem instance.
\ccsp\ can be used to express many computational problems such as finding the number homomorphisms
between two graphs, sampling, and computing the partition functions.
Partition functions have many applications in statistical mechanics,
game theory, and economics. Many of the parameters of a system in Ising model and Potts model 
are expressed with the partition functions.

In this work we mostly focus on complexity of approximately solving the \ccsp\@.
We provide several techniques that
can be used to provide approximation preserving reductions among counting problems.
Next we will focus on 
a complexity class that contains problems that with respect to approximation preserving reductions
are as hard as the problem of finding the number of independent
sets in a bipartite graph. We prove that approximately solving \ccsp\ over relations, we call them
\emph{monotone}, is not harder than the problem of
finding the number of independent sets in a bipartite graph. We also prove that
approximately solving \ccsp\ over relations, we call them \emph{\RBA},
are harder than the problem of finding the number of independent sets in a bipartite graph.

\vspace{1cm}

\noindent \textbf{Keywords:} \ccsp, Approximation, AP-reduction, FPRAS







