\chapter{Introduction}
Constraint Satisfaction is a powerful framework to express many combinatorial problems.
The Constraint Satisfaction Problem (CSP) is the problem of deciding if it is possible to
assign values to some variables such that given constraints are satisfied. \ccsp\ 
is the problem of finding the number of possible assignments for a set of constraints.

Many computational problems in mathematics, economics, and statistical mechanics
can be expressed by \ccsp\@. Graph homomorphism, sampling,
and partition functions are among notions in mathematics
that are closely related to \ccsp\@. Computation of partition function plays a key role
in dealing with problems involving Ising and Potts models. These models are
commonly used in statistical mechanics and game theory.
 
In real world applications, there are many problems that are not efficiently solvable;
however, usually an approximate solution is satisfactory for them. 
In order to find out which problems can be dealt with efficiently, we classify them
by complexity of approximately solving them.

Generally solving the \ccsp\ is a hard problem, however if we restrict the
\ccsp\ to a set of relations \mrelset, which is denoted by \ccsp(\mrelset),
depending on the \mrelset\ the complexity of the problem may vary.
In this work we study the complexity of finding approximate solutions for the 
problem \ccsp(\mrelset).
We will introduce some of methods that can be used to approximate counting problems.

Bulatov \cite{Bulatov} has proved a dichotomy for complexity of the
problem \ccsp(\mrelset). This dichotomy implies that 
for some \mrelset, the problem \ccsp(\mrelset) can be solved in polynomial time
and for the rest if one of them can be solved in polynomial time
all of them can be solved in polynomial time.
Feder and Vardi \cite{cspconj} have conjectured that \ccsp(\mrelset) also exhibits of dichotomy of
this sort. This conjecture has motivated many researchers \cite{dur628, recent08, valeriote}.
With respect to approximation, more complexity classes are expected for the problem \ccsp(\mrelset).
The problem of finding the number of 2-Colorings of a graph can be solved 
in polynomial time. There is the problem of finding the
number of independent sets in a bipartite graph often referred as \cbis\@.
The problem of finding the number of independent sets in a general graphs often referred as
the \cisp\ is among a set of problems which if one of them can be approximately solved
in polynomial time then for all \mrelset, the problem \ccsp(\mrelset) can be approximately
solved in polynomial time.
Goldberg et al. \cite{Trichotomy} have proved that for Boolean \mrelset,
there are three complexity classes for \ccsp(\mrelset)\@. These three classes are:
problems that can be approximately solved in polynomial time, such as finding the
number of 2-Colorings in a graph; problems as hard as the problem \cbis; and
problems that are hard to approximate, such as the problem \cisp\@.

In this work, we introduce some useful techniques to find approximation preserving reductions
among counting problems. We will introduce \emph{monotone} relations
and show that the problem \ccsp(\mrelset) over these families of relation is not harder that
the problem \cbis\@. We will also show that the problem \ccsp(\mrelset) for reflexive oriented graphs 
is at least as hard as the problem \cbis\@. We will also investigate reflexive monotone graphs.

\section{Thesis Organization}
In Chapter~2 we formally define CSP and \ccsp\ and their applications.
In Chapter~3 we define reductions for counting problems and mention major results
on complexity of the problem \ccsp(\mrelset). In Chapter~4 we define approximation preserving
reductions and the known classes for approximated counting. We mention several problems
from each class; we also mention several methods used for approximate counting problems.
In Chapter~5 we introduce our own methods used for approximation preserving reductions.
In Chapter~6 we mention our results on the complexity of approximating the problem \cbis\@.
In Chapter~7 we investigate reflexive monotone graphs.